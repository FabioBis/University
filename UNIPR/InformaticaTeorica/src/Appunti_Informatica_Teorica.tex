%% Notes for Theoretical Computer Science Lectures (italian)
%
%  To generate the pdf file:
%
%  1. latex Appunti_Informatica_Teorica.tex
%  2. dvipsf Appunti_Informatica_Teorica.dvi

\hyphenation {
  ab-bia-mo
  esp-res-sio-ne
  es-ten-sio-na-li-t
  for-ma-liz-za
  in-du-zio-ne
  in-sie-me
  in-sie-mi-sti-ca-men-te
  in-sie-mi-sti-che
  in-ter-pre-ta-zion-ne
  in-ve-ce
  pre-mes-se
  pri-mi-ti-va
  rap-pre-sen-tan-te
  S--esp-res-sio-ne
  vi-sio-ne
  }

\documentclass{book}

\usepackage{latexsym}
\usepackage[italian]{babel}
\usepackage{graphicx}
\usepackage{amsmath,amssymb}
\usepackage{fancyhdr}

%%--Ambienti
\newtheorem{definizione}{Def.}[chapter]
\newtheorem{osservazione}{Oss.}[chapter]
\newtheorem{teorema}{Teorema.}[chapter]
\newtheorem{lemma}{Lemma.}[chapter]
\newtheorem{dimostrazione}{Dim.}[chapter]
\newtheorem{corollario}{Corollario}[chapter]
\newtheorem{esempio}{Esempio.}
\newtheorem{ese}{Esercizio.}
\newtheorem{notazione}{Notazione.}

%%--Comandi / Macro
\newcommand*{\Dom}{\mathop{\mathrm{Dom}}\nolimits}
\newcommand*{\II}{$\mathbf{I}$}    %Text mode
\newcommand*{\KK}{$\mathbf{K}$}    %Text mode
\newcommand*{\SSS}{$\mathbf{S}$}   %Text mode
\newcommand*{\CC}{$\mathbf{C}$}    %Text mode
\newcommand*{\BB}{$\mathbf{B}$}    %Text mode
\newcommand*{\WW}{$\mathbf{W}$}    %Text mode
\newcommand*{\ii}{\mathbf{I}}    %Math mode
\newcommand*{\kk}{\mathbf{K}}    %Math mode
\newcommand*{\sss}{\mathbf{S}}   %Math mode
\newcommand*{\cc}{\mathbf{C}}    %Math mode
\newcommand*{\bb}{\mathbf{B}}    %Math mode
\newcommand*{\ww}{\mathbf{W}}    %Math mode
\newcommand*{\lbc}{$\lambda$-cal\-co\-lo}
\newcommand*{\lbt}{$\lambda$-ter\-mi\-ne}
\newcommand*{\lbts}{$\lambda$-ter\-mi\-ni}
\newcommand*{\asse}{\vdash\!\!           }
\newcommand*{\asset}{\vdash^T\!\!}

%% A proof tree implementation.
\input prooftree

%% Arrows and so on.
\def\ddp{\hbox{\kern-.5em\lower.63ex\hbox{${}^\succ$}}}
\def\bijection{\mbox{$\mapstochar\,\mapsto\ddp$}}
\def\surjection{\mbox{$\to\ddp$}}
\def\injection{\mbox{$\mapstochar\,\mapsto$}}


\renewcommand{\chaptermark}[1]{\markboth{#1}{}}
\renewcommand{\sectionmark}[1]{\markright{\thesection\ #1}}
\renewcommand{\headrulewidth}{0.3pt}
\renewcommand{\footrulewidth}{0pt}

\addtolength{\headheight}{25.5pt}

\fancypagestyle{plain}{%
  \fancyhead{} % ignora, nello stile plain, le intestazioni
  \renewcommand{\headrulewidth}{0pt} % e la linea
}

\pagestyle{fancy}
\fancyhf{}
\fancyhead[LE,RO]{\bfseries\thepage}
\fancyhead[LO]{\bfseries\rightmark}
\fancyhead[RE]{\bfseries\leftmark}


\author{Fabio Biselli}
\title{Appunti di Informatica Teorica\\ (Prof. Ricci a.a. 2005--2006)}
\date {}

\frenchspacing

\begin{document}

\maketitle

\tableofcontents

\chapter{Introduzione al calcolo dei combinatori}

%%--Capitolo 1
\section{Funzioni: due punti di vista}

\begin{enumerate}
\item \emph{visione estensionale:} insieme di coppie  $\langle x, y\rangle$

\[\includegraphics{img/infteofig.1}\]


\item \emph{visione operativa:} la funzione $y = x^{2} - x$  viene vista come
un programma:
prendi $x$ e calcolane il quadrato, sottrai quindi $x$ ed ottieni $y$.
\end{enumerate}

Nelle funzioni aritmetiche e nell'analisi le due visioni si equivalgono,
ma vedremo che non \`e sempre cos\`i.

\begin{esempio}
La funzione pi\`u semplice \`e la funzione identit\`a $\mathbf{I}(x) = x$.
Nella visione operativa questa, dato un argomento $x$, restituisce
l'argomento stesso. La visione estensionale invece formalizza l'insieme
$\bigl\{\, \langle x, x \rangle \mid x \in X \text{ per qualche }X\,\bigr\}$.
\end{esempio}

%%--Visione Estensionale
\subsection{Visione estensionale}Nella Teoria degli Insiemi (secondo il testo
consigliato, J.~D.~Monk, Introduzione alla teoria degli insiemi, Boringhieri, 
1972.) i postulati formalizzano la nozione di appartenenza,
indicata dal simbolo `$\in$' ($x \in X$); da essa derivano quella di
insieme (``insieme piccolo'') e di classe propria (``insieme
grande''). Gli insiemi piccoli sono quegli oggetti $x$ che stanno a
sinistra del simbolo di appartenenza cio\`e gli $x$ tale che $x$
appartiene a $X$ per qualche $X$. Le classi sono quegli oggetti che
stanno a destra cio\`e gli $X$ per cui $x$ appartiene a $X$ per
qualche $x$. Se una classe non \`e un insieme si dice propria. Un
caso di tali classi proprie \`e quello del nostro $\mathbf{I}$.

Per le funzioni esiste la nozione di applicazione di un argomento $x$ (che
pu\`o anche essere una funzione) a una funzione, ottenendo un insieme $f(x)$.
Quando $x$ non appartiene all'insieme degli argomenti effettivi di $f$,
cio\`e al dominio $\Dom f = \bigl\{\,  x \mid \langle x, y \rangle \in f 
\text{ per qualche } y \,
\bigr\}$, il valore di $f(x)$ \`e un ? che insiemisticamente \`e espresso
nella classe propria totale $V$:
\[
V = \{\, x \mid  x \text{ \`e un insieme} \,\}.
\]
Pertanto $\mathbf{I}(\mathbf{I}) = V$ poich\'e $\mathbf{I}$ non appartiene al
$\Dom \mathbf{I}$ (altrimenti non sarebbe classe propria).
Si noti che $\mathbf{I} \neq V$.


Consideriamo ora l' idea della funzione identit\`a dal punto di vista
operativo: lascia le cose come sono.
Tale \emph{funzione operativa} $\mathbf{I}$ avr\`a la propriet\`a che:
\begin{center}
$\mathbf{I}(\mathbf{I}) = \mathbf{I}$.
\end{center}
Gi\`a nella funzione pi\`u semplice le due visioni sono differenti.

%%--Visione Operativa
\subsection{Visione operativa}

Esistono solo regole operative (non oggetti come insiemi etc.). Tali regole
sono formalizzate da come operano fra loro. Ci\`o avviene in due modi
distinti: calcolo dei combinatori e $\lambda$-calcolo.

Nel calcolo dei combinatori si assumono delle funzioni operative elementari
con cui si costruiscono tutte le altre tramite una operazione detta di
applicazione e si introducono delle nozioni di uguaglianza tra tali funzioni
operative dipendenti da nozioni di calcolo effettivo dette \emph{riduzione}.
Nel $\lambda$-calcolo non si assumono delle funzioni elementari, nel senso che
esse siano da sole operativamente definite, ma solo degli oggetti su cui si
opera con una ``applicazione''  e un'astrazione che sar\`a in un certo senso
l'opposto dell'applicazione.

Nel primo modo si costruiscono cos\`i degli oggetti, termini combinatori, che 
corrispondono ai programmi di un linguaggio di programmazione (imperativo) 
idealizzato che operi con delle funzioni predefinite (elementari) mediante 
applicazione di``dati'' a una procedura. Tali dati possono essere procedure.
Non sono ammessi richiami di sottoprocedure con parametri.

Nel $\lambda$-calcolo i $\lambda$-termini invece non hanno funzioni predefinite
 da cui partire, avranno invece le ``call'' (procedure con parametri
richiamabili).


%%--Combinatori
\section{Calcolo dei combinatori}

Si assume l'esistenza di una sequenza di cosiddette costanti che contenga
almeno tre elementi: $\mathbf{I}$, $\mathbf{K}$, $\mathbf{S}$.
Si assume l'esistenza di una sequenza numerabile infinita di cosiddette
variabili: $x$, $y$, $z$, $u$, $v$, etc.
Per atomo si intender\`a sia una costante che una variabile, i termini
combinatori sono definiti dalle seguenti clausole che usano un'operazione di
applicazione fra due termini $X$ e $Y$ che definiremo dopo e che indicheremo
ora con l'accostamento $ XY $:

\begin{enumerate}
\item[a)]ogni atomo \`e un termine combinatorio;
\item[b)]se $X$, $Y$ sono due termini combinatori, allora anche l'applicazione
$XY$ \`e un termine combinatorio.
\end{enumerate}

L'applicazione \`e definita dai seguenti postulati, ove per termine composto
si intende un termine che sia l'applicazione di altri due termini combinatori:
\begin{enumerate}
\item[1)]Le costanti, le variabili ed i termini composti formano tre 
``insiemi''disgiunti.
\item[2)]Se due termini composti sono le stesso termine, allora cos\`i \`e
ordinatamente per i termini componenti:
\begin{equation}
X' Y' \equiv  X'' Y'' \Longrightarrow X' \equiv X''  \text{ e } Y' \equiv Y''
\end{equation}
\end{enumerate}

Si intende inoltre che nessun altro oggetto \`e un termine combinatorio se
non proviene dalle clausule a) e b).

Stiamo usando ``$\equiv$''  perch\`e riserviamo ``$=$'' a delle equivalenze 
che, pur non essendo l'identit\`a, saranno pi\`u frequenti. Stiamo usando la 
parola insieme nel senso insiemistico, ma potremmo farne a meno.
Insiemisticamente potremmo creare un modello dei termini combinatori prendendo
come applicazione l'accoppiamento e come atomi opportuni insiemi.
(L'accoppiamento insiemistico gode di maggiori propriet\`a di quelle
dell'applicazione, ma queste non ci servono).

Sempre (ab)usando nozioni insiemistiche potremmo, indicando con $T$
l'insieme dei termini composti, concludere che:

\begin{enumerate}
\item[-]ogni atomo \`e un termine combinatorio;
\item[-]se $X'Y' = X''Y''$ allora  $ X'=X''$ e $ Y'=Y'' $;
\item[-]per ogni $T'$ contenuto in $T$ vale che $T' = T$ se e solo se ogni 
  atomo appartiene a $T'$ e se $X'$, $Y'$ appartengono a $T'$ allora $X'Y'$ 
  appartiene  a $T$.
\end{enumerate}

Tale lista di propriet\`a corrisponde a quella dei postulati di Peano:

\begin{enumerate}
\item[-]0 \`e un numero naturale;
\item[-]se $n+1 = m+1$ allora $n = m$;
\item[-]per ogni $N'$ contenuto in $N$ vale che  $N' = N$ sse $(0 \in N'$,
$n \in N' \Rightarrow  n+1 \in N')$.
\end{enumerate}

Le due liste di propriet\`a differiscono ``di poco'':
\begin{enumerate}
\item[$\bullet$]i numeri naturali hanno un solo zero, mentre i termini
combinatori ne possono avere molti (gli atomi).
\item[$\bullet$]i numeri naturali provengono dallo zero mediante la funzione
successore che \`e una operazione unaria (ad un argomento), mentre per i
termini combinatori si usa l'applicazione che \`e binaria (a due argomenti).
\end{enumerate}
Per il resto nulla cambia.

%Capitolo 2 - Calcolo dei Combinatori

\chapter{Calcolo dei combinatori}
Dall'analogia relativa ai postulati di Peano tra Termini Combinatori e numeri
naturali possiamo dire che nella maggior parte dei casi per definire una
funzione o un predicato, relativi ai termini combinatori, procederemo con una
base di definizione (le funzioni o predicati vengono definiti per tutti gli
atomi) e un passo induttivo (se la funzione o il predicato si suppongono
definiti per due termini combinatori si da la costruzione per definirli
sull'applicazione dei due termini).

Lo stesso varr\`a per le dimostrazioni.
Nella base della dimostrazione si dimostra l'asserto per gli atomi, nel passo
induttivo, presunta la verit\`a dell'asserto per due termini, si dedurr\`a
l'asserto per la loro applicazione.

Per i termini combinatori conviene, sopratutto all'inizio, considerarne la
rappresentazione grafica mediante alberi binari con foglie etichettate dagli
atomi.

\begin{esempio}Alcune rappresentazioni grafiche.
\begin{figure}[!ht]
\[ \includegraphics{img/infteofig.2} \]
\caption{Rappresentazione grafica di $x\kk$.}

\[ \includegraphics{img/infteofig.3} \]
\caption{Rappresentazione grafica di $(\sss(\kk\kk))y$.}
\end{figure}

\end{esempio}

Nella definizione di termini combinatori non compaiono parentesi. Le parentesi
che scriviamo sono solo sintattiche (servono ad indicare le precedenze nella
costruzione). Invece nella notazione dell'Analisi (espressa nella Teoria degli
Insiemi) le parentesi possono servire ad indicare l'applicazione (insiemistica)
 di un argomento ad una funzione: $f(x)$, $ [M_{v}]_{s} $, etc.
Ci\`o fa s\`i che nella nostra notazione ci siano meno parentesi di quelle
dell'Analisi. Tuttavia restano nei termini combinatori pi\`u grossi un buon
numero di parentesi.

Si pu\`o allora adottare una convenzione di semplificazione di scrittura che
le riduce considerevolmente pur evitando ambiguit\`a:
\emph{in ogni termine o sottotermine si omettono le coppie di parentesi
associate a sinistra.}

\begin{esempio} $(\sss(\kk\kk))y$ diventa $\sss(\kk\kk)y$. \end{esempio}

Data una scrittura semplificata si pu\`o sempre risalire univocamente alla
scrittura non semplificata o meglio all'albero.
\begin{esempio}
Ecco graficamente come vegono rappresentati i seguenti termini:


\begin{figure}[!ht]
\[ \includegraphics{img/infteofig.4} \]
\caption{ $abcxy\kk\ii xa \longrightarrow (((((((ab)c)x)y)\kk)\ii)x)a$.}
\end{figure}

\begin{figure}[!ht]
\[ \includegraphics{img/infteofig.5} \]
\caption{$abc(xy\kk)(\ii x)a \longrightarrow (((ab)c)(((xy)\kk))(\ii x))a$.}
\end{figure}

\begin{figure}[!ht]
\[ \includegraphics{img/infteofig.6} \]
\caption{$\sss(\kk(\sss(xa)))$ resta invariata.}
\end{figure}

\end{esempio}

%%--Operazioni
\section{Operazioni su termini combinatori}
In questa parte del Capitolo parleremo delle operazioni possibili sui termini
combinatori, in particolare della sostituzione, sostituzioni multiple,
riduzione, astrazione e riduzione multipla.

%%--Sostituzione
\subsection{Sostituzione}
\begin{definizione}
Per i termini combinatori si pu\`o definire un'operazione di
\emph{sostituzione} che da una variabile e due termini combinatori
restituisce un altro termine combinatorio.
Siano $N$ ed $M$ termini combinatori ed $x$ una variabile, \emph{la
sostituzione di $N$ al posto di $x$ in $M$} \`e il termine combinatorio
$[N/x]M$ definito da:
\begin{enumerate}
\item$[N/x]a \equiv a$ se $a$ \`e un atomo $\neq x$;
\item$[N/x]x \equiv N;$
\item$[N/x]UV \equiv ([N/x]U)([N/x]V)$.
\end{enumerate}
Talvolta base e passo induttivo sembrano mescolati.
\end{definizione}

%%--Occorrenza
\begin{definizione}
Diciamo che un termine combinatorio $U$ \emph{occorre} in un termine
combinatorio $X$ e scriviamo $U \in X$ (non \`e un'appartenenza) qualora ci\`o
pu\`o essere dedotto da:
\begin{enumerate}
\item[a)]$U \in U$;
\item[b)]Se $U \in X$ o se $U \in Y$ allora $ U \in XY$.
\end{enumerate}
\end{definizione}

Anche questa \`e una definizione induttiva del predicato $U \in X$ con
induzione su $U$. Infatti in $a)$ si d\`a (anche) la base ($U$ pu\`o essere
un'atomo) mentre il passo induttivo viene dato sia da $b)$ che da $a)$.

\begin{esempio}
Vediamo esempi di sostituzioni e di $\in$.


\vspace{0.5 cm}Sia  $M \equiv S(KK)(xI)$ e calcoliamo la sostituzione di 
$[yS/x]M$
\[
\begin{array}{lll}
[yS/x]M & \stackrel{3}{\equiv} & ([yS/x]S(KK))([yS/x]xI) \\
& \stackrel{3}{\equiv} & (([yS/x]S)([yS/x]KK))([yS/x]xI) \\
& \stackrel{1}{\equiv} & (S([yS/x]KK))([yS/x]xI)  \\
& \stackrel{3}{\equiv} & (S(([yS/x]K)([yS/x]K)))([yS/x]xI) \\
& \stackrel{1}{\equiv} & (S(KK)([yS/x]xI) \\
& \stackrel{3}{\equiv} & (S(KK)(([yS/x]x)([yS/x]I))) \\
& \stackrel{2}{\equiv} & (S(KK)((yS)([yS/x]I))) \\
& \stackrel{1}{\equiv} & (S(KK)((yS)I)) \\
& \equiv &  S(KK)(ySI).\\
\end{array}
\]
Lo si sarebbe potuto individuare subito utilizzando la rappresentazione 
grafica:
\begin{figure}[!ht]
\[ \includegraphics{img/infteofig.7} \]
\caption{Rappresentazione grafica di $M$.}

\[ \includegraphics{img/infteofig.8} \]
\caption{Rappresentazione grafica di $[y\sss/x]M$.}
\end{figure}
\end{esempio}

\begin{osservazione}
La definizione di sostituzione riguarda solo atomi $x$ che siano variabili
(\`e questa la ragione della distinzione fra variabili e costanti). In
particolare non si possono sostituire termini al posto delle costanti
riducibili. Come vedremo in seguito anche  una tale ``sostituzione'' pu\`o
avere significato.
\end{osservazione}

\begin{ese}Verifichiamo se $\kk\kk \in \sss(\kk\kk)(x\ii)$ e se
$\kk\kk(x\ii) \stackrel{?}{\in} \sss(\kk\kk)(x\ii)$.

\begin{enumerate}
\item[1)]$\kk\kk \in \sss(\kk\kk)(x\ii).$
\item[]per $b)\ \kk\kk \stackrel{?}{\in} \sss(\kk\kk)$ oppure $\kk\kk
\stackrel{?}{\in} x\ii$.
\item[]Per quanto riguarda l'occorrenza di destra potremmo andare
avanti come prima, ma non potremmo concludere che vi sia. Andiamo quindi ad
analizzare l'occorrenza di sinistra:
\item[]per $b)\ \kk\kk \stackrel{?}{\in} \sss$ oppure $\kk\kk
\stackrel{?}{\in} \kk\kk$.
\item[]$\kk\kk \notin \sss$ poich\`e $\sss$ \`e atomo e $\kk\kk$ \`e termine
composto.
\item[]$\kk\kk \in \kk\kk \longrightarrow$ si, per la clausola $a)$.
\item[]Questo corrisponde al fatto che $\kk\kk$ \`e sottoalbero di
$\sss(\kk\kk)(x\ii)$.
\item[2)]$\kk\kk(x\ii) \stackrel{?}{\in} \sss(\kk\kk)(x\ii).$
\item[]Possiamo fare la deduzione come sopra, ma deduciamo subito
graficamente che non c'\`e occorrenza.

\begin{figure}[!ht]
\[ \includegraphics{img/infteofig.9} \]

\[ \includegraphics{img/infteofig.10} \]
\caption{$\kk\kk(x\ii)$ non \`e sottoalbero di $\sss(\kk\kk)(x\ii)$.}
\end{figure}
\end{enumerate}
\end{ese}

\begin{osservazione}
In $xS(xS)$ abbiamo due verifiche che $xS \in xS(xS)$, per indicare quale dei
due sottoalberi ci interessa parleremo di occorrenze di $xS$ in $xS(xS)$.
\end{osservazione}

%%--Sostituzione Multipla
\subsection{Sostituzione multipla}
In luogo di una variabile si possono contemporaneamente sostituire pi\`u
variabili. Vediamolo nel caso di due variabili.
\begin{definizione}Nel caso di due variabili definiamo la sostituzione di un
termine combinatorio $N'$ al posto della variabile $x$ e di un termine
combinatorio $N''$ al posto della variabile $y$ in un termine $M$ ed
indicheremo il risultato con $[N'/x, N''/y]M$ mediante le seguenti clausole:
\begin{enumerate}
\item[-]$[N'/x, N''/y]a  \equiv  a \quad$ se $a$ \`e un atomo $\neq x, y$;
\item[-]$[N'/x, N''/y]x  \equiv  N'$;
\item[-]$[N'/x, N''/y]y  \equiv N''$;
\item[-]$[N'/x, N''/y]UV  \equiv ([N'/x, N''/y]U)([N/x,  N''/y]V)$.
\end{enumerate}
Doppia induzione.
\end{definizione}

%--Riduzione
\subsection{Riduzione}
\begin{definizione}
Diremo che il termine combinario $X$ si \emph{riduce (debolmente)} al termine
combinatorio $Y$ e scriveremo $X \vartriangleright Y$, qualora ci\`o possa
essere dedotto dai seguenti assiomi.
\end{definizione}
\begin{enumerate}
\item[(I)]$\quad \ii X  \vartriangleright X$;
\item[(K)]$\quad \kk XY  \vartriangleright X$;
\item[(S)]$\quad \sss XYZ  \vartriangleright XZ(YZ)$;
\item[($\rho$)]$\quad X \vartriangleright X$.
\end{enumerate}
Per ogni termine combinatorio $X$, $Y$ e $Z$ utilizziamo le seguenti regole di 
deduzione:
\begin{enumerate}
\item[-]Se $X  \vartriangleright Y \quad \text{ allora } \quad XZ
\vartriangleright YZ$;
\item[-]se $X  \vartriangleright Y \quad \text{ allora } \quad ZX
\vartriangleright ZY$;
\item[-]se $X  \vartriangleright Y \ e \ Y  \vartriangleright Z
\quad \text{ allora } \quad X  \vartriangleright Z$.
\end{enumerate}

%--Definizione di redex
\begin{definizione}[redex]
Diremo che un termine di tipo $\ii X$, $\kk XY$, $\sss XYZ$ \`e un \emph{redex}
 e  che i corrispondenti termini $X$, $X$ e $XZ(YZ)$ sono i rispettivi ridotti 
(o contratti) immediati.

L' uso di uno dei tre  assiomi $(I)$, $(K)$, $(S)$ sar\`a detto contrazione.
\end{definizione}

%%--Definizione di Forma Normale
\begin{definizione}
Se in un termine combinatorio $Y$ non occorrono dei redex allora diremo che
$Y$ \`e  in \emph{forma normale} e se $X  \vartriangleright Y$, diremo che
$Y$ \`e una forma normale di $X$.
\end{definizione}

\begin{esempio}
Consideriamo un termine combinatorio $\sss(\kk\sss)\kk$ che chiamiamo $\bb$ e 
con essocostruiamo il termine generico $\bb XYZ$ con $X$, $Y$, $Z$ termini 
qualsiasi. 
Si ha:
\end{esempio}

\[
\begin{array}{lcl}
\bb XYZ & \equiv & \underbrace{\sss(\kk \sss)\kk X}_{\text{redex}}YZ \\
     & \stackrel{(S)}{\vartriangleright} & \underbrace{\kk\sss X}_{
       \text{redex}}(\kk X)YZ \\
     & \stackrel{(K)}{\vartriangleright} & \underbrace{\sss (\kk X)YZ}_{
       \text{redex}} \\
     & \stackrel{(S)}{\vartriangleright} & \underbrace{\kk XZ}_{
       \text{redex}}(YZ)  \\
     & \stackrel{(K)}{\vartriangleright} & X(YZ). \\
\end{array}
\]


L'esempio precedente mostra una caratteristica desiderabile per considerare
``$\vartriangleright$'' come un procedimento di calcolo: almeno fino al ridotto
$X(YZ)$ il risultato del calcolo era unico (perch\`e ad ogni passo intermedio
del calcolo la contrazione era unica). Si noti che la riduzione $BXYZ
\vartriangleright X(YZ)$ non \`e una contrazione, ma pu\`o comunque essere
usata come tale per le regole di deduzione in $2.4$. 

La distingueremo con 
``$\stackrel{(B)}{\vartriangleright}$''.


\begin{esempio} Sia $\cc \equiv \sss(\bb\bb\sss)(\kk\kk)$ per $X$, $Y$, $Z$ 
generici, si ha:\vspace{0.5 cm}

\[
\begin{array}{lcl}
\cc XYZ & \equiv & \underbrace{\widehat \sss\widehat{(\bb\bb\sss)}
       \widehat{(\kk\kk)}\widehat{X}}_{\text{redex}}YZ \\
     & \stackrel{(S)}{\vartriangleright} & \underbrace{\bb\bb\sss X}_{
       \text{redex}}\underbrace{(\kk\kk X)}_{redex}YZ \\
     & \stackrel{\stackrel{(B)}{\text{scelta}}}{\vartriangleright} & 
       \underbrace{\bb(\sss X)\underbrace{(\kk\kk X)}_{\text{redex}}YZ}_{\text{
       \text{redex}}} \\
     & \stackrel{\stackrel{(K)}{scelta}}{\vartriangleright} & 
       \bb(\sss X)\kk YZ \\
     & \stackrel{(B)}{\vartriangleright} & \sss X(\kk Y)Z \\
     & \stackrel{(S)}{\vartriangleright} & XZ(\kk YZ) \\
     & \stackrel{(K)}{\vartriangleright} & XZY; \\
\end{array}
\]
che pu\`o essere in forma normale. Per\`o contrariamente all'
esempio precedente, le contrazioni scelte non erano le uniche possibili.
Vediamo cosa succede cambiando le scelte:

\[
\begin{array}{lcl}
\underbrace{\bb\bb\sss X}_{\text{redex}}\underbrace{(\kk\kk X)}_
{\text{redex}}YZ &
  \stackrel{\stackrel{(K)}{\text{scelta}}}{\vartriangleright} & 
  \bb\bb\sss X\kk YZ \\
& \stackrel{(B)}{\vartriangleright} & \bb(\sss X)\kk YZ \\
& \stackrel{(B)}{\vartriangleright} & \sss X(\kk Y)Z \\
& \stackrel{(S)}{\vartriangleright} & XZ(\kk YZ) \\
& \stackrel{(K)}{\vartriangleright} & XZY. \\
\end{array}
\]

Lo stesso succeder\`a per la seconda scelta.
\end{esempio}

Quesito: accadr\`a sempre che seguendo strade diverse si possa raggiungere
sempre il medesimo risultato?

\begin{esempio}Sia $J \equiv  \sss\kk\kk$.

\[
JX \equiv \sss\kk\kk X \stackrel{(S)}
{\vartriangleright} \kk X(\kk X) \stackrel{(K)}{\vartriangleright}
 X.
\]

\end{esempio}
Cio\`e $JX  \vartriangleright  X$ cos\`i come ($\ii$), quindi potremmo fare a
meno della $\ii$.


%%-- 2 Lemmi
\begin{lemma}
Se la variabile $x \notin X$ allora $x \notin Y,\ \forall\ Y$ tale che $X
\vartriangleright Y$.
\end{lemma}
\begin{lemma}
Se $X \vartriangleright Y$ allora $[N/X]X \vartriangleright [N/X]Y$ e lo
stesso vale per le sostituzioni multiple (la dimostrazione \`e abbastanza
ovvia per induzione).
\end{lemma}

%%--Definizione combinatori
\vspace{0.3 cm}\begin{definizione}
Mentre nei termini combinatori possono occorrere atomi diversi dalle costanti
irriducibili, nei tre esempi notevoli precedenti solo le costanti riducibili
vi occorrono.
Pertanto i termini (come $B$, $C$, $J$) in cui atomi diversi da tali costanti
non occorrono si dicono \emph{combinatori}.
\end{definizione}

\subsection{Astrazione}
%%--Definizione astrazione
\begin{definizione}[astrazione]
L'\emph{astrazione} di un termine combinatorio M rispetto ad una variabile $x$
indicata con $[x]M$, \`e definita per induzione su $M$ da:
\begin{enumerate}
\item[a)]$[x]x \equiv I$;
\item[b)]$[x]M \equiv KM$, \quad per $x \notin M$;
\item[c)]$[x]Ux \equiv U$, \quad per $x \notin U$;
\item[d)]$[x]UV \equiv S([x]U)([x]V)$, \quad altrimenti.
\end{enumerate}
\end{definizione}

\begin{esempio}
Astrazione dei termini $yx$ e $xy$ rispetto ad $x$:
\begin{enumerate}
\item[-]$[x]yx \quad \stackrel{c)}{\equiv} \quad y,\quad$  (per $y \neq x$);
\item[-]$[x]xy \quad \stackrel{d)}{\equiv} \quad S([x]x)([x]y) \quad
\stackrel{a b)}{\equiv} \quad SI(Ky)$.
\end{enumerate}
\end{esempio}

%%--Teorema della riduzione(semplice) e Dimostrazione
\subsection{Riduzione semplice}
\begin{teorema}[riduzione semplice]
Per ogni variabile $x$ e per tutti i termini combinatori $M$ ed $N$ si ha che:
\[
\bigl([x]M\bigr)N \ \vartriangleright \ [N/x]M.
\]
\end{teorema}

%%--Dimostrazione
\begin{dimostrazione}[riduzione semplice]
Osserviamo dapprima che \`e sufficiente dimostrare l'asserto nel caso $N
\equiv x$, cio\`e dimostrare che:
\[
\bigl([x]M\bigr)x \ \vartriangleright \ [x/x]M \equiv M.\qquad  (1)
\]
Come si vede facilmente dalla definizione di sostituzione, possiamo limitarci
a tale $N$ poich\`e da $X \vartriangleright Y$ segue che $[N/X]X
\vartriangleright [N/X]Y\quad \forall N$. Infatti nel nostro caso ci\`o ci
porterebbe all'asserto, poich\`e $x \notin [x]M$ (come segue dalla definizione
di astrazione) e da ci\`o viene che $[N/x]\bigl([x]M\bigr)x \equiv \bigl([x]M
\bigr)N$.

 Per dimostrare la $(1)$ procediamo per induzione combinatoria su $M$
raggruppando i casi in corrispondenza della definizione di astrazione.

\begin{enumerate}
\item[$\bullet$]Caso $M \equiv x$.
   \item[] $[x]M  \equiv  \ii$, per cui:

 $([x]M)x
\equiv  \ii x  \stackrel{(I)}{\vartriangleright}  x \equiv
 M. \qquad$ da cui la 1)
\item[$\bullet$]Caso $x \notin M$.
    \item[]$[x]M \equiv  \kk M$,  per cui:

 $([x]M)x \equiv  \kk Mx  \stackrel{(K)}{\vartriangleright}  M. \qquad$ da
cui la 1)
\end{enumerate}

Esaurita cos\`i la base.
\vspace{0.3 cm}

\begin{enumerate}
\item[$\bullet$]Caso $M \equiv Ux$ con $x \notin U$.
    \item[]$[x]M \equiv  U$,  per cui:

 $([x]M)x \equiv  Ux  \stackrel{(\rho)}{\vartriangleright} M \qquad$
da cui la 1)
\item[$\bullet$]Caso $M \equiv UV$ non gi\`a trattato, cio\`e $x \in U
$ oppure $ ( x \neq V$ e $x \in V$).In questo caso composto servir\`a
l'ipotesi del passo induttivo, cio\`e :

\[
 ([x]U)x \vartriangleright U\text{ e }([x]V)x \vartriangleright V.
\qquad  (2)
\]

\item[]$[x]M \equiv  S([x]U)([x]V)$, per cui:

 $([x]M)x  \equiv  \sss ([x]U)([x]V)x \stackrel{(S)}{\vartriangleright}  
([x]U)x(([x]V)x) \stackrel{(2)}{\equiv}  UV 
 \equiv  M$.
\end{enumerate}

\end{dimostrazione}

%%--Teorema \text{redex}Riduzione Multipla
\subsection{Riduzione multipla}
\begin{teorema}
Per tutte le variabili $x', x'',$ $\ldots$ $, x^{(n)}$ e per tutti i termini
combinatori $M, N', N'',$ $\ldots,$ $N^{(n)}$ si ha che:
\begin{center}
$([x', x'', $ $\ldots$ $, x^{(n)}]M)N'N''$ $\ldots$ $N^{(n)} \
\vartriangleright \ [N'/x', N''/x'', $ $\ldots$ $ N^{(n)}/x^{(n)}]M$,
\end{center}
qualora le $x^{(i)}$ siano distinte e $[x', x'',$ $\ldots$ $, x^{(n)}]$
indichi l'astrazione iterata $[x']([x'']($ $\ldots$ $[x^{(n)}]N$ $\ldots$ $))$.
\end{teorema}

\begin{esempio}
Purch\`e le variabili siano ancora distinte l'astrazione multipla pu\`o
ridefinire i nostri combinatori, cio\`e le costanti riducibili.
\begin{enumerate}
\item[-]$[x]x \equiv \ii$;
\item[-]$[x, y]x \equiv [x]([y]x) \equiv [x]\kk x \equiv \kk$;
\item[-]$[x, y, z]xz(yz) \equiv [x,y]([z]xz(yz)) \equiv [x,y]\sss([z]xz)([z]yz)
\equiv [x, y](\sss xy) \equiv$
\item[]$\equiv [x]([y]\sss xy) \equiv [x]\sss x \equiv \sss$.
\end{enumerate}
\end{esempio}
La dimostrazione del teorema precedente proviene facilmente da quello per la
riduzione semplice usando le seguenti osservazioni.
\begin{osservazione}
Il lemma della conservazione della sostituzione da parte della
``$\vartriangleright$'' vale anche nel caso multiplo.
\begin{center}
$X \vartriangleright Y \quad \Longrightarrow \quad [N'/x', $ $\ldots$ $ N^{(n)}
/x^{(n)}]X \quad \vartriangleright \quad [N'/x', $ $\ldots$ $
N^{(n)}/x^{(n)}]Y$.
\end{center}
\end{osservazione}

%%--Lemma
\begin{lemma}\label{Lc02n1}
Se $v \notin N$, allora $[v]([v/x]N) \equiv [x]N$, cio\`e non importa come si
chiamano le variabili d'astrazione.
\end{lemma}

\begin{lemma}\label{Lc02n2}
Se $x \not\equiv y$ e $y \notin N$, allora $[N/x]([y]M) \equiv [y]([N/x]M)$ 
per ogni termine $N$ ed $M$.
\end{lemma}


\subsection{Uso del teorema di astrazione}

Mentre la ``$\vartriangleright$'' ci definisce un processo di calcolo tra 
termini combinatori, il teorema di riduzione ci permette di trovare dei termini
combinatori aventi certe propriet\`a di calcolo. Infatti, se vogliamo trovare
un termine combinatorio $X$ tale che:
\[
XN' \ldots N^{(n)} \vartriangleright \underbrace{[N'/x', \ldots ,
N^{(n)}/x^{(n)}]Y}_{da\ pensarsi\ assegnato}\ ,
\]
ci baster\`a per il teorema di riduzione nel caso multiplo calcolare:
\[
[x', \ldots , x^{(n)}]Y \equiv X.
\]

\begin{esempio}Cerchiamo un $X$ tale che per ogni $u$, $v$ e $z$:
\[
XUVZ \vartriangleright U(VZ) \equiv [U/u, V/v, Z/z]\underbrace{u(vz)}_{Y}.
\]

Possiamo trovare un tale $X$ come $[u, v, z]u(vz)$. abbiamo cos\`i:

\[
\begin{array}{lcl}
[u,v,z]u(vz)\ & \equiv\ & [u,v]([z]u(vz)) \\ 
& \equiv & [u, v](\sss([z]u)([z]vz)) \\ 
& \equiv & [u,v]\sss(\kk u)v \\
& \equiv & [u]([v]\sss(\kk u)v) \\ 
& \equiv & [u]\sss(\kk u) \\ 
& \equiv & \sss([u]\sss)([u]\kk u) \\ 
& \equiv & \sss(\kk\sss)\kk .
\end{array}
\]

Che \`e il nostro $\bb$.
\end{esempio}

%%--Section Propriet\`a di riduzione ed uguaglianza debole
\section{Propriet\`a di riduzione ed uguaglianza debole}


%%--Teorema Church-Rosser
\begin{teorema}[Church-Rosser I]
Per ogni termine combinatorio $X, Y, Z$ se $X$ si riduce a $Y$ ($X
\vartriangleright Y$) e $X$ si riduce a $Z$ ($X \vartriangleright Z$), allora
esiste un termine combinatorio $U$ tale che $Y \vartriangleright U$ e $ Z
\vartriangleright U$
\begin{figure}[!h]
\[ \includegraphics{img/infteofig.11} \]
\caption{Il calcolo anche se segue strade diverse pu\`o convergere.}
\end{figure}
\end{teorema}

%%--Dimostrazione
\begin{dimostrazione}
No. (induzione)
\end{dimostrazione}

(Il calcolo, anche se segue strade diverse, pu\`o convergere. Si parla di 
propriet\`a di Church-Rosser per relazioni come la nostra 
``$\vartriangleright$'', non per funzioni)

\begin{osservazione}
Si noti che $U$ non \`e detto che sia in forma normale. Quindi tale propriet\`a
esprime un'unicit\`a di ``risultato'', anche quando il risultato non c'\`e.
\end{osservazione}

\begin{corollario}
L'eventuale forma normale di un termine combinatorio \`e\emph{ unica}.
\end{corollario}

%%--Dimostrazione
\begin{dimostrazione}
Supponiamo che $X \vartriangleright Y, Z$ ove $Y$ e $Z$ sono in forma normale.
Poich\`e esiste $U$ tale che $Y, Z \vartriangleright U$ e poich\`e $Y$, $Z$
non hanno alcun redex, tali ultime riduzioni non sono contrazioni, cio\`e
provengono dall'assioma di riflessivit\`a ($\rho$).
Pertanto $Y \equiv U \equiv Z$ cio\`e per trasitivit\`a $Y \equiv Z$.
\end{dimostrazione}

\begin{definizione}
Diremo che $X$ \`e \emph{(debolmente) uguale}, e scriveremo $X = Y$ qualora
ci\`o si possa dedurre come nella definizione di riduzione ( con $=$ al posto
di $\vartriangleright$) con l'aggiunta della regola di deduzione di simmetria:
\[
X = Y \Longrightarrow Y = X
\]
Cos\`i facendo si intuisce subito che l'uguaglianza debole \`e la chiusura
simmetrica della riduzione. Si pu\`o anche facilmente vedere che:

\begin{enumerate}
  \item[-]Se $X \vartriangleright Y$ allora esiste una catena finita di
    contrazioni che porta da $X$ a $Y$, cio\`e esistono $X', X'', \ldots
    X^{(n)}$ tali che $X \equiv X' \vartriangleright, \ldots \vartriangleright
    X^{(n)} \equiv Y$, ove le ``$\vartriangleright$'' sono contrazioni.
  \item[-]Se $X = Y$ allora esiste una catena finita di contrazioni e
    contrazioni inverse, che porta da $X$ a $Y$ come sopra.
\end{enumerate}
\end{definizione}

%%--Teorema Church-Rosser II
\begin{teorema}[Church-Rosser II]
Se $X = Y$ allora esiste $Z$ tale che $X \vartriangleright Z$ e $Y
\vartriangleright Z$.
\end{teorema}

%%--Dimostrazione
\begin{dimostrazione}
Consideriamo una catena di contrazioni e contrazioni inverse da $X$ a $Y$,
come nel precedente lemma, di lunghezza $m$ e procediamo per induzione
aritmetica su $n = m+1$.
\begin{enumerate}
  \item[$\bullet$]$n = 0$, non ci sono contrazioni $X \vartriangleright Y$ n\`e
    $Y \vartriangleright X$, per l'asserto ($\rho$) $X \equiv Y$. Pertanto
    basta prendere $Z \equiv X, Y$ e usare ($\rho$).
  \item[$\bullet$]Supponiamo l'asserto vero per catene di lunghezza $n$ e
    consideriamo una catena di lunghezza $n+1$.
    Avremo allora che esiste uno $Z'$ tale che $X \vartriangleright Z'$ e
    $X^{(n)} \vartriangleright Z'$ mentre potranno esserci due casi: $X^{(n)}
    \stackrel{contraz.}{\vartriangleright} Y\ $, $\ Y
    \stackrel{contraz.}{\vartriangleright} X^{(n)}$.

  \item[-]Caso $X^{(n)} \vartriangleright Y$ si ha:
  \begin{figure}[!ht]
    \[ \includegraphics{img/infteofig.12} \]
    \caption{Le frecce indicano contrazioni (dirette ed inverse).}
  \end{figure}  

    Pertanto per transitivit\`a della $\vartriangleright$ si ha l'asserto.

  \item[-]Caso $Y \vartriangleright X^{(n)}$,
    basta solo la transitivit\`a:
  \begin{figure}[!ht]
    \[ \includegraphics{img/infteofig.13} \]
    \caption{\`E sufficiente prendere $Z \equiv Z'$.}
  \end{figure}  
\end{enumerate}
\end{dimostrazione}

%%--Corollario A
\begin{corollario}[A]
Se $X = Y$ e $Y$ \`e in forma normale, allora $X \vartriangleright Y$.
\end{corollario}

%%--Dimostrazione
\begin{dimostrazione}
Per il teorema di \emph{Church-Rosser II} esiste $Z$ tale che $X,Y
\vartriangleright Z$. Nella $Y \vartriangleright Z$ non ci pu\`o essere
contrazione, cio\`e si usa ($\rho$).
\end{dimostrazione}

%%--Corollario B
\begin{corollario}[B]
Se $X = Y$ o $X$ e $Y$ non hanno forma normale oppure hanno la stessa forma
normale.
\end{corollario}

%%--Dimostrazione
\begin{dimostrazione}
Consideriamo la prima alternativa e procediamo per assurdo, cio\`e supponiamo
che $X$ abbia forma normale mentre $Y$ non l'abbia (o viceversa).

Sia $Z$ la forma normale di $X$ allora per corollario A si ha che $Z$ \`e
forma normale anche di $Y$ poich\`e $Z \vartriangleleft X = Y$. Pertanto o
vale la prima alternativa o c'\`e questa comune forma normale che (per
Corollario di Church-Rosser I) \`e unica.
\end{dimostrazione}

%%--Corollario C
\begin{corollario}[C]
Se $X$ e $Y$ sono in forma normale, allora il fatto che $X\ \not\equiv Y$ 
implica che $X \neq Y$.
\end{corollario}

%%--Dimostrazione
\begin{dimostrazione}
Supponiamo per assurdo che $X = Y$. Applichiamo il Corollario B, cio\`e
otterremmo che $X \equiv Y$ che \`e assurdo.
\end{dimostrazione}

\vspace{0.5 cm}(Corollario di ``consistenza'' logica del calcolo dei
combinatori relativamente all'uguaglianza debole. Infatti esistono ovvi
termini combinatori in forma normale che non sono identici.)


%%--Sezione 2.3 ``Interpretazione utile CdC''
\section{Interpretazione ``utile'' del calcolo dei combinatori}
L'interpretazione ``utile'' considera i termini combinatori come delle funzioni
insiemistiche, contrariamente al fatto che non lo possono essere, come visto
nella diversit\`a per combinatore identit\`a e classe (insieme grande)
identit\`a.

%%--Sottosezione Richiami Teoria degli Insiemi
\subsection{Richiami sulla teoria degli insiemi}
Nella formulazione del testo consigliato di Teoria degli Insiemi le
funzioni sono particolari insiemi e gli insiemi delle classi costruite
con degli assiomi a partire dallo zero (nei combinatori abbiamo invece
parecchi atomi).

Che $\emptyset$ sia l'unico insieme atomico \`e una conseguenza dell'assioma
di estensionalit\`a (che dice che due insiemi, non importa come costruiti,
sono lo stesso insieme se e solo se hanno gli stessi elementi).

Fra i primi insiemi definibili ci sono le coppie $\langle a, b \rangle$.
Insiemi di coppie sono detti relazioni. Da una relazione $r$ si definiscono il
suo dominio:
\[
\Dom R = \bigl\{\,a \mid \langle a, b \rangle\ \in\ r\ \,\bigr\};
\]
e la controimmagine del dominio $\bigl\{\,b \mid \langle a, b \rangle\ \in\ 
r\ \,\bigr\}$
(tali oggetti sono definiti \emph{dopo} la relazione $r$).

Una relazione si dice funzione se soddisfa la condizione di funzionalit\`a:
\[
\langle a, b'\rangle ,\ \langle a, b'' \rangle \in f \Longrightarrow b' = b''.
\]
La scritta $f \colon A \to B$ significa che $f$ \`e una funzione, che
$A$ \`e il suo dominio ($\Dom f$) e che $B \supseteq \bigl\{\, b \mid\ \langle
a, b\rangle \in\ f \,\bigr\}$ cio\`e che gli ``argomenti'' di $f$
costituiscono $A$ e che i suoi ``valori'' stanno in $B$.

Con $r^{-1}$ si intende la conversa di una relazione cio\`e $\bigl\{\,
\langle b, a \rangle \mid\ \langle a, b\rangle \in\ r \, \bigr\}$. Se per una
funzione $f\colon A \to B$ si ha che $f^{-1}$ sia funzione scriviamo
$f: A \injection B$.
Se $B$ \`e la controimmagine del dominio scriviamo $f: A \surjection B$
(suriezione su $B$) e se vogliamo entrambe scriviamo $f: A \bijection B$ 
(biiezione da $A$ su $B$).


\subsection{``Tipi'' delle funzioni insiemistiche}
L'asserzione $f: A \longrightarrow B$ serve per definire l'insieme
\emph{potenza fra due insiemi}:
\[
B^{A} = \{ f \mid f\colon A \to B \}.
\]
pertanto $f: A \longrightarrow B$ sse $f \in B^{A}$.
\begin{description}
\item[Nota:] Sul testo consigliato si usa mettere l'esponente sulla
sinistra ${}^{A}\textrm{B}$ poich\`e conserva l'ordinamento.
\end{description}
Vi sono poi altre costruzioni, per esempio il \emph{prodotto}
di una funzione $A: I \longrightarrow B$.
\[
\prod_{i \in I}A_{i} = \biggl\{\,f \mid  f: I \longrightarrow
\bigcup_{i \in I}A_{i}\  e \ f_{i} \in A_{i} \,\biggr\}
\]
Siccome questo prodotto \`e un sottoinsieme della potenza la cui base \`e
$\bigcup_{i \in I}A_{i} \subseteq \cup B$
\[
\prod_{i \in I}A_{i} \subseteq \Biggl(\ \bigcup_{i \in I}A_{i}\ \Biggr)^{I}
\subseteq \left(\cup B\right)^I,
\]
questa costruzione \`e secondaria. Noi consideremo principalmente la potenza.
Iterando le potenze si possono ottenere strutture di potenze: $(C^{B})^A$,
$C^{B^{A}}$, \ldots

\begin{esempio}
\[
f(x) = \int_{a}^{x},
\]
 inteso come funzione della funzione itegranda $g$, \`e una funzione $f$ tale 
che:

\[
f \in (R^R)^{R^{R}},
\]


l'esponente al di fuori delle parentesi \`e dovuto alla $g$
tale che $\int_{a}^{x}g(y) \, \mathrm{d} x = f(x)$,
ovvero $f \in (R^{R^{R}})^R$ se si considera la $x$ come prima variabile.
Pertanto avremo a che fare con delle funzioni $F: A \longrightarrow B$ dove
$A$ e $B$ saranno costruzioni insiemistiche complesse, fra cui tali potenze
iterate.

Diremo che la parte destra dell'asserzione $F: A \longrightarrow B$ \`e il
tipo insiemistico di $F$. Quindi il tipo \`e una potenza in cui base ed
esponente possono essere complessi.
\end{esempio}

%%--Interpretazione Insiemistica delle Costanti Riducibili
\subsection{Interpretazione Insiemistica delle costanti riducibili}
L'interpretazione ``arborea'' finora considerata \`e quella corretta.
\begin{figure}
\[
\begin{array}{ccc}
\includegraphics{img/infteofig.14}  & \vartriangleright & \dot{x} \\
\\

\includegraphics{img/infteofig.15}  & \vartriangleright & \dot{x} \\
\\

\includegraphics{img/infteofig.16}  & \vartriangleright &
\includegraphics{img/infteofig.17} \\
\end{array}
\]
\caption{Interpretazione ``arborea'' delle costanti riducibili.}
\end{figure}
L'interpretazione utile \`e invece quella che vede negli $\ii$, $\kk$, $\sss$ 
delle funzioni insiemistiche che partono sempre dagli assiomi riferiti
all'uguaglianza debole vista come uguaglianza insiemistica.

$\ii X = X$ mi dice che $\ii$ \`e identit\`a, 
possiamo (sbagliando) pensarla come identit\`a sulla classe $V$ di tutti gli 
insiemi, allora: $\ii\colon V \to V$ oppure come identit\`a su un 
insieme qualsiasi $\ii\colon A \to A$

Si noti che nel secondo caso si ha anche che $\ii\colon A \to B$ per 
ogni $B \supseteq A$.

\vspace{0.5 cm}Consideriamo $\kk XY = X$, insiemisticamente ci\`o ci dice che
abbiamo una funzione $\kk_{X}$ tale che $\kk_{X}(Y) = X$. Quindi tale $\kk_{X}$
 \`e una funzione costante e $\kk_{X}: A \longrightarrow B$ ove $A \ni Y$ e 
$B \ni X$.

Pertanto con tali $A$ e $B$ si ha che $\kk\colon B \to B^A$ e $K$ ha come
valori delle costanti, in definitiva $K$ \`e insiemisticamente un generatore
di costanti.

\vspace{0.5 cm}Consideriamo $\sss XYZ = XZ(YS)$, aiutiamoci con la scrittura 
delle applicazioni insiemistiche:
\[
[\sss(X)]_{Y}(Z) = X_{Z}(Y_{Z}).
\]
Quindi considerato $Y_Z$ $Y: C \longrightarrow B$ ove $C \ni Z$ e 
$X_{Z}\colon B \to A$, da cui $X \colon C \longrightarrow A^B$.

\begin{esempio}Definiamo i sopracitati insiemi $C$, $X$ e $Y$ nel seguente 
modo:
\[
C = \{0, 1, 2, \} \quad
Y = 
\begin{tabular}{|r|l|}
  \hline
  $Z$ & $Y_{Z}$ \\
  \hline
  0 & 5 \\
  1 & 4 \\
  2 & 2 \\
  \hline
\end{tabular}
\qquad
X = 
\begin{tabular}{|r|l|}
  \hline
  $Z$ & $X_{Z}$ \\
  \hline
  $0$ & successore \\
  $1$ & $\sqrt{\phantom{1}}$ \\
  $2$ & dimezzamento \\
  \hline
\end{tabular}
\]

Cosicch\`e $X_{Z}Y_{Z}$ vale $6$, $2$ o $1$ a seconda che $Z$ sia $0$, $1$ o
$2$. In sostanza $Z$ \`e un indice per dei vettori $X$ e $Y$. Il vettore $X$
\`e un vettore di funzioni (come operazioni unarie) e $Y$ lo \`e di numeri.

\[
\begin{tabular}{|c|c|c|c|}
  \hline
  $X_{Z}$ & successore & $\sqrt{\phantom{1}}$ & dimezzamento \\
  \hline
  $Y_{Z}$ & $5$ & $4$ & $2$ \\
  \hline
  $X_{Z}Y_{Z}$ & $6$ & $2$ & $1$ \\
  \hline
\end{tabular}
\]

\vspace{0.3 cm}
L'ultimo vettore dei risultati \`e stato ottenuto applicando ``in parallelo''
$Y$ ad $X$. Tale vettore corrisponder\`a a $SXY$ poich\`e $(SXY)Z = XZ(YZ)$
ovvero riscrivendo: $([S(X)]_{Y})(Z) = X_{Z}(Y_{Z})$.\\Analogamente $SX$ overo
$S(X)$ corrisponder\`a all'operatore che consente tale operazione parallela ad
$X$ di un vettore di argomenti qualsiasi poich\`e $[S(X)]_{y}$ con $y$
variabile f\`a ci\`o.

Ma allora $S$ non \`e altro che l'operatore \emph{parallelizzazione} che da
$X$ ci da il precedente $S(X)$ per $X$ vettore di operazioni unarie qualsiasi.
\end{esempio}

\vspace{0.3 cm}
Osserviamo ora i tipi:
\begin{displaymath}
\text{Per le notazioni precedenti } \left\{ \begin{array}{l}
Z \in C; \\
Y: C \longrightarrow B; \\
X: C \longrightarrow A^B.
\end{array}\right.
\end{displaymath}
Se consideriamo $[S(X)]_{Y}: C \longrightarrow A$, poich\`e $[S(X)]_{Y}(Z) =
XZ(YZ)$, allora:
\[
S(X): B^C \longrightarrow A^C \text{ \ poich\`e\ }[S(X)]_{Y}: C 
\longrightarrow A \text{ \  e\ } Y \in B^C
\]
\begin{center}e\end{center} 
\[
S: (A^B)^C \longrightarrow (A^C)^{B^C}\text{ \ poich\`e\ }X \in (A^B)^C\
\text{ e\ }S(X) \in (A^C)^{B^C}.
\]
Quanto fatto ora pu\`o essere ripetuto per altri termini.

Consideriamo $\bb$ e $\bb XYZ = X(YZ)$. Sia $A \ni Z$ e $Y\colon A
\to B$ ove $B$ sia il dominio di $X$. Pertanto detto $C$ un
insieme che contenga i valori di $X$ si ha che $X$ sar\`a funzione che va da
$B$ in $C$ ($X\colon B \to C$). Per la nostra relazione di uguaglianza
``$=$'' $\bb XY$ non \`e altro che la composizione funzionale di $Y$ ed $X$, 
cio\`e $\bb XY = X \circ Y$ e $\bb XY \colon A \to C$.

Pertanto $\bb X$ \`e l'operatore che compone una funzione arbitraria $Y$ con 
$X$ e si pensa che $\bb X\colon B^A \to C^A$ poich\`e $Y \in B^A$ e $BXY \in
C^A$, in definitiva si pensa a:
\begin{displaymath}
B: C^B \longrightarrow (C^A)^{B^A};
\end{displaymath}
il quale \`e un operatore di composizione funzionale invertito (in $BXY =
 X \circ Y$ il primo \`e $Y$ anche se si scrive $(X \circ Y)(Z) = (Y(Z))$.
 ``Ragione storica'' per notazione convenzionale dell'applicazione 
insiemistica nata tre secoli fa.

%%--Teorema del punto fisso
\section{Teorema del punto fisso}
Ricordiamo che una funzione insiemistica  $f$ si dice abbia un punto fisso
qualora esista nel suo dominio un $a$ tale che $f(a) = a$.
Le funzioni insiemistiche possono comunque non avere punti fissi; per esempio
nel campo reale $f(x) = x + 1$ ci da una funzione che ne \`e priva.

\begin{teorema}[punto fisso]
Ogni termine combinatorio (rispetto all'applicazione combinatoria) $X$ ha un
punto fisso poich\'e esiste un termine combinatorio $Y$, detto \emph{
combinatore di punto fisso}, i cui valori sono punti fissi per $X$ qualunque
$X$ si scelga.
\[
X(YX) = YX.
\]
\end{teorema}

%%--Dimostrazione
\begin{dimostrazione}[punto fisso]
Definiamo $Y$ come segue:
\[ Y = [x]\bigl([y]x(yy)\bigr)\bigl([y]x(yy)\bigr). \]
Allora dal teorema di riduzione si ha:

\[
\begin{array}{lcl}
YX & \vartriangleright & [y]\underbrace{X(yy)}_{M}\underbrace{\bigl([y]X(yy)
\bigr)}_{N} \\
& \vartriangleright & X\Bigl(\bigl([y]X(yy)\bigr)\bigl([y]X(yy)\bigr)\Bigr)\\
& \vartriangleleft & X\Bigl(\underbrace{[x]\bigl([y]x(yy)\bigr)\bigl([y]x
(yy)\bigr)}_{Y}X\Bigr) \\
& \equiv & X(YX). \\
\end{array}
\]
Abbiamo supposto per semplicit\`a che $x, y \notin X$, si vedano i lemmi 
\ref{Lc02n1} e \ref{Lc02n2}.
\end{dimostrazione}

\begin{osservazione}
Le riduzioni dirette (o inverse) considerate comportano delle contrazioni.
Quindi $YX$ non pu\`o essere in forma normale, ma anche $X(YX)$ non pu\`o
esserlo:
\[ YX = X(YX) = X\bigl(X(YX)\bigr) = \ldots \]
Quindi abbiamo dei termini combinatori che non possono avere forma normale, 
per esempio ci\`o accade prendendo per $X$ una variabile.
Talvolta tale $Y$ \~e servito per implementare la ricorsione nei linguaggi di
programmazione.
\end{osservazione}

\begin{osservazione}
\label{est}
L'uguaglianza debole non conserva l'astrazione e non vale la propriet\`a di
estensionalit\`a (est).
\begin{enumerate}
\item[a)]L'uguaglianza debole non conserva l'astrazione, cio\`e:
  \[ 
  X = Y \Longrightarrow [x]X = [x]Y, \quad \text{(in generale).} 
  \]
  Controesempio:
  \[ 
  X \equiv \sss xyz \text{, } Y \equiv xz(yz), \ \text{per (S) e 
  $X = Y$,  ma:} 
  \]
  \[ 
  [x]X \equiv \sss\bigl([x]\sss xy\bigr)\bigl([x]z\bigr) \equiv 
  \sss(\sss\sss(\kk y))(\kk z); 
  \]
  \[ 
  [x]Y \equiv \sss\bigl([x]xz\bigr)\bigl([x]yz\bigr) \equiv 
  \sss\bigl(\sss\ii(\kk z)\bigr)\bigl(\sss(\kk y)(\kk z)\bigr). 
  \]
  Sia $[x]X$ che $[x]Y$ sono in forma normale, ma non sono 
  equivalenti e per il corollario del teorema di Church-Rosser devono essere
  non debolmente uguali ($\neq$).
\item[b)]Non vale la propriet\`a di estensionalit\`a:
  \[ XZ = YZ \text{ per ogni termine combinatorio } Z \text{ allora } X = Y;
  \qquad \text{(est)} \]
  ci\`o si pu\`o vedere dal controesempio precedente, poich\`e per 
  $X' \equiv [x]X$ e $Y' \equiv [x]Y$:

  \[
   \begin{array}{lcl}
    X'Z & = & SZyz \\
        & = & Zz(yz) \\
        & \equiv & [Z/x]xz(yz) \\
        & = & Y'Z.
  \end{array}
  \]

\end{enumerate}
La propriet\`a (est) \`e detta di \emph{estensionalit\`a}. Infatti essa
corrisponde a quella di estensionalit\`a delle funzioni insiemistiche (che
deriva dall'assioma di estensionalit\`a degli insiemi) che asserisce:
\[ f(a) = g(a) \ \forall \ a \Longrightarrow f = g \]
Si noti che per i termini combinatori l'uguaglianza debole deriva dalla 
riduzione debole che rappresenta un processo di calcolo idealizzato su un 
programma completo (dei suoi dati). 
\end{osservazione}

%%--Esercizi Significativi
\section{Esercizi significativi}
In questa sezione del capitolo mostreremo due esercizi importanti svolti in 
classe con la relativa interpretazione funzionale.

\subsection{Diagonalizzatore}
Partendo dal seguente esercizio otterremo un'operatore che dato un termine 
combinatorio $X$, visto come una matrice, ne trova la diagonale.

\begin{ese}
Trovare un termine combinatorio $\ww$ tale che:
\[
\ww XY \vartriangleright XYY.
\]
\end{ese}
Quindi consideriamo: $\ww xy \vartriangleright xyy$.

\[
\begin{array}{lcl}
  \ww & \equiv & [x,y]xyy \\
   & \equiv & [x]\bigl([y]xyy\bigr) \\
   & \equiv & [x]\Bigl(\sss \bigl([y]xy\bigr)\bigl([y]y\bigr)\Bigr) \\
   & \equiv & [x](\sss x\ii) \\
   & \equiv & \sss\bigl([x]\sss x\bigr)\bigl([x]\ii \bigr) \\
   & \equiv & \sss\sss (\kk \ii). \\
\end{array}
\]
Applichiamo ora tale $\ww \equiv \sss\sss (\kk \ii)$ alle variabili richieste 
$xy$:
\[
\begin{array}{lcl}
\ww xy & \equiv &  \sss\sss (\kk \ii)xy \\
    & \vartriangleright & \sss x(\kk \ii x)y \\
    & \vartriangleright & \sss x\ii y\\
    & \vartriangleright &  xy(\ii y)\\
    & \vartriangleright &  xyy.\\
\end{array}
\]
Questo dimostra che il termine combinatorio $\ww$ sopra definito \`e corretto.

\subsection{Interpretazione funzionale}
Partendo dal secondo membro ($xyy$) si nota che $x$ \`e una funzione i cui 
argomenti sono pensabili come $y \in A$ per qualche $A$. Ma i suoi valori 
($xy$) sono a loro volta funzioni i cui argomenti sono contenuti in un insieme 
$B$. 

Allora $xy$ \`e da pensare come $x_y\colon A \to B$.
Pertanto:

$$x\colon A \to B^A.$$

Per esempio se consideriamo una matrice quadrata $X$ (nel senso di array) vista
 per colonne, potremmo pensare a $B$ come ad un insieme che contenga gli 
elementi delle colonne.

Se noi uguagliamo $xyy$ a $Wxy$ possiamo ora considerare il primo membro $Wxy$
come il valore in $y$ (colonna o riga $y$-esima) di una funzione $Wx$ che \`e
definita da $X$. Tale $Wx$ (o $W_x$) restituisce allora $x_y(y)$, cio\`e il
$y$-esimo elemento della $y$-esima colonna (o riga) di $X$.

Perci\`o $W_x$ \`e la funzione che ci da la diagonale della matrice $X$. 
Potremo quindi dire che $W$ \`e un diagonalizzatore.

\subsection{Estrattore parametrico}
\label{Phi}
Nel precedente esercizio $\ww$ estraeva da una matrice il ``percorso'' della
sua diagonale, cio\`e quello definito dai parametri $U$, $V \equiv \ii$ per
cui $Uy = y$ e $Vy = y$ individuano un elemento di diagonale.
Partendo dal seguente esercizio otterremo un'operatore che dato un termine 
combinatorio $X$ visto come una matrice, estrae il percorso definito 
parametricamente da $U$ e $V$ arbitrariamente (con indice $Y$).

\begin{ese}
Trovare un termine combinatorio $\Phi$ tale che:
\[
\Phi XUVY \vartriangleright X(UY)(VY).
\]
\end{ese}
Quindi consideriamo: $\Phi xuvy \vartriangleright x(uy)(vy)$.

\[
\begin{array}{lcl}
\Phi & \equiv & [x,u,v,y]x(uy)(vy) \\
     & \equiv & [x,u,v]\bigl([y]x(uy)(vy)\bigr) \\
     & \equiv & [x,u,v]\Bigl(\sss \bigl([y]x(uy)\bigr)\bigl([y]vy\bigr)\Bigr)\\
     & \equiv & [x,u,v]\Bigl(\sss \bigl(\sss ([y]x)([y]uy)\bigr)v\Bigr) \\
     & \equiv & [x,u,v]\Bigl(\sss \bigl(\sss (\kk x)u\bigr)v\bigr) \\
     & \equiv & [x,u]\Bigl([v]\sss \bigl(\sss (\kk x)u\bigr)v\bigr) \\
     & \equiv & [x,u]\sss \bigl(\sss (\kk x)u\bigr) \\
     & \equiv & [x]\Bigl([u]\sss \bigl(\sss (\kk x)u\bigr)\Bigr) \\
     & \equiv & [x]\sss \bigl(\sss (\kk x)\bigr) \\ 
     & \equiv & \sss([x]\sss)\Bigl([x] \bigl(\sss (\kk x)\bigr) \Bigr)\\
     & \equiv & \sss(\kk\sss)\bigl(\sss([x]\sss)([x]\kk x)\bigr) \\
     & \equiv & \sss(\kk\sss)\bigl(\sss(\kk\sss)\kk\bigr).\\
\end{array}
\]

\subsection{Interpretazione funzionale}
$\Phi$ differisce da $\ww$ solo perch\`e in luogo di $U$ e $V$ (arbitrarie) 
usiamo le $\ii$ (identit\`a).
Ma allora specificando diversamente $U$ e $V$ si da un percorso differente 
dalla diagonale. Il percorso \`e ``specificato parametricamente'' da $U$ e $V$
mentre $Y$ \`e sempre l'indice d ``percorrenza''.
$\Phi$ \`e quindi un ``estrattore parametrico''.

\subsection{Conclusioni}
Con le costanti riducibili (le primitive del linguaggio di programmazione
idealizzato) $\sss$ e $\kk$ siamo stati in grado di costruire altri operatori
(diversi dal generatori di costanti e dal parallelizzatore):

\begin{enumerate}
\item[-]$\bb$: operatore di composizione invertito;
\item[-]$\cc$: invertitore, che come vedremo sotto generalizza la 
trasposizione;
\item[-]$\ww$: diagonalizzatore;
\item[-]$\Phi$: estrattore parametrico.  
\end{enumerate}


Da $\cc XYZ \vartriangleright XZY$ con il ragionamento visto per $\ww$ si ha 
che, pensando ad $X$ come matrice per colonne, $\cc X$ \`e la stessa mastrice 
ma per righe. Cio\`e $\cc$ \`e la generalizzazione della trasposizione 
matriciale.
Si noti anche che il nome \`e necessario e che la matrice $X$ sia pensata
quadrata.

Con i soliti riferimenti insiemistici si pu\`o vedere la $\cc$ come:
\[
\cc \colon (A^B)^C \to (A^C)^B.
\]

Ci si pu\`o allora chiedere che cosa d'altro le nostre primitive consentano di
calcolare.

%%
%%--Capitolo 3
%%
\chapter{Funzioni ricorsive}
Secondo la tesi di Church (e l'equivalente tesi di Turing), tutto ci\`o
che \`e calcolabile pu\`o essere definito come tutto ci\`o che ci viene dato 
da qualche funzione parziale ricorsiva.

\section{Funzioni parziali ricorsive}
Una funzione fra numeri naturali si dice parziale ricorsiva quando pu\`o essere
 ottenuta da delle funzioni base mediante due operatori.

\begin{description}\item[Funzioni di base]:

\begin{enumerate}
\item[-]costante $0$;
\item[-]somma;
\item[-]proiezioni $U_{i}^{n}(x_0, \ldots  , x_{n-1}) = x_i$.
\end{enumerate} 
\end{description}
\begin{description}
\item[Operatori]:

\begin{enumerate}
\item[-]Composizione multipla;
  \item[]data una funzione $f$ ad $n$ argomenti ed $n$
  funzioni $g_i$ ad $m$ argomenti, otteniamo una funzione $h$  ad $n$ argomenti
  tale che: 
  \[
  h(x_0, \ldots, x_{n-1}) = f\bigr(g_0(x_0, \ldots , x_{m-1}), 
  \ldots, g_{n-1}(x_0, \ldots  , x_{m-1})\bigr).
  \]
\item[-]Minimalizzazione;
  \item[]data una funzione $g$ ad $n+1$ argomenti otteniamo una funzione 
    $h$  ad $n$ argomenti tale che: $h(x_0, \ldots, x_{n-1}) = \text{min } y$ 
    per il quale $g(y, x_0,\ldots , x_{n-1}) = 0$ e si scrive:
    \[
    h(x_0, \ldots, 
    x_{n-1}) = \mu_y \bigl[g(
    y,  x_0, \ldots , x_{n-1}) = 0\bigr].
    \]

\end{enumerate}
In realt\`a l'operatore $\mu$ \`e un operatore d'inversione.
\end{description}
\begin{esempio}Vediamo una possibile applicazione dell'operatore $\mu$:
\[
\mu_y(x_oy - x_1 = 0) = \frac{x_1}{x_0}
\] 
cio\`e $\mu$ ha invertito la moltiplicazione.
\end{esempio}
Per certe operazioni vi possono essere pi\`u risultati d'inversione.
\begin{esempio}L'equazione:
\[
(y-2)^2 - x_0 = 0
\]
per $x_0 = 4$ restituisce sia $0$ che $4$, al fine di ottenere un solo 
risultato si sceglie il minimo.
\end{esempio}

%%--Calcolabilita' delle funzioni parziali ricorsive
\section{Alcune considerazioni}
Che la classe delle funzioni parziali ricorsive possa essere una candidata
per la definizione della classe delle funzioni calcolabili segue dalle seguenti
considerazioni:ù
\begin{enumerate}
  \item[A]le funzioni di base sono notoriamente calcolabili,
  \item[B]le costruzioni date dagli operatori di composizione multipla e 
 minimizzazione ($\mu$) conservano la calcolabilit\`a\footnote{La composizione 
    multipla fra funzioni calcolabili ci da sempre una funzione calcolabile 
    mediante il ``passaggio'' dei valori delle $g_i$ agli argomenti della $f$. 
    Anche dalla $\mu$ otteniamo funzioni calcolabili, basta eseguire le 
    seguenti procedure:
    \begin{enumerate}
    \item[a)]$y = 0$;
    \item[b)]calcola $g(y, x_0,\ldots )$;
    \item[c)]verifica se $g(y, x_0,\ldots ) = 0$, se si esci;
    \item[d)]altrimenti incrementa $y$ e vai a $b)$.
    \end{enumerate}
    Si pu\`o verificare che esiste una procedura equivalente, sia pure pi\`u
    complicata, anche per il caso che qualche valore di $g$ non sia definito.}.
\end{enumerate}
Tale classe \`e detto di funzioni parziali perch\`e la $\mu$ anche nel caso di
$g$ ovunque definita pu\`o dare una fuzione che non \`e ovunque definita 
(esempio $\mu_y(2y - x = 0)$ \`e una funzione di $x$ definita solo per gli $x$ 
pari).

Tali funzioni si dicono semplicemente \emph{ricorsive} quando sono ``totali'', 
cio\`e definite per qualunque scelta degli argomenti.
In generale le funzioni parziali ricorsive sono dette ``ricorsive'' perch\`e si
pu\`o dimostrare che con gli operatori funzionali e funzioni di base si possono
generare funzioni definite tramite delle ricorsioni.

%%--Ricorsione
\section{Ricorsione primitiva}
Vi sono parecchi casi di ricorsione. Uno dei pi\`u semplici ed apparentemente
potente \`e quello della \emph{ricorsione primitiva}.

\begin{definizione}
Una funzione $h$ di $n + 1$ argomenti si dice ottenuta per \emph{ricorsione 
primitiva} da una funzione $g$ ad $n$ argomenti e da una funzione $f$ a $n + 2$
 argomenti se:
\[
\left\{ 
\begin{array}{l}
h(0, x_0, \ldots , x_{n-1}) = g(x_0, \ldots , x_{n-1});
 \\
h(k+1, x_0, \ldots , x_{n-1}) = f\bigl(k, h(k, x_0, \ldots , x_{n-1}),
x_0, \ldots , x_{n-1}\bigr). \\
\end{array} \right.
\]
\end{definizione}
Si noti che se la $f$ e la $g$ sono totali, tali equazioni definiscono una $h$
che \`e totale a sua volta e conserva ovviamente la calcolabilit\`a.

\begin{osservazione}
Non tutte le costruzioni ricorsive sono primitive
\end{osservazione}
\begin{esempio}
Le due seguenti condizioni (ove $\frac{k}{2}$ indica il quoziente di $k$ diviso
$2$) ci definiscono una $l$ ad un argomento che \`e totale.
\[l(0) = 0\]
\[l(k+1) = l\left(\frac{k}{2}\right) + 1\]
Infatti da esse si ha:
\[
\begin{tabular}{c|c}
n & l(n) \\
\hline
0 & 0 \\
1 & 1 \\
2 & 1 \\
3 & 2 \\
4 & 2 \\
\vdots & \vdots \\
7 & 3 \\
8 & 3 \\
\vdots & \vdots \\
15 & 4 \\
\end{tabular}
\]
Cio\`e ``quasi un logaritmo''.

Tuttavia la costruzione della $l$ \emph{non} \`e ricorsiva primitiva. Infatti 
il passo di ricorsione \`e diverso (usa un valore ``$\frac{k}{2}$'' differente 
dal  predecessore per calcolare $l(k+1)$).
\end{esempio}

%%--Funzioni ricorsive primitive
\section{Funzioni ricorsive primitive}
Usando opportune funzioni di base (totali): la precedente composizione multipla
 e tale ricorsione primitiva, si pu\`o definire un'ampia sottoclasse delle 
funzioni parziali ricorsive dette \emph{funzioni primitive ricorsive}.
Per quanto osservato tali funzioni sono totali.
Si vede facilmente che fra di esse vi sono:
\begin{enumerate}
\item[] $h(x, y) = x + y$;
\item[] $h(x, y) = xy$;
\item[] $h(x, y) = x^y$;
\item[] $h(x, y) = \left\{ \begin{array}{ll}
                           0 & \text{se } x < y; \\
			   x - y & \text{altrimenti}; \\
			 \end{array} \right.$
\item[] etc.
\end{enumerate}
Tuttavia esistono funzioni chiaramente calcolabili, anche totali, che non sono
primitive ricorsive, cio\`e tali che non pu\'o esistere alcuna costruzione di 
esse nella classe delle funzioni primitive ricorsive. Una di esse \`e la 
funzione di Ackerman.

Tale funzione in due variabili: $ack(x, y)$ al crescere di $y$ restituisce
delle funzioni in $x$ come le seguenti:

$\begin{array}{l}
\vdots \\
x \\
x^2 \\
x^{x} \ldots \\
\vdots \\
\text{(superesponenziale)} \\
\end{array}$

Essa \`e totale ed esiste una definizione ricorsiva (non primitiva) abbastanza 
semplice per tale $ack$.

(Si noti che per la funzione $l$ vista in precedenza l'esistenza della
costruzione ricorsiva non primitiva vista \emph{non} \`e sufficiente per
affermare che tale funzione non sia primitiva.)

%%--Lemma di Parzialita'
\section{Lemma di parzialit\`a}
Si potrebbe pensare che possa esistere una classe cos\`i ampia di funzioni 
totali (che contenga ad esempio $ack$) e costruibile senza costruzioni (come 
la $\mu$), che non conservano la totalit\`a, da contenere ogni funzione 
``ragionevolemente calcolabile''. Tuttavia esiste un lemma che, definendo 
opportunamente delle condizioni di ragionevolezza di calcolabilit\`a, mostra
la costruzione della classe necessita di operatori (come $\mu$) che non 
conservano la totalit\`a.

%%--Tesi di Church
\section{Tesi di Church}
La tesi di Church afferma che la definizione delle funzioni fra numeri 
naturali, che formalizza nel modo pi\`u ampio possibile la nozione di 
calcolabilit\`a, \`e quella della classe delle funzioni parziali ricorsive. 
Pertanto tale classe non \`e solo una candidata ma, a meno di equivalenze, la
soluzione del problema definitorio.

La giustificazione di tale tesi consiste nelle seguenti asserzioni:

\begin{enumerate}
\item[-]\emph{Sufficienza puntuale}. Non \`e mai stata trovata una funzione 
  che si fosse in grado di calcolare per la quale si potesse dimostrare che 
  non era nella nostra classe (per esempio $ack$, anche se definita con una
  ricorsione senza minimalizzazione, \`e dimostrato che sia definibile con
  le costruzioni della nostra classe).
\item[-]\emph{Sufficienza di classe}. Tutte le classi notevoli di funzioni
  sinora scoperte sono state dimostrate essere contenute nella nostra classe.
  In particolare si \`e dimostrata l'uguaglianza con classi di funzioni la cui
  generalit\`a di calcolabilit\`a era intuibile da considerazioni differenti 
  (ad esempio la classe delle funzioni calcolabili secondo Turing).
\end{enumerate}

%%--Forma normale delle funzioni parziali ricorsive
\section{Forma normale delle funzioni parziali ricorsive}
Secondo la tesi di Church funzione calcolabile, cio\`e parziale ricorsiva,
pu\`o essere definita a 
partire dalle funzioni di base ($0$, $+$, $U_i^n$) relative mediante un uso 
iterato degli operatori di composizione (multipla) e $\mu$.
Il seguente teorema ci dice che per qualunque funzione parziale ricorsiva 
basta una sola $\mu$, a patto di usare delle funzioni primitive ricorsive.

Quest'ultime sono definite a partire da funzioni di base come le precedenti, 
ma con la somma sostituita dal \emph{successore} da degli operatori che
conservano la totalit\`a delle funzioni (non c'\`e la $\mu$).
Le funzioni parziali ricorsive pertanto sono realizzabili da programmi che 
danno sempre il risultato della funzione. L'operatore cirico ($\mu$), che 
invece produce funzioni parziali, potrebbe essere usato una volta sola.

\begin{teorema}
Per ogni funzione parziale ricorsiva $\varphi$ ad $n$ argomenti esistono due
funzioni primitive ricorsive $f$ ad un argomento e $g$ a $n + 1$ argomenti
tale che per tutte le $n$-uple $(x_0, \ldots , x_{n-1})$ nel suo dominio si ha:
\[
\varphi(x_0, \ldots , x_{n-1}) = f\Bigl(\mu_y\bigl(g(y, x_0, \ldots , x_{n-1}
) = 0 \bigr) \Bigr).
\]
\end{teorema}

%%--Dimostrazione
\begin{dimostrazione}
No.
\end{dimostrazione}
Il teorema pu\`o essere dimostrato mediante la macchina di Turing. Si 
formalizza tramite delle funzioni primtive ricorsive il funzionamento di una 
macchina che inizia a calcolare con dati iniziali $(x_0, \ldots , x_{n-1})$.
Le transizioni di configurazioni della macchina di Turing corrispondono 
sostanzialmente al parametro $y$ e la funzione primitiva ricorsiva $g$ 
rappresenta, quando ha valore nullo, l'arresto della macchina.
Dalla configurazione corrispondente a tale $y$ ``nullificante'' la $f$ 
estrae il risultato.

%%--
%%--Capitolo Rappresentazione combinatoria delle funzioni parziali ricorsive
%%--
\chapter[Rapp.ne funz. parz. ric.]{Rappresentazione combinatoria delle 
funzioni parziali ricorsive}
Il calcolo dei combinatori (e il $\lambda$-calcolo) sono nati anche come 
possibile fondazione della matematica, alternativa alla teoria degli insimi.
Siccome la seconda \`e stata sviluppata precedentemente, non \`e stata 
sviluppata (completamente) la prima. Tuttavia esiste la dimostrazione che tale 
calcolo avrebbe potuto diventare una fondazione alternativa. Tale dimostrazione
 consiste nel mostrare come all'interno del calcolo dei combinatori si possano 
rapresentare le funzioni parziali ricorsive. Per vedere ci\`o occorre dapprima 
rappresentare i numeri naturali.

Vi sono parecchi modi per rappresentare i numeri naturali con termini combinatori, quello che vedremo consiste in una rappresentazione ``naturale'' mediante 
soli combinatori, detti \emph{iteratori di Church}. Essi catturano il significato \emph{operativo} di un numero naturale appunto come ``iteratore''. Ci\`o 
sar\`a dimostrato dal primo lemma seguente.

\begin{notazione}
Conveniamo di indicare con $X^{n}Y \text{ ove } \ n \in \mathbb{N}$ i termini 
seguenti:

\[
X^0Y \equiv Y;
\]

\[
X^{n+1}Y \equiv \underbrace{X(X(\ldots (X}_{n+1}Y)\ldots )).
\]

Definiamo per ogni numero naturale $n$ il suo \emph{rappresentante combinatorio
} come il combinatore $\overline{n}$ tale che:

\[
\overline{0} \equiv KI;
\]

\[
\overline{n+1} \equiv (SB)^{n+1}(KI) \text{ ovvero } \overline{n+1} = SB(
\overline{n}).
\]

\end{notazione}

%%--Lemma
\begin{lemma}
Per ogni termine combinatorio $F$ e $Y$ si ha che:

\[
\overline{n}FY = F^nY \text{ o meglio } \overline{n}FY \vartriangleright
 F^nY.
\]

\end{lemma}

%%--Dimostrazione
\begin{dimostrazione}
Per induzione su $n$.

\begin{enumerate}

\item[$\bullet$]Base  $n = 0$.
  
$\begin{array}{lll}
    \overline{0}FY & \equiv & KIFY \\
                   & \vartriangleright & IY \\
                   & \vartriangleright & Y \\
                   & \equiv & F^0Y.
  \end{array}
$

\item[$\bullet$]Ipotesi induttiva $\overline{n}FY \vartriangleright F^nY$.
\item[$\bullet$]Passo induttivo:

$\begin{array}{lll}
    \overline{n+1}FY & \equiv & SB\overline{n}FY \\
                     & \vartriangleright & BF(\overline{n}F)Y \\
                     & \vartriangleright & F(\overline{n}FY) \\
                     & \vartriangleright & F(F^nY) \\
                     & \equiv & F^{n+1}Y.
  
  \end{array}
$

\end{enumerate}
 
\end{dimostrazione}

\begin{definizione}
Data una funzione tra numeri naturali $\varphi$ ad $n$ argomenti diremo che 
un termine combinatorio $X$ la \emph{rappresenta combinatoriamente (in senso 
debole)} qualora:

\[
X\overline{x}_0\overline{x}_1 \ldots \overline{x}_{n-1} = \overline{\varphi(x_0, x_1,  \ldots, x_{n-1})} \quad \forall \ 
x_0, x_1,  \ldots, x_{n-1} \in \Dom{\varphi}.
\]
\end{definizione}

Se inoltre succede che per le variabili $x_0, x_1,  \ldots, x_{n-1}$ fuori
dal dominio di $\varphi$ il primo membro non ha forma normale diremo che $X$ la
rappresenta combinatoriamente \emph{in senso forte}. Le rappresentazioni in 
senso forte non sono per\`o necessarie per noi.

\begin{osservazione}
Si noti che mentre l'applicazione insiemistica per la $\varphi$ riguardi 
argomenti che sono $n$-uple, quella combinatoria per la $X$ riguarda un singolo
 argomento (cio\`e $x_0$). Tuttavia tale applicazione viene iterata (su $x_1
\ldots x_{n-1}$).

Questo non dipende dal fatto che nel calcolo dei combinatori non si possano 
rappresentare delle $n$-uple; essa \`e solo una scelta di semplicit\`a ``suffi
ciente'' per i nostri scopi (cos\`i come quella della riduzione debole).
Inoltre nei lemmi per la rappresentazione delle funzioni parziali ricorsive si 
devono comunque introdurre delle rappresentazioni (di coppie) e queste 
farebbero confusione con quelle per gli argomenti.
\end{osservazione}

%%--Teorema di rappresentazione
\begin{teorema}[di rappresentazione]
Ogni funzione parziale ricorsiva pu\`o essere rappresentatan in senso debole 
da qualche termine combinatorio.
\end{teorema}
Di questo enunciato non daremo la 
dimostrazione completa, ma solo dei cenni ad essa ed ai lemmi che usa.

%%--Dimostrazione
\begin{dimostrazione}[cenno]
Per il teorema della forma normale delle funzioni parziali ricorsive $\varphi(
x_0, \ldots , x_{n-1}) = f(\mu_y(g(y, x_0, \ldots , x_{n-1}) = 0))$, data una 
tale $\varphi$ esistono due funzioni primitive ricorsive $f$ e $g$ che la 
definiscono mediante $\mu$. Pertanto basta dimostrare che ogni funzione 
primitiva ricorsiva \`e rappresentabile combinatoriamente e che date una $g$ 
ed una $f$ cos\`i rappresentabili, \`e pure rappresentabile la minimizzazione 
della $g$ e la composizione di una funzione rappresentabile con la $f$.

Noi trascureremo la rappresentabilit\`a della $\mu$ e accenneremo a quella 
delle funzioni ricorsive primitive (che comporta anche un caso di 
composizione).
\end{dimostrazione}

%%--Lemma
\begin{lemma}
Ogni funzione primitiva ricorsiva \`e rappresentabile combinatoriamente in 
senso debole.
\end{lemma}

%%--Dimostrazione
\begin{dimostrazione}[cenno]
Occorre dimostrare che le funzioni di base per la ricorsione primitiva sono
 rappresentabili. Queste sono:
\begin{enumerate}
\item[-]$0$, inteso come una funzione (costante) di zero argomenti (o anche 
  come  una funzione costante di un argomento);
\item[-]$s$, successore;
\item[-]$U_i^n$, proiezioni.
\end{enumerate}
Esse sono immediatamente rappresentabili poich\`e:
\begin{enumerate}
\item[-]$\overline{0}$ cio\`e $KI$, \`e gi\`a stato rappresentato (c'\`e una 
  piccola modifica per lo $0$ con un argomento);
\item[-]$\overline{s}$ altro non \`e che $S\!B$ poich\`e $S\!B\overline{n}\ = 
\ \overline{n+1}\ =\ \overline{s(n)}$;
\item[-]$\overline{U_i^n}$ si ottengono dagli $X \equiv [x_0, \ldots , x_{n-1}]
  x_i$, poich\`e dal teorema di riduzione si ha che $X\overline{x}_0\,
  \overline{x}_1 \cdots \overline{x}_{n-1} = \overline{x}_i$, ove gli $x_i$
  siano numeri naturali.
\end{enumerate}
Una volta ottenuta la rappresentabilit\`a  di tali funzioni di base occorre 
dimostrare che la rappresentabilit\`a si conserva per composizione (multipla) 
e ricorsione primitiva.

Per la composizione si osservi che date $f$ ad $m$ argomenti e $g_0, \ldots , 
g_{m-1}$ tutte ad $n$ argomenti, il rappresentanto della $h$ tale che:
\[
h(x_0, \ldots , x_{n-1}) = f(g_0(x_0, \ldots , x_{n-1}), \ldots ,  g_{m-1}(x_0,
 \ldots , x_{n-1});
\]
pu\`o ovunque essere definito dai rappresentanti $\overline{f}$, 
$\overline{g}_0$, $\ldots$ , $\overline{g}_{n-1}$ mediante:

\[
\overline{h} = [x_0, \ldots , x_{n-1}] \overline{f}(\overline{g}_0x_0
x_{1}\cdots x_{n-1})\cdots (\overline{g}_{m-1}x_0x_{1}\cdots x_{n-1}).
\]
Che tale termine combinatorio $\overline{h}$ sia effettivamente il 
rappresentante richiesto segue ancora dal teorema di riduzione multipla:
\[
\overline{h}\,\overline{x}_1\cdots\overline{x}_{n-1} \ = \ \overline{f}(
\overline{g}_0\,\overline{x}_0 \cdots \overline{x}_{n-1})\cdots(
\overline{g}_{m-1}\,\overline{x}_0\cdots\overline{x}_{n-1}).
\]
Per la ricorsione primitiva si procede in due passi:
\begin{enumerate}
\item[\rm{I)}]Si dimostra che la rappresentabilit\`a si conserva nel caso che 
  la funzione $g$, nella clausola di base della ricorsione, sia una funzione a 
  zero argomenti.
\item[\rm{II)}]Si estende tale dimostrazione al caso di pi\`u argomenti.
\end{enumerate}
Noi omettiamo il passo $\rm{II)}$  ed accenniamo al primo.

Nel caso di zero argomenti definiamo una $\varphi$ ad un argomento mediante 
una ricorsione primitiva semplificata del tipo:
\[
\left \{ \begin{array}{lc}
  \varphi(0) = c;  & \text{(costante)} \\
  \varphi(n + 1) = \chi\bigl(n , \varphi(n)\bigr). & \\
\end{array}
\right .
\]

L'idea su cui si basa la dimostrazione \`e la seguente. Consideriamo la 
$\varphi$ (che \`e funzione insiemistica) come l'insieme delle coppie: 
\[
\Bigl\{
\bigl \langle 0, c \bigr\rangle, \bigl\langle 1, \varphi(n) \bigr\rangle , 
\ldots , \bigl\langle n, 
\varphi(n) \bigr\rangle , \ldots 
\Bigr\}.
\] 
In tali coppie si ha che $\varphi(n + 1)$ \`e uguale a $\chi\bigl(n, \varphi(n)
\bigr)$; per esempio si ha la coppia $\langle 1, \chi(0, c) \rangle$. 

Pertanto possiamo enumerare tale insieme 
mediante una funzione $f$ dipendente da $\chi$, $f = q_{\chi}$, tale che:

\[
f\bigl(n, \varphi(n)\bigr) 
= \Bigl\langle n+1,\ \varphi(n+1) \Bigr\rangle 
= \Bigl\langle n+1,\ \chi \bigl(n, \varphi(n)\bigr) \Bigr\rangle.
\]

Se saremo in grado di rappresentare tale $f$ mediante un termine combinatorio 
$F$ potremo ottenere la rappresentazione della generica coppia $\langle n+1, \ 
\varphi(n+1) \rangle$ mediante iterazione fatto sulla $F$. In altre parole 
potremo potremo ridurre il nostro caso semplificato (zero argomenti) di 
ricorsione primitiva a una semplice iterazione usando $F^n$, cio\`e 
utilizzando gli iteratori come nel lemma d'iterazione.

Per precisare tale idea occorre tuttavia essere in grado di rappresentare 
coppie di numeri naturali e di manipolarle in modo da poter estrarre dal 
rappresentante di $\langle n, \ \varphi(n) \rangle$ il secondo componente 
$\varphi(n)$ una volta individuato il primo.
\end{dimostrazione}

%%--Section Rappresentazione delle coppie insiemistiche
\section{Rappresentazione delle coppie insiemistiche}
Una prima idea per rappresentare una coppia di termini combinatori con un'altro
 termine \`e quella di considerare l'applicazione di due termini componenti.
Tuttavia ci\`o non funziona nonostante che l'applicazione goda (per 
definizione) di alcune propriet\`a delle coppie insiemistiche. Esistono 
funzioni di ``decomposizione'' $p$ ed $s$ tali che:

\[
p\bigl(\langle A, B\rangle\bigr) =  A\quad \text{e}\quad s\bigl(\langle A, 
B\rangle\bigr) = B. 
\]
Ci\`o non \`e possibile con l'applicazione poich\`e nella rappresentazione 
l'uguale insiemitico deve diventare l'uguale combinatorio (definito da un 
processo di calcolo). Infatti si ha ad esempio che non pu\`o esistere un $P$
che si comporti come il $p$ insiemistico, cio\`e non esiste un termine 
combinatorio $P$ tale che:
\[
P(XY) = X, \quad \text{per ogni termine combinatorio } X \text{ e }Y.
\]

Altra idea. Definiamo $D' \equiv [x,y,z]zxy$, dati $X$ e $Y$ abbiamo cos\`i il
termine combinatorio $D'XYZ = ZXY$ per ogni $Z$. Se prendiamo $Z \equiv \kk$
\[
D'XYZ = \kk XY = X;
\]
pertanto possiamo definire come rappresentante della coppia $X$ e $Y$ il 
termine combinatorio $D'XY$ e come $P$:
\[
P \equiv [x]xK, \quad \text{(di modo che } P(D'XY) = D'XY\kk = X \text{ ).}
\]

Oltre a tale $D'$ esistono parecchi combinatori che consentono di rappresentare
 le coppie insiemistiche. Il combinatore di ``coppia'' che chiameremo 
\emph{diade}, che serve nel nostro caso, usa come termini $P$ ed $S$ (per 
rappresentare $s$) dei termini
derivati da rappresentanti numerici. Infatti definiamo:
\[
D \equiv [x, y, z]z(\kk y)x.
\]
Si ha allora che $DXY\overline{0}$ per il teorema di riduzione:
\[
DXY\overline{0} = \overline{0}(\kk Y)X = \kk \ii(\kk Y)X = \ii X = X;
\]
\[
DXY\overline{n+1}\ = \ \cdots \ = \ Y.
\]
Cio\`e da $DXY$ possiamo estrarre $X$ o $Y$ mediante $P$ ed $S$ come i 
precedenti, ma definiti a partire dai rappresentanti numerici.

Usando tale $D$ possiamo dimostrare il seguente lemma.

%%--Lemma
\begin{lemma}
Esiste un combinatore $\mathbf{R}$, che ha forma normale, tale che:
\begin{enumerate}
  \item[1)] $\mathbf{R}XY\overline{0} \ = \ X$;
  \item[2)] $\mathbf{R}XY\overline{n+1} = Y\overline{n}(\mathbf{R}XY
    \overline{n})$.
\end{enumerate}
Tale operatore di ricorsione pu\`o essere usato allora per rappresentare la 
ricorsione primitiva semplificata ($n = 0$) poich\`e $X$ pu\`o essere preso 
come il rappresentante di $c$ ($\overline{c}$) ed $Y$ come il rappresentante di
$\chi$ ($\overline{\chi}$), per cui, dalla 1) e 2) $\mathbf{R}XY$ 
rappresenta $\varphi$ in accordo a $\varphi(0) = c$ e $\varphi(n+1) = \chi\bigl
(n,\varphi(n)\bigr)$.
\end{lemma}

%%--Dimostrazione
\begin{dimostrazione}[cenno]
Si segue l'idea precedente di numerare i rappresentanti delle coppie $\langle n
, \varphi(n) \rangle$ mediante il rappresentante di $f = q_{\chi}$.
Pertanto si definisce un termine $Q$ che rappresenti $q$, cio\`e tale che $QY$
rapresenti la $f$, quindi la $Y$ rappresenta la $\chi$. Tale $Q$ \`e costruita
tramite la diade $D$, infatti si ha 

\[Q \equiv [x, d]D\bigl(SB(d\overline{0})\bigr)
\bigl(x(d\overline{0})(d\overline{1})\bigr);\]
 ove la $x$ servir\`a per $\chi$ e la $d$ per
$\langle n, \varphi(n) \rangle$. Mediante tale $Q$ si pu\`o poi costruire il 
termine combinatorio $\mathbf{R}$ tramite iterazione ed utilizzando 
l'estrazione dei componenti della coppia rappresentata ottenendo la 1) e la 2).
In definitiva dal lemma di iterazione si ottiene un lemma di ricorsione 
primitiva semplificata.
\end{dimostrazione}

%%--Viceversa del Teorema di rappresentabilita' delle f.ni par. ric.
\begin{lemma}
Qualsiasi funzione rappresentabile combinatoriamente \`e parziale ricorsiva.
\end{lemma}

\begin{dimostrazione}
No.
\end{dimostrazione}
Non si fa perch\`e \`e ovvio a causa dell'equivalenza fra calcolabilit\`a
secondo Turing e funzioni parziali ricorsive.

%%
%%-- Capitolo - Indecidibilit\`a
%%
\chapter{Indecidibilit\`a}
La teoria e ``filosofia'' di A.~Turing ha chiarito che ``tutto il fattibile'' 
\`e tale se e solo se qualche macchina di Turing lo pu\`o fare.

\[
\includegraphics{img/infteofig.18}
\]

Se si danno in entrata dei dati e si aspettano in uscita i risultati, allora
non si pu\`o distinguere se all'interno della scatola vi sia un ``uomo'' o
un'opportuna macchina di Turing (M. di T.).
Turing si \`e inoltre occupato delle limitazioni di tale ``fattibilit\`a 
totale''\footnote{Precedentemente a Turing, K.~G\"odel aveva trovato dei 
risultati a proposito delle limitzioni della dimostrabilit\`a totale in 
matematica.}.
La prima di tali limitazioni \`e data dal teorema dell'arresto: non pu\`o
esistere alcuna M. di T. che, ricevendo in entrata la descrizione di un'altra
M. di T. qualsiasi e dei relativi dati d'ingresso, sia in grado di dare in 
uscita per il si e per il no risposta al quesito ``il calcolo della seconda
macchina da un risultato, cio\`e la seconda macchina si arresta?''.

Dal teorema dell'arresto derivano numerosi altri risultati di 
\emph{indecidibilit\`a}. Per esempio:
\begin{enumerate}
\item Non \`e possibile decidere sempre se una funzione (calcolabile) sia
  definita per certi argomenti; 
\item Non \`e possibile decidere sempre se una funzione (calcolabile) abbia per
  qualche argomento un certo valore;
\item Come corollario del $2)$ si ha che non \`e possibile calcolare una 
  funzione reale di variabili reali discontinua\footnote{La calcolabilit\`a
  riferita a funzioni tra numeri naturali si estende a quelle fra reali 
considerando ad esempio le rappresentazioni posizionali dei numeri reali che
sono assegnabili come funzioni fra numeri naturali};
\item Non \`e possibile decidere sempre se due funzioni (calcolabili) siano
  uguali oppure no.
\item Si recuperano i risultati di indecidibilit\`a della logica matematica;
\item[\vdots] 
\end{enumerate}
Buona parte di tali risultati di indecidibilit\`a, provenienti dal teorema 
dell'arresto sono ``condensati'' in un (pesante) teorema 
(di Rice-Shopiro-Myhill) che asserisce qualcosa di grossolanamente simile a
``qualsiasi cosa che sia sistematicamente prevedibile a proposito delle
funzioni effettivamente calcolabili \`e banale''.

I risultati di indecidibilit\`a servono nello sviluppo del software. Per 
esempio, se si vuole progettare un linguaggio di programmazione che abbia la
garanzia di dare programmi che diano sempre un risultato (evitando ad esempio
cicli non voluti dal programmatore) allora tale linguaggio \emph{non} pu\`o 
essere universale, cio\`e non deve consentire che sia programmabile qualsiasi
funzione calcolabile. Tali limitazioni dell'universalit\`a possono essere 
suggeriti dalla teoria dell'indecidibilit\`a.

%%--Section - Indecidibilit\`a per i termini combinatori
\section{Indecidibilit\`a per i termini combinatori}
Mentre l'indecidibilit\`a secondo Turing riguarda inizialmente solo l'agente di
calcolo (M. di T.), per il calcolo combinatorio si hanno subito dei risultati
d'indecidibilit\`a che riguardano i programmi idealizzati espressi dai termini
combinatori. In particolare avremo un risultato simile al $4.$ precedente, che
spesso sostituisce (tramite dei corollari immediati) il teorema di 
Rice-Shopiro-Myhill.

Si parte con la ``G\"odellizzazione'' dei termini combinatori. Ci\`o significa 
che ogni termine combinatorio $X$ viene identificato da un numero naturale 
$gd(X)$ secondo una enumerazione, data da una funzione $gd$ che abbia delle 
caratteristiche di fattibilit\`a. In particolare si richiede che esistano una
funzione ricorsiva (totale) $\mu$ a due argomenti ed una ricorsiva totale $\nu$
ad un argomento tali che:

\[
\mu\bigl(gd(X), gd(Y)\bigr) = gd(XY) \quad \forall \ \text{termine combinatorio
$X$ e $Y$};
\]

\[
\nu(n) = gd(\overline{n}) \quad \forall \ n \in \mathbb{N}.
\]
Noi trascureremo come definire una tale $gd$, poich\`e non dimostreremo il 
teorema.

\begin{definizione}
Diremo che due insiemi $\mathcal{A}$ e $\mathcal{B}$ di numeri naturali sono
\emph{ricorsivamente separabili} quando esiste una funzione ricorsiva (totale)
con valori $0$ ed $1$ tale che:
\begin{enumerate}
\item[-]se $n \in \mathcal{A} \ \Longrightarrow \ \varphi(n) = 0$;
\item[-]se $n \in \mathcal{B} \ \Longrightarrow \ \varphi(n) = 1$.
\end{enumerate}
\end{definizione}
Si noti che il semplice fatto che $\varphi$ sia funzione implica che 
$\mathcal{A} \cap \mathcal{B} = \emptyset$. Tuttavia non vale il viceversa,
poich\`e richiediamo che $\varphi$ sia anche ricorsiva.

In sostanza la $\varphi$ \`e una funzione calcolabile che sia la funzione
caratteristica di un insieme $\mathcal{B'}$ contenente $\mathcal{B}$ e 
disgiunto da $\mathcal{A}$.
\[
\includegraphics{img/infteofig.19}
\]

\begin{definizione}
Diremo che un insieme $\mathcal{A}$ \`e \emph{ricorsivo} quando la sua funzione
caratteristica $\varphi$:
\[
\varphi(n) = \left\{ \begin{array}{ll}
                      1 & \text{ se } n \ \in \ \mathcal{A}; \\
		      0 & \text{ altrimenti;} \\
		    \end{array}
            \right.
\]
\`e ricorsiva.

Similmente per una relazione fra numeri naturali diremo che
\`e \emph{ricorsiva} se la sua funzione caratteristica lo \`e: 
\[
\varphi(n, m) = \left\{ \begin{array}{ll}
                      1 & \text{ se }\langle n, m \rangle\ \in \ r; \\
		      0 & \text{ altrimenti.} \\
		    \end{array}
            \right.
\]
\end{definizione}
Tali definizioni relative a numeri naturali sono estese ai termini combinatori
considerando i numeri di G\"odel che li identificano.

\begin{definizione}
Diremo che un insieme $A$ di termini combinatori \`e \emph{chiuso rispetto 
all'uguaglianza (debole)} quando:
\[
X \in A \text{ e } Y = X \text{ implica } Y \in A.
\]
\end{definizione}

\begin{teorema}[inseparabilit\`a ricorsiva]
Nessuna coppia di insiemi $A$ e $B$ non vuoti di termini combinatori chiusi
rispetto all'uguaglianza debole \`e ricorsivamente separabile.
\end{teorema}
\begin{dimostrazione}
No.
\end{dimostrazione}
Contrariamente all'uso precedente di nozioni insiemistiche (che era evitabile),
ora le nozioni di insieme sono essenziali. In effetti le nostre definizioni e 
teoremi non sono propriamente della teoria dei combinatori.

\begin{corollario}[A]
Nessun insieme non banale (non vuoto o non totale) di termini combinatori che
sia chiuso rispetto all'uguaglianza (debole) pu\`o essere ricorsivo.
\end{corollario}

\begin{dimostrazione}
Basta prendere, nel teorema d'inseparabilit\`a ricorsiva, la coppia d'insiemi
formata da tale insieme e dal suo complemento (rispetto all'isieme di tutti i
termini combinatori).
\end{dimostrazione}

\begin{corollario}[B]
L'insieme di tutti i termini combinatori aventi forma normale non \`e 
ricorsivo.
\end{corollario}

\begin{dimostrazione}
Basta vedere che non \`e banale ed usare il corollario A (non \`e vuoto 
perch\`e gli atomi stanno in esso e non \`e totale poich\`e vi sono termini
combinatori senza forma normale).
\end{dimostrazione}
Si noti che questo corollario corrisponde al teorema dell'arresto.

\begin{corollario}[C]
La relazione di uguaglianza debole non \`e ricorsiva.
\end{corollario}

\begin{dimostrazione}
Basta considerare per assurdo l'insieme dei termini uguali a un dato termine
combinatorio. Allora dalla funzione caratteristica dell'uguaglianza debole 
possiamo costruire quella di tale insieme in modo che resti ricorsiva.
\end{dimostrazione}

%%
%%-- Capitolo - Introduzione al lambda calcolo
%%
\chapter{Introduzione al \lbc}
Per introdurre il \lbc~ l'idea di partenza \`e quella di effettuare 
un'inversione logica del teorema di riduzione dei termini combinatori:
\[
\bigl([x]M\bigr)N \ \vartriangleright \ [N/x]M;
\]
da questa si potevano trovare le costanti riducibili. 

Pertanto se incorporiamo 
nelle nostre definizioni tale asserto non avremo bisogno delle costanti
riducibili $\ii$, $\kk$, $\sss$. Il primo passo sar\`a quello di incorporare
l'astrazione combinatoria nei termini.

\section{\lbts}
Definiamo delle sequenze di variabili e di costanti per definire gli atomi, ma 
non richiediemo che fra le costanti ci siano  $\ii$, $\kk$, $\sss$. Pertanto
gli atomi potrebbero anche provenire solo dalla sequenza infinita delle 
variabili.

Definiamo i \lbts~ con le clausole:

\begin{enumerate}
\item[a -]Ogni atomo \`e un \lbt;
\item[b -]Se $X$ e $Y$ sono \lbts ~allora $XY$ \`e un \lbt;
\item[c -]Se $x$ \`e una variabile e $X$ \`e un \lbt, allora $\lambda x.X$ \`e
un \lbt;
\item[]($\lambda x.X$ \`e da pensare come $[x]X$).
\end{enumerate}
Nella clausola $b$ si definisce l'applicazione (come nel caso dei termini 
combinatori), nella clausola $c$ si definisce invece la $\lambda$-astrazione 
che non \`e un operatore esterno sui termini, ma \`e incorporato in essi. Tale
$\lambda$-astrazione \`e definita analogamente alla applicazione, cio\`e:
\begin{enumerate}
\item[-]$\lambda x'.X' \equiv \lambda x''.X''$ implica $x' \equiv x'' 
\text{ e } X' \equiv X''$;
\item[-]Solite clausole di disgiunzione\footnote{Atomi, \lbts ~composti per 
applicazione, \lbts ~composti per astrazione formano ``insiemi'' disgiunti.}.
\end{enumerate}
\begin{description}\item[Notazioni semplificative:]
\begin{enumerate}
\item[-]Parentesi sintattiche omesse se associate a sinistra (come per i 
termini combinatori); 
\item[-]Parentesi sintattiche per $\lambda$-astrazione omesse se associate a 
destra.
\end{enumerate}
\end{description}

\begin{esempio}
\[
X(((\lambda x.Y)X)Y) \text{ pu\`o diventare } X((\lambda x.Y)XY).
\]
\[
\lambda x.(\lambda y.(\lambda z.X)) \text{ pu\`o diventare } \lambda xyz.X
\quad \text{(come per $[x,y,z]X$)}. 
\]

\end{esempio}

Per visualizzare la struttura di un \lbt ~ora non bastano gli alberi binari
(con sola etichettatura delle foglie). Ci serviremo di alberi 
``al pi\`u binari'' con i nodi unari corrispondenti alle $\lambda$-astrazioni
etichettati dalla variabile di astrazione.

\begin{esempio}Alcune rappresentazioni grafiche.
\begin{figure}[!ht]
\[\includegraphics{img/infteofig.20}\]
\caption{$\bigl((\lambda x.y)(\lambda x.zy)\bigr)x$}
\end{figure}
\begin{figure}[!ht]
\[\includegraphics{img/infteofig.21}\]
\caption{$\lambda xyz.zy = \lambda x.\bigl(\lambda y.(\lambda z.zy)\bigr)$}
\end{figure}
\end{esempio}

\begin{definizione}
Diremo che $X$ \emph{occorre} in $Y$ e scriveremo $X \in Y$ quando ci\`o pu\`o
essere dedotto da:
\begin{enumerate}
\item[-]$X \in X$;
\item[-]Se $X \in U$ o $X \in V$ allora $X \in UV$;
\item[-]Se $X \in V$ allora per ogni $x$ variabile $X \in \lambda x.V$.
\end{enumerate}
\`E da notare che $x \notin \lambda x.y$ per ogni $y \neq x$.
\end{definizione}
Siccome $\lambda x.$ \`e da pensare come ``$[x]$'' e sappiamo che 
nell'astrazione combinatoria i nomi delle variabili d'astrazione non contano
($[x]X = [y]([y/x]X)$), allora dobbiamo distinguere le occorrenze di variabili 
a seconda che ``contino'' o no.

\begin{definizione}
Un'occorrenza di una variabile $x$ in un \lbt~ $Y$ si dir\`a \emph{vincolata}
---cio\`e ivi il nome di $x$ non conta--- quando tale occorrenza sia presente 
in un sottotermine di $Y$, che sia del tipo $\lambda x.X$ (cio\`e stia in $X$).
Se ogni occorrenza di $x$ in $Y$ \`e vincolata diremo che $x$ \`e 
\emph{vincolata} in $Y$. Se invece esiste almeno una occorrenza di $x$ in $Y$
che non sia vincolata diremo che $x$ \`e \emph{libera} in $Y$.
\end{definizione}
Con le nostre convenzioni di rappresentazione grafica tali nozioni sono pi\`u
semplici:
\begin{figure}[!ht]
\[\includegraphics{img/infteofig.22}\]
\caption{$X \in Y$ quando l'albero di $X$ \`e un sottoalbero di $Y$.}
\end{figure}
In particolare per $X \equiv x$ avremo una foglia.

Sempre considerando la rappresentazione grafica, 
possiamo notare che $x$ 
occorre libera in $Y$ quando da almeno una foglia etichettata con $x$ si pu\`o
risalire alla radice senza incontrare un nodo  (unario) etichettato con $x$.
Un'occorrenza di $x$ in $Y$ \`e vincolata se risalendo da tale foglia si
incontra almeno un tale nodo.

\[\includegraphics{img/infteofig.23}\]

\begin{definizione}
Si definisce la \emph{sostituzione} $[N/x]M$ \emph{del} \lbt ~$N$ \emph{al
posto} della variabile $x$ \emph{nel} \lbt ~$M$ con le clausole\footnote{
L'ultima clausola non \`e stata formalizata poich\'e rappresenta
un ``caso'' abbastanza complicato.} del tipo:
\begin{enumerate}
\item[-]$[N/x]x \equiv N$;
\item[-]$[N/x]a \equiv a$ per $a \not\equiv x$ atomo;
\item[-]$[N/x]UV \equiv \bigl([N/x]U\bigr)\bigl([N/x]V\bigr)$;
\item[-]$[N/x]\lambda x.X \equiv \lambda x.X$ (poich\`e le occorrenze di $x$ 
  sono vincolate);
\item[-]$[N/x]\lambda y.X \equiv \left\{ \begin{array}{l}
                      \text{si distinguono due casi a seconda che } y 
		      \text{ occorra in } N  \\
		      \text{(libera) in tal caso si rinomina } y. \\
		    \end{array} \right.$

\end{enumerate} 
\end{definizione}

\begin{definizione}
Diremo che un \lbt ~ $X$ \`e congruente ad un \lbt ~$Y$ e d'ora in poi 
scriveremo $X \equiv Y$ (non pi\`u identit\`a) quando si pu\`o ottenere $Y$ da
$X$ mediante una ``ridenominazione completa'' di variabili in occorrenze 
vincolate.
\end{definizione}
\begin{esempio}
$\lambda x.yx \equiv \lambda z.yz$
\end{esempio}
Coerentemente alla nostra interpretazione della $\lambda$-astrazione le
differenze tra \lbts ~congruenti saranno da ignorare. In sostanza il nostro
``$\equiv$'' (che non \`e un identit\`a) sar\`a da pensare equivalente 
all'identit\`a ``$\equiv$'' per i termini combinatori.

\begin{definizione}
Un \lbt ~della forma $(\lambda x.M)N$ che occorra in $X$ si dir\`a un 
\emph{redex} di $X$. Diremo sua contrazione il \lbt ~ $[N/x]M$.
Diremo che $X$ si riduce a $Y$ e scriveremo $X \vartriangleright Y$ (nella 
letteratura ci sono altre definizioni) quando $Y$ si ottenga da $X$ tramite
una sequenza finita di contrazioni. Formalmente quando ci\`o provenga da:
\[
(\lambda x.M)N \vartriangleright [N/x]M;
\]

\[
X \vartriangleright X;
\]
mediante le solite regole di deduzione.
\end{definizione}
Mediante simmetrizzazione e congruenza si definisce l'uguaglianza debole 
``$=$''.

Per $\lambda$-$\beta$ riduzione s'intende la relazione ottenuta aggiungendo 
all'assioma di contrazione precedente quello di cambiamento delle variabili
vincolate. Continueremo ad indicarla con ``$\vartriangleright$'', ma la useremo
solo nei teoremi, mentre negli esempi od esercizi useremo la riduzione senza
cambiamenti di variabili vincolate.

L'uguaglianza debole per \lbt ~viene anche chiamata $\lambda$-conversione o
$\lambda$-uguaglianza.

\begin{esempio}
Vediamo come esercizio alcune applicazioni di riduzioni partendo da redex.

\begin{equation}
\begin{array}{lcl}
(\lambda x.xy)F & \vartriangleright & [F/x]xy \\
                & \equiv & Fy. \\
\end{array}
\end{equation}

\begin{equation}
\begin{array}{lcl}
(\lambda x.y)F & \vartriangleright & [F/x]y \\
               & \equiv & y.  \\
\end{array}
\end{equation}

\begin{equation}
\label{c6e4}
\begin{array}{lcl}
\underbrace{\bigl(\lambda x.(\underbrace{\lambda y.yx)z}\bigr)v}
_{redex} 
           & \vartriangleright & \bigl(\lambda x.[z/y]yx\bigr)v \\
           & \equiv            & (\lambda x.zx)v \\
           & \vartriangleright & [v/x]zx \\
	   & \equiv            & zv.
\end{array}
\end{equation}
Nel caso di cui sopra abbiamo scelto di contrarre per primo il redex interno,
vediamo cosa succede se avessimo scelto il redex pi\`u esterno:
\begin{equation}
\begin{array}{lcl}
\underbrace{\bigl(\lambda x.(\underbrace{\lambda y.yx)z}\bigr)v}
_{redex} 
           & \vartriangleright & [v/x](\lambda y.yx)z \\
           & \equiv & (\lambda y.yv)z \\
           & \vartriangleright & [z/y]yv \\
           & \equiv & zv.
\end{array}
\end{equation}
Analogamente a quanto detto per i termini combinatori, anche nel \lbc, pur
seguendo strade diverse otteniamo il medesimo risultato.
\begin{equation}
\label{c6e5}
\begin{array}{lcl}
(\lambda x.xxy)(\lambda x.xxy) 
    & \vartriangleright & [\lambda x.xxy/x]xxy \\
    & \equiv            & (\lambda x.xxy)(\lambda x.xxy)y \\
    & \vartriangleright & [(\lambda x.xxy)/x]xxyy \\
    & \equiv            & (\lambda x.xxy)(\lambda x.xxy)yy \\
    & \vdots            & \vdots \\
    & \equiv            & (\lambda x.xxy)(\lambda x.xxy)y\cdots y. \\
\end{array}
\end{equation}
La riduzione   pu\`o continuare con contrazioni infinite.
\begin{equation}
\label{c6e6}
\begin{array}{lcl}
(\lambda x.xx)(\lambda x.xx) 
    & \vartriangleright & [\lambda x.xx/x]xx \\
    & \equiv            & (\lambda x.xx)(\lambda x.xx) \\
    & \vartriangleright & [(\lambda x.xx)/x]xx \\
    & \equiv            & (\lambda x.xx)(\lambda x.xx) \\
    & \vdots            & \vdots \\
    & \equiv            & (\lambda x.xx)(\lambda x.xx). \\
\end{array}
\end{equation}
La riduzione   pu\`o continuare con contrazioni infinite (effettive), ma di
fatto ``non contraggono''.
\end{esempio}
Da tali esempi si vede come si definisce per il \lbt~ la forma normale o la
sua assenza come in \ref{c6e5} e \ref{c6e6}. L'equazione \ref{c6e4} ci 
preannuncia il primo teorema di Church-Rosser per i \lbts~ (che non 
dimostreremo) cos\`i come il corollario sull'unicit\`a della forma normale, che
vale a meno di congruenze e che ha la stessa dimostrazione vista.
Vale pure il secondo teorema di Church-Rosser (con la medesima dimostrazione)
sempre a meno di congruenze e relativi corollari.

Come per i termini combinatori non vale la regola di estensionalit\`a.

\begin{esempio}Non estensionalit\`a. 
\[
X \equiv \lambda x.yx, \ Y \equiv y \ \text{ allora per ogni \lbt} \ V, \
XV = yV \equiv YV.
\]
Tuttavia $X \neq Y$ poich\`e sono in forma normale e non congruenti.
\end{esempio}

\section{Alberi di De Bruijn}
Gli Alberi di De Bruijn costituiscono una definizione alternativa dei \lbts~
che evita il problema delle variabili vincolate. Quindi non c'\`e bisogno di
introdurre congruenze, la riduzione debole diventa la $\lambda$-$\beta$ 
riduzione. Oltre a tali vantaggi per la teoria essi possono servire 
nell'implementazione del linguaggio LISP che accenneremo, poich\`e permettono
controlli di congruenza immediati.

Ritorniamo alla rappresentazione grafica di un \lbt

\begin{figure}[!ht]
\[\includegraphics{img/infteofig.24}\]
\caption{$\bigl(\lambda x.x(\lambda y.yx)\bigr)z$}
\end{figure}
e consideriamo il ruolo delle variabili.
\begin{enumerate}
\item[-]La $z$ \`e una variabile libera che possiamo considerare come la prima
nella successione delle variabili libere. Quindi la possiamo chiamare $1$.
\item[-]La $x$ \`e invece vincolata (come la $y$). Il suo ruolo nell'albero \`e
identificato dal nodo unario etichettato $x$. Se vogliamo ignorare le 
congruenze , non importa che sia stata chiamata $x$, a tale fine non \`e 
necessario avere tale nodo, basta sapere dov'\`e. Tale posizione \`e 
univocamente determinata dal numero di nodi binari che dobbiamo risalire dalla
foglia nominati $x$ per raggiungere tale nodo unario, cio\`e dal \emph{livello
d'astrazione}. Nel nostro caso \`e $1$ o $2$. Possiamo allora convenire che
tale occorrenza di $x$ nella foglia sia indicata dall'opposto $-1$ di tale 
livello d'astrazione.
\item[-]Per le $y$ valgono ancora le osservazioni precedenti e continuiamo a 
chiamarla $-1$.
\end{enumerate}
Pertanto il nostro albero al pi\'u binario diventa un puro albero binario con
foglie etichettate da numeri interi.

\begin{figure}[!ht]
\[\includegraphics{img/infteofig.25}\]
\caption{Albero di De Bruijn.}
\end{figure}

Si noti che per le variabili libere possiamo stabilire una corrispondenza 
biunivoca fra i loro nomi e l'intero positivo che la identifica. Ci\`o non vale
per le variabili vincolate. Nel nostro esempio il nome $x$ corrisponde sia a
$-1$ che a $-2$, invece il numero $-1$ corrisponde sia al nome $x$ che al nome
$y$.

Anche se la dimostrazione \`e un po' complicata \`e chiaro che da ogni albero 
al pi\`u binario di un \lbt~corrisponde un unico albero del secondo tipo che
chiameremo \emph{albero di De Bruijn}. Viceversa dato un albero binario con
foglie etichettate da interi  possiamo sempre costruire almeno un albero del
primo tipo con opportune altre convenzioni Si pu\`o dimostrare che tutti gli 
alberi cos\`i costruiti sono quelli di \lbts~congruenti.

\begin{esempio}Vediamo alcune trasformazioni.
\begin{figure}[!ht]
\[\includegraphics{img/infteofig.26}\]
\caption{$(\lambda x.x)y$}
\end{figure}
\begin{figure}[!ht]
\[ \includegraphics{img/infteofig.27} \]
\caption{altra possibile interpretazione, $(\lambda z.z)y$.}
\end{figure}

Similmente, convenendo che in mancanza di nodi binari si conteggino nodi 
d'astrazione, si ha che:

\[ \includegraphics{img/infteofig.28} \]

ci da ad esempio $\lambda xwyuv.xyv$:


\[ \includegraphics{img/infteofig.29} \]

\end{esempio}

In conclusione possiamo ridefinire i \lbts~a meno di congruenze, semplicemente
come alberi binari su foglie di interi qualsiasi. L'uso delle variabili
vincolate \`e solo per facilit\`a di comprensione.

%%--Section Uguaglianza estensionale
\section{Uguaglianza estensionale}
Sinora l'uguaglianza per i termini combinatori e i \lbts~era debole poich\'e
non valeva la regola di estensionalit\`a \ref{est}. Tale uguaglianza \`e 
sufficiente per i risultati visti (teorema di Church-Rosser e conseguenze,
esempi di ``consistenza logica'' ed altri), tuttavia non \`e sufficiente per 
dimostrare l'equivalenza tra \lbc~e calcolo dei combinatori, come vedremo.

Si noti che possiamo considerare i \lbts~ed i termini combinatori come
programmi idealizzati per un linguaggio di programmazione senza o con 
primitive\footnote{Nel caso di \lbts~in luogo di primitive abbiamo la 
$\lambda$-astrazione che pu\`o essere considerata come la generazione di una
procedura con passaggio di parametri secondo la nostra definizione di 
contrazione $(\lambda x.M)N \vartriangleright [N/x]M$.}.

Con ``$=$'' identifichiamo cos\`i, in ambedue i casi, un'uguaglianza di
programmi idealizzati comprensivi dei loro ``dati''. Tale equivalenza asserisce
che due programmi o hanno lo stesso risultato (forma normale eventualmente a 
meno di congruenze) oppure il loro calcolo pu\`o essere ricondotto sempre allo
stesso punto nel senso del secondo teorema di Church-Rosser.

Invece un'uguaglianza che soddisfi la regola (est) riguarda programmi non 
necessariamente completi dei loro dati (in est il dato \`e $V$). Quindi mentre
in Teoria degli Insiemi si ha una sola nozione di uguaglianza, proveniente 
dall'assioma di estensionalit\`a, l'uguaglianza dal punto di vista operativo
o \emph{intensionale} si divide in due diverse nozioni.

Consideriamo ora tre regole diverse da (est).

\begin{enumerate}
\item[$(\zeta)$]$Xx = Yx$ per ogni variabile $x$ che non occorra in $X$ o $Y$,
implica $X = Y$.
\end{enumerate}
\`E una semplificazione della (est), ma come quest'ultima ha una premessa 
infinita poich\'e le variabili che non occorrono nei due termini sono infinite,
tuttavia si pu\`o considerare come una singola premessa.
Tale regola pu\`o riguardare sia \lbts~che termini combinatori.
\begin{enumerate}
\item[$(\xi)$]$X = Y$ implica $[x]X = [x]Y$ per ogni variabile $x$ (che 
 riguarda solo i termini combinatori).
\end{enumerate}
A differenza della (est) e ($\zeta$) la premessa \`e singola. Infine per i 
\lbts~abbiamo:
\begin{enumerate}
\item[$(\eta)$]$\lambda x.Xx = X$ per ogni variabile $x$ che non occorra in 
$X$.
\end{enumerate}
Si noti che ($\eta$) non \`e un implicazione e pu\'o essere essere aggiunta 
come assioma ad una definizione di uguaglianza. In pratica essa amplia il tipo
dei redex da considerare. Essa corrisponde alla terza clausola della 
definizione di astrazione per i termini combinatori.

\begin{teorema}
La regola (est) \`e equivalente a: ($\zeta$) nel caso dei \lbts~o termini 
combinatori, ($\xi$) nel caso dei termini combinatori ed ($\eta$) nel caso dei
\lbts.
\end{teorema}
\begin{dimostrazione}
No.
\end{dimostrazione}
Nell'ultimo caso l'uguaglianza ottenuta (fra \lbts) viene detta
\emph{$\lambda \beta \eta$ convertibilit\`a}.

%%--Section Riduzione forte (cenno)
\section{Riduzione forte (cenno)}
Nel caso dell'uguaglianza debole potevamo caratterizzarla come la chiusura
simmetrica della riduzione ``$\vartriangleright$'' (debole per i termini
combinatori o $\lambda \beta$ per i \lbts). Ci  si pu\`o chiedere se esista
un'analoga relazione ``$\succ$'' per il caso dell'uguaglianza estensionale. La
risposta \`e affermativa e la ``$\succ$'' viene detta riduzione forte. Pertanto
si pu\`o pensare all'uguaglianza forte come ad un'equivalenza generata da un
``processo di calcolo''. Tuttavia tale processo \`e pi\`u complicato delle 
varie riduzioni.

In corrispondenza alle uguaglianze forti si hanno  le varianti delle nozioni 
relative alla forma normale dei teoremi di Church-Rosser e d'inseparabilit\`a
ricorsiva sia per i \lbts~che per i termini combinatori. Con alcune limitazioni
si avranno anche delle varianti dei risultati di decidibilit\`a che vedremo in
seguito.

Ci\`o permette di utilizzare le uguaglianze estensionali per risolvere 
problemi di decidibilit\`a che nascono in Teoria degli Insiemi a patto di
poterli tradurre in problemi relativi a \lbts~e termini combinatori\footnote{
Tale traduzione pu\`o essere assai difficile.}.

%%--Section Equivalenza tra \lbc e calcolo dei combinatori
\section{Equivalenza tra \lbc~e calcolo dei combinatori}
In questa parte del testo verificheremo l'equivalenza tra \lbc~e calcolo dei 
combinatori, per facilitare la notazione il simbolo ``$=$'' indicher\`a solo
l'uguaglianza estensionale.

\begin{definizione}
Dato un \lbt~$X$ diremo suo \emph{H-trasformato} il termine combinatorio
$X_H$, ottenuto per induzione combinatoria, tale che:
\begin{enumerate}
\item[-]$a_H \equiv a$ per ogni atomo\footnote{Si suppone che tutti gli atomi
siano gli stessi nei due sistemi (\lbc~e calcolo dei combinatori), salvo le 
costanti riducibili.};
\item[-]$(UV)_H \equiv U_H V_H$;
\item[-]$(\lambda x.X)_H \equiv [x]X_H$.
\end{enumerate}
\end{definizione}

\begin{definizione}
Dato un termine combinatorio $X$ diremo suo $\lambda$-tras\-for\-ma\-to il 
\lbt~$X_\lambda$ tale che:
\item[-]$a_\lambda \equiv a$ per ogni atomo diverso dalle costanti riducibili;
\item[-]$\ii_\lambda \equiv \lambda x.x$;
\item[-]$\kk_\lambda \equiv \lambda xy.x$;
\item[-]$\sss_\lambda \equiv \lambda xyz.xz(yz)$;
\item[-]$(UV)_\lambda \equiv U_\lambda V_\lambda$. 
\end{definizione}

\begin{esempio}Vediamo alcune H e $\lambda$ trasformazioni.

\[
\begin{array}{lcl}
\bigl(y(\lambda x.xzy)\bigr)_H
   & \equiv & y_H (\lambda x.xzy)_H \\
   & \equiv & y\bigl([x](xzy_H \bigr) \\
   & \equiv & y\bigl([x]xzy\bigr) \\
   & \equiv & y\bigl(\sss([x]xz)(\kk y)\bigr) \\
   & \equiv & y\Bigl(\sss \bigl( \sss \ii (\kk z)\bigr) (\kk y) \Bigr). \\
\end{array}
\]

\[
\begin{array}{c}
   \biggl(y\Bigl(\sss \bigl( \sss \ii (\kk z)\bigr) (\kk y) 
              \Bigr)\biggr)_\lambda  \\
    \equiv \\ y_\lambda
             \Bigl(\sss \bigl( \sss \ii (\kk z)\bigr) (\kk y) \Bigr)_\lambda \\
    \equiv \\ y\biggl( \Bigl(\sss \bigl( \sss \ii (\kk z)\bigr) \Bigr)_\lambda
              (\kk y)_\lambda \biggr) \\
    \equiv \\ y\biggl(\Bigl( \sss_\lambda \bigl( \sss \ii (\kk z)\bigr)_\lambda
              \Bigr) (\kk_\lambda y_\lambda) \biggr) \\
    \equiv \\ y\biggl(\Bigl(\bigl(\lambda xyz.xz(yz)\bigr) 
              \bigl((\sss \ii)_\lambda (\kk z)_\lambda \bigr) \Bigr)
              \bigl((\lambda xy.x)y\bigr) \biggr) \\
    \equiv \\ y\biggl(\Bigl(\bigl(\lambda xyz.xz(yz)\bigr) 
              \bigl((\sss_\lambda \ii_\lambda) (\kk_\lambda z_\lambda)
              \bigr) \Bigr)\bigl((\lambda xy.x)y\bigr) \biggr) \\
    \equiv \\ y\Biggl(\biggl(\bigl(\lambda xyz.xz(yz)\bigr) 
              \Bigl(\bigl((\lambda xyz.xz(yz)) (\lambda x.x)\bigr) \bigl(
             (\lambda xy.x)z\bigr) \Bigr)\bigl((\lambda xy.x)y\bigr) \Biggr).\\
\end{array}
\]
Pur essendo partiti dall'H-trasformazione di $\bigl(y(\lambda x.xzy)\bigr)$
\emph{non} abbiamo riottenuto lo stesso termine applicando la 
$\lambda$-trasformazione al risultato intermedio. Potrebbe valere l'uguaglianza
debole?
\end{esempio}
Per verificarne l'uguaglianza (debole) dovremmo cercare la forma normale del
risultato della seconda equazione, il calcolo \`e lungo, tuttavia possiamo
considerare un'occorrenza di tale risutato: $(\lambda xy.x)y$.
Per riduzione e cambiamento di variabile si ottiene $\lambda u.y$ che non pu\`o
essere congruente ad alcun sottotermine nel \lbt~di partenza, rappresentato 
graficamente da:


\[ \includegraphics{img/infteofig.30} \]
mentre la nostra occorrenza \`e rappresentata da: 
\[ \includegraphics{img/infteofig.31} \]

Tale occorrenza potrebbe scomparire per effetto di altre riduzioni, ma in tal
caso scomparirebbe anche la variabile libera $y$ contrariamente al fatto che 
nell'albero sia la seconda occorrenza libera di $y$.
In definitiva possiamo concludere che $\Bigl(\bigl(y(\lambda x.xzy)\bigr)_H 
\Bigr)_\lambda$ non \`e debolmente uguale a $y(\lambda x.xzy)$

\begin{teorema}(di equivalenza)

\begin{enumerate}
\item[-]Se $X$ e $Y$ sono termini combinatori, allora:
\[X = Y\text{ se e soltanto se }
X_\lambda = Y_\lambda.\]
\item[-]Se $X$ e $Y$ sono \lbts, allora:
\[X = Y \text{ se e soltanto se }
X_H = Y_H.\]
\item[-]Se $X$ \`e termine combinatorio, allora: \[(X_\lambda)_H \equiv X.\]
\item[-]Se $X$ \`e \lbt, allora: \[(X_H)_\lambda \equiv X.\]
\end{enumerate}
\end{teorema}
Tale teorema ``aggiusta'' l'esempio precedente.
\begin{dimostrazione}
No.
\end{dimostrazione}

Quindi se consideriamo i termini combinatori come programmi di un linguaggio di
programmazione idealizzato con primitive (\II, \KK, \SSS) e i \lbts~come
programmi senza primitive (ma con procedure richiamabili con parametri), 
vediamo che i due linguaggi idealizzati sono equivalenti a patto di considerare
tali programmi come non completi dei loro dati (uguaglianza estensionale).

%%
%%-- Capitolo - Sistemi combinatori alternativi
%%
\chapter{Sistemi combinatori alternativi}
Abbiamo implicitamente visto che il sistema di termini combinatori basato sulle
costanti riducibili \II, \KK, \SSS~potrebbe essere sostituito, a meno di
uguaglianze estensionali, da quello basato solo da \KK~e \SSS.

Di sistemi alternativi, in tal senso, ve ne sono parecchi altri. Uno di 
particolare importanza \`e il sistema \BB-\CC-\KK-\WW, ove tali costanti
riducibili sono definite dai seguenti assiomi:

\begin{enumerate}
\item[(B)]$\quad\bb XYZ \vartriangleright X(YZ)$;
\item[(C)]$\quad\cc XYZ \vartriangleright XZY$;
\item[(K)]$\quad\kk XY  \vartriangleright X$;
\item[(W)]$\quad\ww XY  \vartriangleright XYY$.
\end{enumerate}

Per vedere l'equivalenza di tale sistema con i precedenti, si potrebbe per 
esempio usare l'equivalenza con il \lbc, ma basta vedere con i seguenti esempi 
che si pu\`o definire sempre 
l'astrazione di un termine in modo che valga il teorema di riduzione.

\begin{esempio}
Cerchiamo $[x, y, z]xz(yz)$ che, se varr\`a il teorema di riduzione, ci dar\`a
\SSS. 

\begin{osservazione}Se manteniamo nel nuovo sistema la terza clausola 
dell'astrazione: 
$$[x]Mx \equiv M,$$ con $x \notin M$, ci baster\`a costruire un 
termine $S'$ tale che $S'xyz = xz(yz)$ e tale che $x, y, z \notin S'$. 
Infatti per la terza clausola varr\`a  che $S' = [x, y, z]xz(yz)$ e varr\`a il 
teorema di riduzione.
\end{osservazione}
Gli assiomi precedenti se letti all'incontrario ci permettono tale costruzione.
Infatti $(B)$ consente di eliminare le parentesi sintattiche, $(C)$ di 
scambiare fra loro dei sottotermini, $(K)$ di ``inventare'' termini che non
c'erano e $(W)$ di eliminare doppie occorrenze (adiacenti nell'applicazione)di 
termini.
\end{esempio}
Vediamo come ottenere $S'$.
\[
\begin{array}{lcl}
xz(yz) & = & \cc x(yz)z \\
       & = & \bb (\cc x)yzz \\
       & = & \ww \bigl( \bb (\cc x)y\bigr)z \\
       & = & \ww ( \bb \bb \cc xy)z \\
       & = & \bb \ww (\bb \bb \cc x)yz \\
       & = & \bb (\bb \ww)(\bb \bb \cc)xyz. \\
\end{array}
\]
Quindi otteniamo $S' \equiv \bb (\bb \ww)(\bb \bb \cc)$ e tale $S'$ avr\`a
la propriet\`a $(S)$.

\section{Uso del sistema \BB-\CC-\KK-\WW}
L'esprimere un termine combinatorio nel sistema \BB-\CC-\KK-\WW~(con la 
procedura accennata) ha dei vantaggi quando si voglia usare il calcolo 
combinatorio in Matematica (in particolare in Algebra). 

La Matematica solita \`e (almeno idealmente) fondata sulla Teoria degli 
Insiemi. I primi sviluppi a partire da tale fondazione sono di tipo 
estensionale: di un oggetto (di solito funzione) ci si cura di pi\`u delle
propriet\`a (che lo collocano in un estensione di oggetti simili) che non di 
come sia costruito.

Il come \`e costruito pu\`o essere trovato insiemisticamente (con parecchia
fortuna) oppure tramite il calcolo dei combinatori ``intensionalmente''. Nel
secondo modo, che comunque pu\`o servire ad ispirare il primo, si hanno a 
disposizione metodi standard (teorema di riduzione).

Per quanto il sistema \II, \KK, \SSS, appaia pi\`u semplice da usare rispetto
al \BB-\CC-\KK-\WW, esso \`e meno adatto per ottenere i risultati di cui 
sopra. Ci\`o dipende dal fatto che il parallelizzatore \SSS~ha meno propriet\`a
``categoriche'' dei \BB, \CC, \WW. Accenneremo in seguito a tali 
propriet\`a. Queste propriet\`a consentono di trovare propriet\`a algebriche
degli oggetti costruiti.

\section{Possibile uso di sistemi combinatori alternativi}
Se in ogni termine combinatorio del sistema \KK, \SSS~sostituissimo \SSS~con il
termine che lo rappresenta nel sistema \BB-\CC-\KK-\WW, allora otterremmo un
sistema equivalente al secondo. In sostanza avremmo fatto una ``sostituzione 
multipla'': $$[\kk / \kk,\ \bb (\bb \ww ) ( \bb \bb \cc ) / \sss ]$$ sulle 
costanti riducibili
anzich\`e sulle variabili. In pratica tali ``sostituzioni'' servono a tradurre
un programma idealizzato nel primo linguaggio di programmazione (sistema \II, 
\KK, \SSS) in uno relativo al secondo (sistema \BB-\CC-\KK-\WW).

Pertanto abbiamo un esempio di traduttore di linguaggi di programmazione
idealizzati. L'uso di tali ``traduttori ideali'' pu\`o servire a studiare 
propriet\`a dei traduttori reali. Pu\`o anche servire a definire, in ambito
puramente linguaggistico, nozioni equivalenti a quelle note dalla Teoria di 
Turing (per esempio l'universalit\`a) che sono espresse tramite agenti di 
calcolo.

\section{Sistemi di riscrittura}
Quale che sia il sistema combinatorio di riferimento, abbiamo una nozione di
riduzione espressa, oltre che da regole di deduzione, da assiomi del tipo:
\[
\Gamma X' X'' \cdots X^{(n)} \vartriangleright T,
\] 
ove $T$ \`e un termine formato dagli $X^{(i)}$.

In realt\`a per definire una nozione di calcolo ``$\vartriangleright$'', non
\`e necessario che gli assiomi siano fatti in tal modo\footnote{Per esempio che
la costante riducibile $\Gamma$ stia in testa o che si abbia un'applicazione
binaria};  infatti nei linguaggi di programmazione effettivamente usati tale
restrizione non esiste. D'altra parte il tipo di risultati che si ottengono con
il calcolo combinatorio, per dei linguaggi idealizzati, \`e molto interessante
anche per quelli effettivamente utilizzati.

Per rispondere a tale esigenza sono stati definiti i ``sistemi di riscrittura''
(rewriting systems) in cui gli assiomi possono essere scritti pi\`u 
liberamente.
In tali sistemi si richiede che valga comunque il teorema di Church-Rosser, in
modo che la relazione ``$\vartriangleright$'' abbia quelle carateristiche di
``quasi funzionalit\`a'' che l'idea di calcolo richiede.
\begin{description}
\item[Nota:]i sistemi di riscrittura \emph{non} sono imparentati con i 
``sistemi di riscrittura  parallela'' che riguardano si l'informatica teorica, 
ma solo quella algebrica.
\end{description}

\begin{ese}
Trovare nel sistema \BB-\CC-\KK-\WW~un combinatore equivalente al ``$\Phi$'',
visto in \ref{Phi}, tale che:
\[
\Phi XYZU = X(YU)(ZU).
\]

\[
\begin{array}{lcl}
x(yu)(zu) & \stackrel{(B)}{\equiv} & \bb xyu(zu) \\
          & \stackrel{(C)}{\equiv} & \cc (\bb xy)(zu)u \\
          & \stackrel{(B)}{\equiv} & \bb \bigl(\cc (\bb xy)\bigr)zuu \\
          & \stackrel{(W)}{\equiv} & \ww \Bigl(\bb\bigl(\cc(\bb xy)\bigr)\Bigr)
                                     zu \\
          & \stackrel{(B)}{\equiv} & \ww\Bigl(\bb\bigl(\bb\cc(\bb x)y\bigr)
                                     \Bigr)zu \\
          & \stackrel{(B)}{\equiv} & \bb\ww\bb\bigl(\bb\cc(\bb x)y\bigr)zu \\
          & \stackrel{(B)}{\equiv} & \bb\ww\bb\bigl(\bb(\bb\cc)\bb xy\bigr)zu 
                                     \\
          & \stackrel{(B)}{\equiv} & \bb\ww\bb(\bb\bb\bb\cc\bb xy)zu \\
          & \stackrel{(B)}{\equiv} & \bb(\bb\ww\bb)\bb\bb\bb\cc\bb xyzu \\
          & \stackrel{(B)}{\equiv} & \bb\bb\bb\ww\bb\bb\bb\bb\cc\bb xyzu. \\
\end{array}
\]

Togliendo le variabili $xyzu$ si ha un tale $\Phi '$.
\end{ese}

%%
%%-- Capitolo - Tipi e termini con tipo
%%
\chapter{Tipi e termini con tipo}
In Teoria degli Insiemi si considera la nozione di appartenenza, tramite tale 
nozione ``$x \in X$'' se ne ottengono altre: $f\colon A \to B$ o $f\colon A^X
\to B^X$ etc. Nella logica combinatoria tale appartenenza non c'\`e. Si ha 
invece la nozione (operativa) di tipo.

Si assume l'esistenza di una sequenza (non vuota) di \emph{tipi} detti \emph{
fondamentali}: $\alpha$, $\beta$, $\gamma$, $\ldots$ . Un tipo fondamentale 
pu\'o essere correttamente inteso come l'indicazione dove trovare qualcosa, per
esempio l'etichetta di un supporto dati, il nome di una periferica, etc.
Non correttamente (ma utilmente) lo si pu\`o interpretare come un insieme di 
base (con cui poi costruire altri insiemi mediante esponenziazione).

\begin{definizione}
I tipi sono defini dalle seguenti clausole.
\begin{itemize}
\item[-]Ogni tipo fondamentale \`e un tipo;
\item[-]Se $\alpha$ e $\beta$ sono tipi, allora $F\alpha\beta$ \`e un tipo.
\end{itemize}
Ivi $F\alpha\beta$ indica un'applicazione diversa da quella per i termini 
combinatori, ma definita allo stesso modo (cio\`e si scrive $F$ come funzione
solo per agevolare la scrittura).
\end{definizione}
Per attribuire tipi a dei termini ci sono due modi parziali e uno 
completo\footnote{Che non faremo, ma si trova sul testo ``non consigliato'' e
successivo di Hindley e altri.}. 

Consideriamo ora il primo modo; consiste nel ridefinire i termini combinatori
(si pu\`o fare anche per i \lbts~ma non lo vedremo) in modo che vengano 
generati con un tipo fisso.

Si assume per ogni tipo $\mathcal{t}$ una successione infinita di variabili
e per indicare che una variabile $x$ \`e di tipo $\mathcal{t}$ scriveremo
$x^\mathcal{t}$. Similmente per ogni costante non riducibile $a$ avremo 
$a^\mathcal{t}$. Per ogni tipo $\alpha$ avremo una costante (riducibile) 
$\ii_\alpha$ alla quale attribuiremo il tipo $F\alpha\alpha$. Per ogni coppia
di tipi $\alpha$, $\beta$ avremo una costante $\kk_{\alpha,\beta}$ a cui
attribuiremo tipo $F\alpha(F\beta\alpha)$. Infine per ogni terna di tipo
$\alpha$, $\beta$, $\gamma$ avremo una costante $\sss_{\alpha,\beta,\gamma}$ 
cui attribuiamo tipo 
$F(F\alpha(F\beta\gamma))(F(F\alpha\beta)(F\alpha\gamma))$.

Si suppone che tutti questi oggetti, che chiamiamo \emph{atomi} siano distinti.

\begin{definizione}
I termini con tipo sono definiti dalle seguenti clausole.
\begin{itemize}
\item[-]Ogni atomo \`e un termine con tipo;
\item[-]Se $U$ \`e un termine con tipo $F\alpha\beta$ e $V$ \`e un termine con
tipo $\alpha$, allora $UV$ \`e un termine di tipo $\beta$.
\end{itemize}
\end{definizione}
Verranno considerati solo termini costruiti come sopra (nel primo modo 
par\-zia\-le). Vale la solita definizione di applicazione con le clausole di
disgiunzione.

\begin{description}
\item[Nota:]spesso si hanno tipi complicati come
$F\alpha_1(F\alpha_2(\ldots(F\alpha_n \beta)\ldots))$, tale scrittura si
semplifica nel seguente modo: $F_n \alpha_1 \alpha_2 \cdots \alpha_n \beta$.
Per esempio possiamo scrivere $F_2 \alpha\beta\alpha$ come tipo per
$\kk_{\alpha,\beta}$ e $F_3 (F_2 \alpha\beta\gamma)(F\alpha\beta)\alpha\gamma$
come tipo di $\sss_{\alpha,\beta,\gamma}$.
\end{description}

\begin{lemma}
Per ogni tipo $\alpha$, $\alpha \not\equiv F\alpha\beta$, $F\beta\alpha$.
\end{lemma}
\begin{dimostrazione}
Nel caso $\alpha$ sia tipo fondamentale ci\'o deriva dall'ipotesi di 
disgiunzione implicita nella definizione: $F\alpha\beta$ \`e un'applicazione.
Altrimenti si procede per induzione binaria secondo la definizione di tipo
(in sostanza si ritorna al caso fondamentale).
\end{dimostrazione}

\begin{definizione}
La sostituzione $[Y^\alpha / x^\alpha]X$ \`e definita come per il \lbts, ma
con la restrizione che la variabile e il termine sostituito abbiano lo stesso
tipo.
\end{definizione}

\begin{definizione}
Diremo che un termine combinatorio $X$ si riduce (debolmente) a un termine
combinatorio $Y$ e scriveremo $X \vartriangleright Y$ quando ci\`o pu\`o essere
dedotto dagli assiomi:
\begin{itemize}
\item[($I_\alpha$)]$\ii_\alpha X^\alpha \vartriangleright X^\alpha$.
\item[($K_{\alpha,\beta}$)]$\kk_{\alpha,\beta}X^\alpha Y^\beta 
\vartriangleright X^\alpha$.
\item[($S_{\alpha,\beta,\gamma}$)]$\sss_{\alpha,\beta,\gamma} 
X^{F\alpha(F\beta\gamma)}Y^{F(F\alpha\beta)(F\alpha\gamma)}Z^\alpha
\vartriangleright XZ(YZ)$ che tipo avr\`a?
\item[]Mediante le regole di deduzione:
  \begin{itemize}
    \item[-]Assioma di riflessivit\`a.
    \item[-]$X^{F\alpha\beta} \vartriangleright Y^{F\alpha\beta}$ implica
            $X^{F\alpha\beta}Z^\alpha \vartriangleright 
            Y^{F\alpha\beta}Z^\alpha$.
    \item[-]$X^\alpha \vartriangleright Y^\alpha$ implica $Z^{F\alpha\beta}X
            \vartriangleright Z^{F\alpha\beta}Y^\alpha$.
    \item[-]Transitivit\`a.
\end{itemize}
\end{itemize}
\end{definizione}
Sostituendo ``$=$'' al simbolo ``$\vartriangleright$'' e aggiungendo la regola
di simmetria si definisce l'uguaglianza debole ($=$).

\begin{teorema}
Se $X = Y$, allora $X$ e $Y$ hanno lo stesso tipo.
\end{teorema}

\begin{dimostrazione}
Applichiamo le definizioni di uguaglianza ($=$) considerando dapprima gli 
assiomi.
\begin{itemize}
\item[($I_\alpha$)]Immediata dalla clausola di composizione.
\item[($K_{\alpha,\beta}$)]Quasi immediata. $\kk_{\alpha,\beta}$ ha tipo 
  $F\alpha(F\beta\alpha)$ e $X$ ha tipo $\alpha$ per cui 
  $\kk_{\alpha,\beta}X^\alpha$ ha tipo $F\beta\alpha$ e 
  $\kk_{\alpha,\beta}X^\alpha Y^\beta$ avr\`a tipo $\alpha$ come il secondo 
  membro dell'applicazione ($X^\alpha$).
\item[($S_{\alpha,\beta,\gamma}$)]Consideriamo dapprima il secondo membro 
  $XZ(YZ)$. Per essere un termine combinatorio con tipo (che indicheremo con
  $\gamma$) il primo componente $XZ$ dovr\`a avere tipo  $F\beta\gamma$ per 
  qualche $\beta$. Affinch\`e ci\`o sia possibilie $X$ dovr\`a avere tipo 
  $F\alpha(F\beta\gamma)$ per qualche $\alpha$. Similmente affinch\`e $YZ$
  abbia tipo $\beta$, $Y$ dovr\`a avere tipo $F\delta\beta$ ove $\delta$ sia
  il tipo di $Z$, ma per la costruzione di $XZ$ tale $\delta$ \`e $\alpha$,
  cio\`e $Y$ ha tipo $F\alpha\beta$. In definitiva:
  \[\begin{array}{rl}
    Z & \text{ha tipo } \alpha. \\
    Y & \text{ha tipo } F\alpha\beta. \\
    X & \text{ha tipo } F\alpha(F\beta\gamma). \\
    XZ(YZ) & \text{ha tipo } \gamma. \\
  \end{array}\]

  Passiamo ora al primo membro dell'uguaglianza, che dovr\`a risultare di tipo
  $\gamma$. Avendo assegnato 
  $F(F\alpha F\beta\gamma)(F(F\alpha\beta)(F\alpha\gamma))$ a 
  $\sss_{\alpha,\beta,\gamma}$ per la clausola di composizione (e per il tipo
  di $X$) si ha che $\sss_{\alpha,\beta,\gamma}X$ \`e termine combinatorio con
  tipo che risulta $F(F\alpha\beta)(F\alpha\gamma).$ Da qui troviamo il tipo di
  $\sss_{\alpha,\beta,\gamma}XY$ allo stesso modo, cio\`e $F\alpha\gamma$.
  Come prima si trova il tipo desiderato di $\sss_{\alpha,\beta,\gamma}XYZ$, 
  ovvero $\gamma$.
\end{itemize}
Infine \`e immediato controllare che tale conservazione del tipo viene 
rispettata dalle regole di deduzione.
\end{dimostrazione}

\begin{definizione}
Data una variabile $x^\alpha$ e un termine $X^\beta$ definiamo l'astrazione
$[x]X$ con le segenti clausole:
\begin{itemize}
\item[-]$[x]x \equiv \ii_\alpha$
\item[-]$[x]M \equiv \kk_{\beta,\alpha}M$ ove $x \notin M$ e $\beta$ sia il 
  tipo di $M$
\item[-]$[x]Ux \equiv U$ ove $x \notin U$
\item[-]$[x]UV \equiv \sss_{\alpha,\delta,\beta}([x]U)([x]V)$ ove $\delta$
  sia il tipo di $V$ e $F\delta\beta$ il tipo di $U$.
\end{itemize}
\end{definizione}

\begin{lemma}
$[x^\alpha ]X^\beta$ ha tipo $F\alpha\beta$.
\end{lemma}
\begin{dimostrazione}
Per induzione combinatoria su $X$ (seguendo il raggruppamento della 
definizione).
\begin{itemize}
\item[a)]$[x^\alpha]X^\beta \equiv \ii_\alpha$ che ha tipo $F\alpha\alpha$,
  ma $X^\beta \equiv x^\alpha$, cio\`e $\beta \equiv \alpha$.
\item[b)]$[x^\alpha]X^\beta \equiv \kk_{\beta,\alpha}X^\beta$ che, per clausole
  combinatorie\footnote{$\kk_{\beta,\alpha}$ ha tipo $F\beta(F\alpha\beta)$.}
  ha tipo $F\alpha\beta$.
\item[c)]$[x^\alpha]X^\beta \equiv U$ poich\'e $X^\beta \equiv Ux$, avendo $x$ 
  tipo
  $\alpha$, per le solite clausole $U$ deve avere tipo $F\beta\alpha$.
\item[d)]$[x^\alpha]X^\beta \equiv \sss_{\alpha,\delta,\beta}([x]U)([x]V)$.

 In questo caso ser\-vo\-no le premesse di induzione 
 $([x]U)^{F\alpha(F\delta\beta)}$ e $([x]V)^{F\alpha\delta}$. Da tali premesse,
 poich\`e $\sss_{\alpha,\delta,\beta}$ ha tipo $F(F\alpha(F\delta\beta))
 (F(F\alpha\delta)(F\alpha\beta))$ si ha dapprima che 
 $\sss_{\alpha,\delta,\beta}([x]U)$ \`e termine combinatorio con tipo che 
 risulta $F(F\alpha\delta)(F\alpha\beta)$ e poi che 
 $\sss_{\alpha,\delta,\beta}([x]U)([x]V)$ \`e termine combinatorio di tipo
 $F\alpha\beta$.
\end{itemize}
\end{dimostrazione}

\begin{osservazione}
Si noti che da tale lemma si ha che per un'astrazione multipla (iterata come
al solito) il tipo di $[x^{\alpha_1}, x^{\alpha_2}, \ldots, x^{\alpha_n}]
X^\beta$ \`e $F_n \alpha_1\alpha_2\cdots\alpha_n\beta$ (giustificazione della 
semplificazione $F_n$).
\end{osservazione}

%%--Section Risultati
\section{Risultati}
Il sistema dei termini con tipo pu\`o essere considerato come una restrizione
di un sistema \II, \KK, \SSS~(espanso agli $\ii_\alpha$, $\kk_{\alpha,\beta}$,
$\sss_{\alpha,\beta,\gamma}$) poich\'e certe applicazioni non sono ammesse.
Pertanto \`e abbastanza ovvio che tutti i risultati per \II, \KK, \SSS~valgano
ancora (con dimostrazioni leggermente variate). Non vale invece il teorema
(insiemistico) di inseparabilit\`a ricorsiva. Infatti oltre ai predetti
risultati si ha il seguente teorema.

\begin{teorema}
Ogni termine con tipo ha forma normale.
\end{teorema} 
\begin{dimostrazione}
No (idea: mancanza di universalit\'a).
\end{dimostrazione}

\begin{corollario}
L'uguaglianza fra termini con tipo \`e decidibile.
\end{corollario}
\begin{dimostrazione}
Segue dal terzo corollario di Church-Rosser $II$.
\end{dimostrazione}

\begin{description}
\item[Nota:]Si pu\`o pensare di fare in ``parallelo'' tutte le riduzioni
possibili e di controllare ogni volta se abbia una forma normale che sappiamo
per il teorema esistere.
\end{description}

Ci\`o permette di verificare l'uguaglianza per programmi idealizzati. Risultati
di questo tipo sono stati ottenuti anche per linguaggi di programmazione
non idealizzati (reali). Ci\`o ha dato luogo allo sviluppo della programmazione
strutturale o modulare. 

Con opportuni linguaggi di programmazione si
costringe il programmatore a strutturare il programma in moduli che abbiano 
delle configurazioni d'entrata ed uscita ben precisate e compatibili. Non solo
tali programmi sono a ``terminazione garantita'' ma in fase di compilazione il
compilatore si accorge di errori in un'ampia classe di errori, rilevabili 
cos\`i \emph{automaticamente} prima dell'esecuzione (ovviamente si perde
l'universalit\`a).

%%-- Section assegnamento di tipi a termini senza tipo
\section{Assegnazione di tipi a termini senza tipo}
Considerando che i precedenti termini con tipo non sono tutti quelli senza tipo
(estesi), ci si pu\`o chiedere quando un termine senza tipo corrisponda ad uno
con tipo.
A tal fine (re)introduciamo i termini senza tipo ma con costanti che 
comprendano i tipi fondamentali ed $F$ (oltre ad \II, \KK, \SSS). Pertanto
avremo come termini anche $F\alpha\beta$ (inteso come al solito 
$(F\alpha)\beta$), $F\alpha$, $F$, etc\footnote{Ci\'o serve ad applicazioni
relative alla Logica Matematica e alla Teoria delle Strutture di Dati, che
non vedremo.}. 
Nella definizione di assegnazione introdurremo la notazione:
\begin{itemize}
\item[$(1)$] $\quad\asse \alpha X$
\end{itemize}
per indicare che un tipo $\alpha$ pu\`o essere assegnato ad un termine $X$ 
(infatti nelle applicazioni logiche, che non faremo, ci\'o si legger\`a ``\`e 
dimostrato che $X$ ha la propriet\`a $\alpha$). Si noti che, in accordo con 
l'interpretazione (scorretta ma utile) insiemistica dei tipi, ci\'o pu\`o 
essere letto come $X$ appartiene ad $\alpha$.
\begin{esempio}
 $\asse F\alpha\alpha\ii$ corrisponde insiemisticamente
a $\ii \in \alpha^\alpha$ cio\`e $\ii \colon \alpha \to \alpha$.
\end{esempio}

L'appartenenza ad un insieme \`e (classicamente) la soddisfazione di una 
propriet\`a (insieme $\leftrightarrow$ propriet\`a). Una scrittura come la 
$(1)$ sar\`a detta \emph{enunciato}, mentre $X$ sar\`a detto il \emph{soggetto}
ed $\alpha$ il \emph{predicato}.

\begin{definizione}
Diremo \emph{base} una sequenza di enunciati come la $(1)$.
\end{definizione}

\begin{definizione}
Diremo che un tipo $\alpha$ \`e \emph{assegnato (debolmente)} a un termine 
combinatorio $X$ e scriviamo $\asse \alpha X$ quando ci\`o pu\`o essere dedotto
dai seguenti assiomi:
\begin{itemize}
\item[$(FI)$]$\quad \asse F\alpha\alpha\ii$ per ogni tipi $\alpha$
\item[$(FK)$]$\quad \asse F\alpha(F\beta\alpha)\kk$ per ogni tipo $\alpha$,
  $\beta$
\item[$(FS)$]$\quad \asse F\bigl(F\alpha(F\beta\gamma)\bigr)
  \bigl(F(F\alpha\beta)(F\alpha\gamma)\bigr)$ per ogni tipo $\alpha$, $\beta$, 
  $\gamma$
\end{itemize}
mediante la regola:
\begin{itemize}
\item[$(F)$]$\quad \asse F\alpha\beta U$ e $\asse\alpha V$ implica $\asse 
\beta(UV)$
\end{itemize}

\end{definizione}

\begin{description}
\item[Nota: ]tale scrittura viene di solito abbreviata in:

$
F\alpha\beta U,\ \alpha V\ \asse \beta(UV),
$

e similmente per enunciati di una base.
\end{description}

Con tale definizione possiamo assegnare tipi solo a dei combinatori. Nel caso
generale si considera una base $B$ e si definisce l'assegnazione 
$B \asse \alpha X$ come nella definizione precedente per\`o aggiungendo 
$B$ agli assiomi.

\begin{definizione}
Consideriamo una base $B$ in cui non occorrano variabili, diremo che un termine
$X$ \`e \emph{stratificato relativamente} a $B$ se esiste un tipo $\alpha$ tale
che:
\[
B \asse \alpha X.
\]
Se $B$ \`e vuota diremo che $X$ \`e \emph{stratificato} (non stiamo usando una 
base).
\end{definizione}
Si pu\`o vedere che tale definizione riguarda solo degli $X$ ``costanti'', 
cio\`e privi di variabili.

\begin{definizione}
Nel caso in cui le variabili $x_1, x_2, \ldots, x_n$ occorrano in $X$ diremo
che $X$ \`e stratificabile relativamente a $B$ qualora esistano dei tipi
$\xi_1, \xi_2, \ldots, \xi_n$ ed $\alpha$ tali che:
\[
B, \xi_1x_1, \xi_2x_2, \ldots, \xi_nx_n\ \asse \alpha X
\]
Se $B$ \`e vuota $X$ si dice \emph{stratificabile}.
\end{definizione}

In pratica determinare se un termine \`e stratificato (o straficabile) 
relativamente a $B$  significa cercare di trovarne il tipo e vedere cosa 
succede. Per esempio consideriamo $X \equiv \kk \sss$.

Da $(FS)$ vediamo che possiamo applicare la regola $(F)$ qualora in $(FK)$ si
prenda come $\alpha$ un tipo $\alpha'$ come ivi indicato.
\[
\asse F\underbrace{\Bigl(F\bigl(F\alpha(F\beta\gamma)\bigr)
\bigl(F(F\alpha\beta)(F\alpha\gamma)\bigr)\Bigr)}_{\alpha'} \bigl(F\beta'
\underbrace{(\ldots)}_{\alpha'}\bigr)\kk \qquad \asse \underbrace{(\ldots)}
_{\alpha'}\sss .
\]
Usando la regola $(F)$
\[
\asse F\beta'\bigl(\underbrace{(\ldots)}_{\alpha'}\bigr)(\kk\sss).
\]
Quando il termine \`e ancora un po' pi\`u complicato, per esempio per 
$X \equiv \bb \equiv \sss(\kk\sss)\kk$, tali esplosioni dei sottotipi necessari
per applicare la $(F)$ diventano proibitive (almeno manualmente). Ci\`o avviene
nonostante che il tipo di $\bb$ sia relativamente semplice.
\[
\asse F(F\beta\gamma)\bigl(F(F\alpha\beta)(F\alpha\gamma)\bigr)\bb
\]
D'altra parte a tale tipo di \BB~si arriva molto pi\`u facilmente utilizzando 
l'interpretazione insiemistica ($\bb\leftrightarrow$ composizione invertita)
per la quale:
\[
\bb \colon C^B \to (C^A)^{B^A}.
\]
Pertanto ci si pu\`o chiedere se in un'assegnazione $\asse \alpha X$ non si 
possa utilizzare (questa volta rigorosamente) lo stesso procedimento.
Tale procedimento insiemistico era basato sull'uso dell'astrazione (se un 
argomento di $f$ st\`a in $A$ e $f_a$ st\`a in $B$ allora $f\colon A \to B$).
Vedremo in seguito che ci\`o vale.

\begin{esempio}
Consideriamo \SSS\II\II.
Esso \`e ``cattivo'' nel senso che per ogni $X$ $\sss\ii\ii X = XX$, che 
insiemisticamente non ha un'interpretazione pulita. Verifichiamo se \`e 
stratificato. Se lo fosse ci sarebbe $\alpha$ tale che:
\begin{itemize}
\item[$(2)$]$\quad \asse\alpha(\sss\ii\ii)$;
\end{itemize}
ma la $(2)$ pu\`o provenire solo dalla $(F)$ poich\'e il termine \`e composto.
Allora dovrebbe esistere un $\beta$ tale che:
\[
\asse F\beta\alpha(\sss\ii), \ \asse\beta\ii.
\]
analogamente dovrebbe esserci $\delta$ tale che:
\begin{itemize}
\item[$(3)$]$\quad \asse F\delta(F\beta\alpha)\sss$;
\item[]$\quad \asse \delta\ii$.
\end{itemize}
Ma la $(3)$ deve essere un caso di $(FS)$ perci\`o $\delta \equiv F\alpha'(F
\beta'\gamma')$, $\beta \equiv F\alpha'\beta'$ e $\alpha \equiv F\alpha'\gamma'
$ per qualche $\alpha'$, $\beta'$, $\gamma'$. Tuttavia dell'assegnazione di
$\beta$ ad $\ii$ ($\asse\beta\ii$) sappiamo che $\alpha' \equiv \beta'$.
Da ci\'o e da $\asse \delta\ii$ si ha che (essendo $\alpha' \equiv F\beta'
\gamma'$) $\beta' \equiv F\beta'\gamma'$. Quindi se vi fosse $\alpha$ tale che
$\asse(\sss\ii\ii)$ avremmo un $\beta'$ che coincide con un suo sottotipo. In
definitiva $\sss\ii\ii$ non \`e stratificato.
\end{esempio}

Ci\'o pu\`o far sembrare che sia collegato con la propriet\`a precedente
($\sss\ii\ii$ era combinatore di autoapplicazione). I due esempi successivi
chiariscono il problema.
\begin{esempio}
\KK\KK~\`e stratificato. Infatti per la $(FK)$ abbiamo le due premesse della 
deduzione:
\[
\asse F\alpha(F\beta\alpha)\kk,
\]
\[
\asse F\bigl(F\alpha(F\beta\alpha)\bigr)\Bigl(F\beta'\bigl(F\alpha(F\beta\alpha
)\bigr)\Bigr)\kk.
\]
Tramite la regola $(F)$ otteniamo:
\[
\asse F\beta'\bigl(F\alpha(F\beta\alpha)\bigr)(\kk\kk).
\]
\end{esempio}

\begin{esempio}
$xx$ per ogni variabile $x$ non \`e stratificabile. Infatti se lo fosse 
avremmo $\xi$ ed $\alpha$ tali che $\xi x \asse \alpha(xx)$. Essendo $xx$
composto ci\'o dovrebbe provenire dalla $(F)$ per cui
\[
\xi x \asse F\xi\alpha x,
\]
\[
\xi x \asse \xi x.
\]
Tramite la regola $(F)$ otteniamo:
\[
\xi x \asse \alpha(xx).
\]
Mentre la seconda premessa \`e valida, la prima ci dice che $\xi \equiv F\xi
\alpha$ che \`e impossibile.
\end{esempio}

\begin{description}
\item[Nota: ]il procedimento visto negli esempi pu\`o essere generalizzato
in un algoritmo che consente di terminare con esito positivo o negativo 
(algoritmo ricorsivo) se un termine \`e stratificato (o stratificabile) 
relativamente a delle basi (sufficientemente generali).
\end{description}

Pertanto nell'insieme $\mathcal{T}$ dei termini combinatori, il sottoinsieme
$\mathcal{T'}$ dei termini stratificati (o stratificabili)\footnote{Nel quale
l'uguaglianza debole diventa decidibile.} \`e ricorsivo. Sostanzialmente questa
\`e la ragione per cui nella programmazione strutturata la compilazione 
\`e in grado di trovare i ``bachi'' che con altri linguaggi di programmazione 
non era possibile trovare.

\begin{teorema}
Se $B$ \`e una base senza variabili e se per ogni variabile $x$ e termine $X$
$ B, \alpha x \asse \beta X$,
allora:
\[ B \asse F\alpha\beta([x]X). \]
\end{teorema}
\begin{dimostrazione}
Come per il lemma analogo dei termini con tipo.
\end{dimostrazione}
Possiamo quindi utilizzare l'astrazione per calcolare i tipi (come si faceva
nell'interpretazione insiemistica).

%%--Teorema di Riduzione del soggetto
\begin{teorema}[riduzione del soggetto]
Sia $B$ una base in cui ogni soggetto sia irriducibile rispetto alla 
``$\vartriangleright$'' e in cui esso non cominci con un combinatore.
Se $B \asse \alpha X$ e $X \vartriangleright Y$ allora:
\[ B \asse \alpha Y. \]
\end{teorema}
\begin{dimostrazione}
No.
\end{dimostrazione}

Si noti che nella terminologia logica introdotta ci\'o significa che una 
propriet\`a espressa dal predicato $\alpha$ a proposito del soggetto $X$ viene
sempre ereditata da un soggetto $Y$ cui si pervenga tramite il nostro processo
di calcolo ``$\vartriangleright$'' partendo da $X$.
Di tale teorema esiste anche l'estensione relativa al processo di calcolo
``$\succ$'' (riduzione forte che definisce l'uguaglianza estensionale). Non
vale tuttavia tale conservazione per ``$=$'' (sia debole che forte) poich\'e 
nella premessa del teorema non si pu\`o sostituire la clausola 
$X \vartriangleright Y$ con $Y \vartriangleright X$.

\begin{esempio}
Sia $Y \equiv \sss\kk\sss\ii$. 

Allora si pu\`o vedere che (dalla base vuota)
$\asse F\delta\delta Y$ vale solo per dei $\delta$ composti.
Invece considerando che:
 $$Y \vartriangleright \kk\ii(\sss\ii) \vartriangleright
\ii \equiv X.$$ abbiamo che $\asse F\delta\delta X$ per ogni $\delta$.
\end{esempio}
In modo analogo si pu\`o vedere che per $Y \equiv \sss\ii\ii\ii 
\vartriangleright \ii\ii(\ii\ii) \vartriangleright \ii \equiv X$, mentre $X$ 
\`e ovviamente 
stratificata, $Y$ non pu\`o esserlo --- poich\'e se lo fosse lo sarebbe
$\sss\ii\ii$ che abbiamo visto non esserlo (problema espansione del soggetto).

Consideriamo l'enunciato del teorema d'astrazione nel caso semplificato di base
vuota (che ovviamente soddisfa le condizioni sulle variabili).
\[
\asse \alpha x, \ \asse \beta X \text{ implica } \asse 
F\alpha\beta([x]X).
\]
(non c'e' $\asse \alpha x$ nella conclusione)

In sostanza abbiamo dedotto $F\alpha\beta$ per $[x]X$ senza una base, ma usando
una premessa $\asse\alpha x$ della deduzione che \`e ``scomparsa''. Quindi per
usare tali deduzioni si devono ``cancellare'' tali premesse.

Da tale asserto nasce la seguente definizione.

\begin{definizione}
Diremo che, assegnata una base $B$ in cui non occorrano variabili, possiamo
assegnare \emph{naturalmente} il tipo $\alpha$ ad un termine combinatorio $X$
e scriveremo $B \asset \alpha X$ quando ci\'o pu\`o essere dedotto dagli 
enunciati in $B$ tramite le regole:
\begin{itemize}
\item[$(F)$]$\quad B \asset F\alpha\beta X$ e $B \asset \alpha Y$ implica $B 
  \asset\beta (XY)$.
\item[$(F_i)$]$\quad B, \alpha x \asset \beta X$ implica $B \asset F\alpha\beta
  ([x]X)$ ove in tutte le premesse dei passi induttivi precedenti si cancellano
  le assegnazioni provvisorie.
\end{itemize}
\end{definizione}

Si noti che non vi sono assegnazioni per le costanti riducibili. Nel caso di
base vuota ``si parte da niente'', nel senso che le assegnazioni da cui si
parte devono essere cancellate. Si noti, inoltre, che la clausola di 
cancellazione \`e automaticamente soddisfatta nel caso in cui l'assegnazione
provvisoria da cancellare non esista.

Nel caso della vecchia assegnazione ``$\vdash$'' la deduzione di un tipo 
$\alpha$ per $X$ consisteva in un albero binario (isomorfo a quello di $X$) le
cui foglie corrispondono alle assegnazioni per le costanti riducibili e agli
enunciati in $B$, mentre gli altri nodi corrispondevano ad impieghi della
regola $(F)$.

Ora tale albero per una deduzione ``$\asset~$'', oltre a non avere le foglie 
per le costanti riducibili, ha dei nodi corrispondenti agli impieghi della
$(F_i)$. Tali nodi sono unari perch\`e dalla singola assegnazione (genitore)
ad $X$ si passa all'assegnazione per $[x]X$.
Quindi l'albero ora \`e \emph{al pi\`u} binario. Inoltre occorre ad ogni uso 
della $(F_i)$ fare le cancellazioni. Per fare ci\'o si marca ogni impiego 
delle $(F_i)$ con un'etichetta da ripetere sull'eventuale assegnazione da 
cancellare.

\begin{teorema}
Per ogni base $B$ in cui non occorrono variabili si ha che:
\[
B \asse \alpha X \text{ se e solo se } B \asset \alpha X \text{per ogni 
$\alpha$ ed $X$.}
\]
\end{teorema}

\begin{dimostrazione}
La parte ``se'' deriva dal teorema d'astrazione. Per la parte ``solo se'' basta
mostrare che gli assiomi $(FI)$, $(FK)$ ed $(FS)$ possono essere dedotti col 
nuovo sistema. Saltiamo i primi due, che sono abbastanza banali (per $(FK)$ si
cancella una premessa che non esiste), consideriamo $(FS)$.
Notiamo che $[z]xz(yz) \equiv \sss xy$ (dall'ultima clausola per l'astrazione).
Perci\'o useremo la $(F)$ con assegnazione provvisoria fino a raggiungere 
$xz(yz)$ e poi useremo $(F_i)$.

\[\begin{prooftree}
              \asse F\alpha(F\beta\gamma)x  \phantom{1}^{[3]}   \qquad      
              \asse \alpha z \phantom{1}^{[1]}                  
      \justifies              
                F\beta\gamma (xz)                
      \thickness=0.08em
      \shiftright 2em
      \using
              (F)
\end{prooftree}\]

\[\begin{prooftree}
               \asse F\alpha\beta y \phantom{1}^{[2]}  \qquad      
                \asse \alpha z \phantom{1}^{[1]}          
      \justifies              
                 \asse \beta (yz)               
      \thickness=0.08em
      \shiftright 2em
      \using
              (F)
\end{prooftree}\]

\[\begin{prooftree}
            \asse F\alpha\beta y     \qquad      
            \asse \beta (yz)              
      \justifies              
            \asse \gamma \bigl(xz(yz)\bigr)                    
      \thickness=0.08em
      \shiftright 2em
      \using
              (F)
\end{prooftree}\]

\[\begin{prooftree}
\justifies              
            \asse F\alpha\gamma \qquad         [z]xz(yz)    
      \thickness=0.08em
      \shiftright 2em
      \using
              (Fi)  \quad [1]
\end{prooftree}\]

\[\begin{prooftree}
\justifies              
      \asse F(F\alpha\beta)(F\alpha\gamma) \qquad  \sss x \equiv [y]\sss xy   
      \thickness=0.08em
      \shiftright 2em
      \using
              (Fi)  \quad [2]
\end{prooftree}\]

\[\begin{prooftree}
\justifies              
        \asse F\bigl(F\alpha(F\beta\gamma)\bigr)\bigl(F(F\alpha\beta)(F\alpha\gamma)
\bigr) \qquad \sss                                                    
      \thickness=0.08em
      \shiftright 2em
      \using
              (Fi)  \quad [3]
\end{prooftree}\]
\end{dimostrazione}

%%
%%-- Capitolo - Applicazioni matematiche
%%
\chapter{Applicazioni matematiche}

\section{Logica Matematica} Il calcolo dei combinatori serve allo studio di 
sistemi di Logica Formale. Si introducono nuove costanti per esprimere ad 
esempio i quantificatori, l'implicazione, etc. Le propriet\`a combinatorie 
ci danno propriet\`a di tali sistemi (coerenza, completezza, etc.). 
La deduzione precedente ci fornisce il sistema di ``deduzione naturale di 
Guntzen''.
\section{Algebra (e informatica algebrica)}
\begin{description}
\item[Premessa.]Molte propriet\`a algebriche discendono da propriet\`a 
insiemistiche delle funzioni. In particolare da quelle delle classi di funzioni
iniettive (\injection), suriettive (\surjection), e biettive (\bijection) su
qualche insieme.

\begin{figure}
\[ \includegraphics{img/infteofig.32} \]
\caption{Classi delle funzioni iniettive, biiettive e suriettive.}
\end{figure} 
Tali classi sono ciascuna chiusa rispetto alla composizione funzionale ed hanno
elementi neutri a destra e a sinistra (tramite opportune restrizioni ci\`o d\`a
luogo ad importanti \emph{categorie}).
\end{description}
Per tali applicazioni il sistema \BB-\CC-\KK-\WW~\`e particolarmente utile.
Analizziamo le nostre costanti riducibili.

\subsection{Combinatore \BB}
Dopo applicazione a sinistra ``conserva'' tali classi di 
funzioni qualora lo si intenda nell'interpretazione insiemistica. Ovvero:

\[
f \colon A \injection B \text{ implica } \bb_X f \colon A^X \injection B^X;
\]

ove $\bb_X$ indica l'operatore funzionale insiemistico di ``tipo'':

\[
\bb_X \colon B^A \to (B^X)^{A^X}.
\]
Si pu\`o dimostrare facilmente anche che:
\[
f \colon A \surjection B \text{ implica } \bb_X f \colon A^X \surjection B^X.
\]
Tale $\bb$ \`e di composizione invertita. Vedremo poi quello di composizione
diretta \CC\BB.

\subsection{Combinatore \CC}
Insiemisticamente $\cc \colon (G^B)^{A} \to (G^A)^B$ e si 
vede facilmente che:
\[
\cc \colon (G^B)^{A} \bijection (G^A)^B.
\] 
Inoltre con la solita
notazione per iterazione $\cc^2 = \ii$. Tale \CC~compare in propriet\`a
algebriche dette universalit\`a dei prodotti diretti (che oltre 
all'esponenziazione insiemistica coinvolgono anche il prodotto insiemistico).
Componendo con \BB~si ha l'operatore di composizione diretta.
\[
\cc\bb_X \colon A^X \to (B^X)^{B^A}.
\]
Tale operatore \`e ben noto in algebra (monoidi, sottomonoidi, categorie e
composizione funzionale) ma \`e importante anche al di fuori di tale 
interpretazione.

\begin{esempio}
Rappresentazione matriciale.

Sia $E \subseteq A^A$ l'insieme degli indomorfismi di uno spazio vettoriale su 
$A$. Una funzione  $b \colon X \to A$ \`e una base dello spazio se e soltanto 
se
\[
\cc\bb_Xb \colon E\, \bijection A^X
\]
cio\`e se e soltanto se per ogni $h \in E, \ \cc\bb_Xbh = \bb_Xhb = h \circ b 
\colon X \to A$ rappresenta biunivocamente $h$.
\end{esempio}
In altre parole \CC\BB~ci permette di rappresentare endomorfismi tramite una 
\emph{base} $b$ (sistema di riferimento) con delle \emph{matrici} in $A^X$ ove
per ogni $m$ tale che:
\[m \colon X \to A\]
 ($m = h \circ b$) $m_x$ \`e la
colonna $x$-esima della matrice.
La differenza con la presentazione geometrica di tali nozioni \`e che abbiamo 
usato dei combinatori. Siamo andati a vedere dal punto di vista operativo, o
``intensionale'', come vanno le cose. Ci\`o fa si che tale costruzione funzioni
anche al di fuori degli spazi vettoriali.
Vi sono infatti algebre che non sono spazi vettoriali (monoidi, gruppi, 
reticoli, etc.) considerate in Matematica e molte altre considerate in
Informatica Algebrica.
Ci\`o fa si che l'implementazione  dell'Algebra Lineare  (nel senso di 
Geometria Analitica) possa ``transitare'' su tutte le altre algebre. Questo
realizza la congettura di A.~N.~Whitehead su di un metodo d'interpretazione 
unica per tutte le algebre derivate dall'Algebra Lineare.
Mediante tale impostazione sono stati riscritti problemi che l'Algebra aveva 
posto ma non risolto.

\subsection{Combinatore \KK}
Nell'interpretazione insiemistica \`e iniettivo:
\[
\kk \colon A\, \injection A^B;
\]
cio\`e: $\kk_a = \kk_{a'} \text{ implica } a = a'$.

\subsection{Combinatore \WW}
Nell'interpretazione insiemistica \`e suriettivo:
\[
\ww \colon (B^A)^A\, \surjection B^A;
\]
cio\`e per ogni $g \colon A \to B, \text{ esiste }
f \colon A \to B^A \text{ tale che } g = \ww f$.

%%
%%-- Capitolo - Applicazioni informatiche
%%
\chapter{Applicazioni informatiche}
In questo capitolo vedremo le varie applicazioni dei termini combinatori in 
ambito informatico. Salteremo, avendola gi\`a considerata nel capitolo dei
termini con tipo, la \emph{programmazione strutturata}, mentre vedremo in breve
il linguaggio di programmazione LISP.

\section{Introduzione al LISP}
\begin{description}
\item[Storia:]
Nel 1960 McCarty (allievo di Church) propose un linguaggio di programmazione
basato sul \lbc. Nel 1962 abbiamo la prima implementazione (funzionante).
Fino ad allora la programmazione veniva fatta esclusivamente in Assembler.
Nel 1964 furono proposti dapprima l'Algol, poi il Fortran; linguaggi orientati 
al calcolo numerico. Il Fortran era sostanzialmente solo un Assembler 
potenziato con un traduttore di formule e divenne operativo nel 1968. L'Algol
era pi\`u elevato, ma fu ``ucciso'' commercialmente a causa della ``non 
linearit\`a''\footnote{Per mercato non lineare si intende un mercato che non 
segue le regole ``canoniche'', in cui un prodotto migliore o conveniente ha 
pi\`u successo di uno peggiore o pi\`u costoso, ma al contrario segue regole
dettate dalla facilit\`a di apprendimento o altri fattori.} del mercato. 
Tuttavia l'Algol gener\`o altri linguaggi di tipo imperativo.
\end{description}
Il LISP \`e orientato all'elaborazione simbolica (pur essendo universale), i
campi d'applicazione che vedremo danno un'idea di tale tipo di elaborazione;
i problemi relativi a questi furono dapprima risolti con il LISP, ma in seguito
questo fu sostituito da strumenti specifici nella maggior parte dei casi.
Sostanzialmente il LISP \`e un liguaggio per problemi innovativi.

\subsection{Campi d'applicazione LISP}
\begin{itemize}
\item[-]Verifica automatica di dimostrazioni (formalizzate).
\item[-]Inferenza induttiva su sequenze. Ad esempio:
Qual'\`e il numero successivo a $5\ 2\ 7\ 9\ 16\ $ ?

\item[-]Calcoli in fisica delle particelle (gruppi).
\item[-]Compilatori.
\item[-]Compile-compiler. Generano almeno in parte i compilatori partendo da
specifiche del linguaggio di programmazione.
\item[-]Pattern matching. Trova somiglianze all'interno di due sequenze di 
simboli.
\item[-]Circuiti elettrici.
\item[-]Programmazione per calcolo simbolico integro-differenziale. Fino alla
fine degli anni '60 si faceva solo calcolo numerico puro, ora esistono
pacchetti specifici ad esempio Maple.
\item[-]Giochi. Sia deterministici (dama, scacchi, etc) che non (carte).
\item[-]Generazione di simulatori.
\item[-]Generazione di codici di compressione o di protezione.
\item[-]Linguistica.
\item[-]Information-retreval, Data-mining.
\item[-]Generazione di equazioni diofantine\footnote{A variabili intere.} 
(teorema di superincompletezza di Chaitin).
\end{itemize}

\subsection{Idee di base}
\begin{itemize}
\item[-]\lbt~ $\Longleftrightarrow$ S-espressione $+$ $\lambda$-astrazione.
\item[-]Omogeneit\`a tra dati e procedure. Conseguenze: unicit\`a della
struttura di memoria, generabilit\`a di programmi o procedure in esecuzione.
\end{itemize}

\subsection{Differenza dal \lbc}
\begin{itemize}
\item[-]Notazione.
\item[-]Aumento delle primitive (utilizzo registri aritmetici, primitive di
Ingresso/Uscita, primitive complesse tipo funzioni trigonometriche, etc.).
\item[-]Meta-primitive (``define'' per richiamo procedure).
\item[-]Restrizioni su $\lambda$-conversioni.
\item[-]Istruzioni per governo sequenzialit\`a di esecuzione.
\end{itemize}

\subsection{Organizzazione memoria LISP}
Almeno idealmente la memoria LISP (e la relativa notazione) \`e basata su 
alberi binari. Essi possono essere implementati mediante \emph{celle} di 
memoria liberamente sparse nella memoria ad accesso diretto. Ogni cella \`e 
costituita almeno idealmente da due campi contenenti di regola indirizzi di 
altre celle (o di tavole di ``atomi'').

\begin{esempio}\label{AIT10E1}
Indirizi hardware.

\begin{tabular}{cc}
indirizzo & contenuto \\
\hline \\
1 & \begin{tabular}{|r|l|} \hline xxx & xxx \\ \hline \end{tabular} \\
2 & \begin{tabular}{|r|l|} \hline xxx & xxx \\ \hline \end{tabular}  \\
\vdots & \vdots \\
325 & \begin{tabular}{|r|l|} \hline 360 & 923 \\ \hline \end{tabular} \\
\vdots & \vdots \\
360 & \begin{tabular}{|r|l|} \hline 325 & 411 \\ \hline \end{tabular} \\
\vdots & \vdots \\
411 & \begin{tabular}{|r|l|} \hline 923 & 924 \\ \hline \end{tabular} \\
\vdots & \vdots \\
923 & \begin{tabular}{|r|l|} \hline At1 & Nil \\ \hline \end{tabular} \\
924 & \begin{tabular}{|r|l|} \hline Nil & Nil \\ \hline \end{tabular} \\
\end{tabular}
\end{esempio}

\begin{figure}
\[\includegraphics{img/infteofig.33} \]
\caption{Rappresentazione grafica della memoria nell'esempio \ref{AIT10E1}}
\end{figure}

Possiamo pensare agli atomi come ai campi di una tabella che inizi in $900$,
la foglia $360$ invece non \`e un atomo e ci\'o pu\`o servire a (ri)-ciclare
sulla cella $360$. Per gli atomi il contenuto non \`e un indirizzo, anzi si
pu\`o allora semplificare l'interpretazione dicendo che la cella che
indirizza un atomo contiene nel campo corrispondente l'atomo stesso. Cio\`e 
nell'esempio \ref{AIT10E1} si deve vedere in $411$ la cella 
\begin{tabular}{|r|l|} \hline At1 & Nil \\ \hline \end{tabular} trattandosi
di una struttura a \emph{puntatori} gode dei vantaggi di tali strutture: facile
aggiornabilit\`a, facile esecuzione di operazioni strutturali.
\begin{notazione}
Per indicare tali strutture si usano le \emph{S-espressioni}. Si ottengono
abbinando col simbolo ``.'' (punto) altre S-espressioni o atomi (cos\`i
come nella definizione dei termini combinatori ma con il punto che sostituisce
l'accostamento).
\begin{esempio}\label{AIT10E2}
Nel caso di:
% Inserire figura
\[\includegraphics{img/infteofig.34} \]
avremo $At1.(At2.Nil)$.
\end{esempio}
Col crescere dell'albero l'S-espressione si complica e aumentano le parentesi, 
si introduce allora una notazione secondaria: quella di lista.
Le liste sono sostanzialmente parole (non alberi), cio\`e  sono generate da
una parola vuota (corrispondente all'atomo Nil) e dall'abbinamento di una 
parola con una lettera dell'alfabeto di atomi diversi da Nil.
\end{notazione}

Nell'esempio precedente (\ref{AIT10E1}) l'alfabeto \`e 
fissato per costruire delle parole, tuttavia in generale, si usa come 
``alfabeto'' anche l'insieme delle liste. Per esempio si ha anche la ``parola''
\[At1.(\underline{(At2.\underline{(At1.Nil)})}.Nil)\] ove le ``lettere'' 
 sottolineate non sono atomi. Pertanto le liste non sono propriamente 
parole, ma conviene considerarle tali per ricordarsi che necessitano dell'atomo
$Nil$, che funziona come termine di lista e corrisponde al generatore $\empty$
delle parole.

Per le liste si usa omettere le parentesi associate a destra (non a sinistra
come i termini combinatori), rappresentandole come le parole (con 
l'accostamento sostituito da ``$\not b$'' o da ``.'') e quindi omettendo $Nil$.
Nel primo esempio scriveremo $At1\ At2$ e nel secondo esempio scriveremo
$At1\ (At2\ At1)$. Si noti che questa \`e una lista diversa da quella indicata 
con $At1\ At2\ At1$ che corrisponderebbe all'S-espressione 
$At1.(At2.(At1.Nil))$.

\section{Rappresentazione grafica di S-termini}[Rap. grafica di S-term.]
Per rappresentare graficamente gli S-termini si usano alberi binari con nodi
costituiti da una doppia cella: \begin{tabular}{|r|l|} \hline ind-1 & ind-2 \\ 
\hline \end{tabular}. Ove per\`o $ind-1$ e $ind-2$ sono graficamente sostituiti
da un nome di atomo nel caso che siano indizizzi di un atomo diverso da $Nil$,
da una barra nel caso dell'atomo $Nil$ o da una freccia che punta ad un'altro
nodo altrimenti.

\begin{esempio}Ecco la rappresentazione grafica dell'esempio \ref{AIT10E2}:
\[\includegraphics{img/infteofig.35}\]
\end{esempio}

\begin{esempio}Di seguito vediamo la reppresentazione grafica della parola 
vista sopra: $At1.((At2.(At1.Nil)).Nil)$
\[\includegraphics{img/infteofig.36}\]
\end{esempio}
Tale ultima rappresentazione \`e evidentemente differente da quella per
l'S-espres\-sione $At1\ At2\ At1$ per la quale l'albero binario \`e composto
da una catena, ma con frecce che ora puntano sempre in basso a destra.

\begin{osservazione} Alcune osservazioni rilevanti:
\begin{enumerate}
\item[-]Per come sono definite le liste, possono sempre essere indicate con 
delle  S-espressioni (ne formano un sottoinsieme). Invece non tutto le 
S-espressioni possono essere scritte come liste, per esempio $At1.At2$ 
(senza $Nil$) in una lista possono occorrere come lettere.
\item[-]L'atomo $Nil$ oltre che atomo \`e anche lista.
\item[-]Sia con le S-espressioni che con le liste non possono essere indicati
degli ``alberi'' in memoria che ``ricorrano su se stessi'', come nell'esempio
iniziale di organizzazione di memoria. Questi ``alberi'' non sono ammessi nel
LISP, ma non possono essere generati con le sole notazioni predette (ne con
le istruzioni elementari per le S-espressioni). Per gestire tali alberi
impropri esistono delle istruzioni speciali che non vedremo.
\item[-]Tra gli atomi si considerano vari casi: 
\begin{itemize}
\item Sequenze alfanumeriche scelte dal programmatore (per indicare 
variabili, nomi di procedure, etc.). Queste sono organizzate in una tabella.
\item Costanti numeriche per vari numeri e diverse rappresentazioni.
\item Costanti logiche, per il vero (T) e il falso (talvolta si usa il 
$Nil$).
\item Parole chiave del linguaggio (in particolare LAMBDA, identificatori 
di funzioni elementari o primitive).
\end{itemize}
\end{enumerate}
\end{osservazione}

\section{Uso delle S-espressioni}
Le S-espressioni hanno due principali utilizzi: rappresentano strutture 
complesse di dati e l'organizzazione del programma sorgente sorgente.

\begin{itemize}
\item[a)]Rappresentazione di strutture complesse di dati. Oltre agli alberi 
binari e alle parole si possono rappresentare alberi generali (per esempio gli 
alberi delle strategie del gioco degli scacchi che hanno nodi i cui discendenti
determinati dalle mosse possibili dell'avversario variano nel progredire del 
gioco). Per fare ci\`o \`e possibile rappresentare i discendenti di un nodo con
delle liste.

% Inserire figura
\[\includegraphics{img/infteofig.37}\]
Ove $n1$ \`e la lista $n2,n3,n4$; $n2$ \`e la lista $n9$; $n3$ \`e la lista 
$n6,n7$ ed $n4$ \`e la lista $()$ ovvero $Nil$.
Si possono rappresentare con le liste (o con S-espressioni) matrici o tabelle 
a pi\`u indici. Per esempio una tabella a $2$ indici $3 \times 2$ pu\`o essere
rappresentata da:


\[\includegraphics{img/infteofig.38}\]
Si possono rappresentare facilmente le strutture dati corrispondenti a 
dispositivi speciali di memorie (pile, code, etc.).
\item[b)]Rappresentazione dell'organizzazione del programma sorgente.
\begin{esempio}
La lista LAMBDA (lista1) (lista2) rappresenta la procedura corrispondente al
\lbt: $\lambda x_1 \cdots x_n.T$, ove (lista1) $= (x_1 \cdots x_n)$ e (lista2)
$= T$. Cio\`e (lista1) individua i parametri di chiamata e (lista2) il corpo 
della procedura.
\end{esempio}
Tali procedure sono sostanzialmente delle macro che vengono eseguite quando si 
applicano dei dati ``corrispondenti a parametri di chiamata'', vedremo poi in 
cosa consiste l'esecuzione.
\begin{esempio}
Mentre la lista precedente fornisce una macro parametrizzata (che conunque deve
essere riscritta ogni volta che la si deve usare) la lista DEFINE (lista1) 
permette di associare a uno o pi\`u di tali macro dei nomi richiamabili 
(mediante lista1).
\end{esempio}
In generale tutto un programma sorgente scritto dal programmatore (o 
autogeneratosi) \`e una lista (LISP = LISt Processor). 
\end{itemize}

\section{Funzioni elementari}
Il primo gruppo di funzioni elementari \`e quello delle istruzioni per gestire
S-espressioni.
\begin{itemize}
\item[-]CONS($A\ B$) crea l'albero che ha come due discendenti delle 
S-espressioni $A$ e $B$. Cio\`e aggiunge in memoria una cella di tipo:

\[\includegraphics{img/infteofig.39}\]
Ove $a$ e $b$ sono gli indirizzi delle (persistenti) radici di $A$ e $B$.
\item[-]CAR e CDR fanno l'opposto: CAR($A$) trova l'indirizzo del sottoalbero
di sinistra di $A$, mentre CDR($A$) trova quello di destra. Queste due 
istruzioni di etrazione ad un passo generano poi quelle a pi\`u passi, per 
esempio, CAADR($A$) trova:
% Inserire figura
\[\includegraphics{img/infteofig.40}\]
\end{itemize}

\begin{osservazione}
Le istruzioni elementari composte come le precedenti sono sostanzialmente 
parole binarie (Caadr corrisponde a $0010$). Possiamo dire che le parole 
binarie non sono altro che gli indirizzi verso i nodi di un albero binario 
(abbastanza grande). La stessa cosa succede negli alberi unari (catene finite) 
che possiamo identificare coi numeri naturali. Per precisare con un esempio, 
tali alberi unari sono gli indirizzi per una (grande) memoria ad accesso 
diretto tradizionale.
Per passare da un indirizzo ad un altro (precedente), cio\`e per passare dal 
vertice di un'albero unario ad un nodo precedente, usiamo numeri relativi 
negativi. Quindi tali istruzioni composte sembrano i corrispondenti dei numeri 
relativi tradizionali rispetto ad un ambiente binario (anzich\'e unario).
\end{osservazione}

Nasce allora il problema di verificare tale intuizione e pi\`u in generale di 
vedere se esistano numeri ``duo-ari'' relativi in grado di trasformare alberi 
binari e possibilmente con un embrione di aritmentica\footnote{Per quelli 
negativi, cio\`e parole binarie, conosciamo la catenazione che corrisponde in 
ambito unario alla somma. Ma ci sar\`a una somma per tutti gli ipotetici 
numeri ``duari'' relativi?}.

\subsection{Altre funzioni elementari}
\begin{itemize}
\item[-]Funzioni elementari aritmetiche: sommatoria, produttoria, esponenziali,
etc.
\item[-]Funzioni elementari logiche: connettivi logici, predicati, etc.
\end{itemize}

\section{Espressioni condizionali}
LISP adotta, come nei linguaggi imperativi, espressioni condizionali, ma di
tipo multiplo.
\begin{esempio}
COND($p1, \ e1$)($p2,\ e2$) significa che se il predicato $p1$ \`e vero si 
calcola $e1$, se $p2$ \`e vero si calcola $e2$.
\end{esempio}

\section{Istruzioni di controllo esecuzione}
Normalmente il LISP esegue sequenzialmente le istruzioni (come negli altri 
linguaggi). Tuttavia vi sono situazioni in cui ``non sa cosa fare''. Per 
esempio se una procedura ha lo scopo di genrare altre procedure, quando viene 
eseguita la prima non si sa che fare della seconda. Si potrebbe, infatti, 
volerla eseguire oppure si potrebbe usarla, senza eseguirla subito, per 
costruire una terza procedura.

Allora il LISP ha la convenzione che normalmente la seconda procedura non venga
eseguita. Per comandargli invece l'esecuzione occorre inserire un'istruzione 
apposita (EVAL). 

In molti casi sussiste il problema inverso: si giunge ad una procedura (per 
esempio seguendo l'ordine sequenziale di esecuzione) che tuttavia \emph{non} si
deve eseguire al momento. A tale scopo esiste l'istruzione QUOTE che sospende 
l'esecuzione nel senso che quanto ``quotato'' resta invariato.

\begin{esempio}
La procedura (LAMBDA($X$)(CAR $X$))($A.B$) diventa CAR ($A.B$) $= A$. Invece la
procedura (LAMBDA($X$)(QUOTE $X$))($A.B$) diventa $X$ pich\'e l'esecuzione ora 
consiste nel ``non eseguire''.
\end{esempio}

Esiste anche una funzione TRACE che (almeno nella prima versione LISP) ha solo 
scopi diagnostici cio\`e serve a stampare l'andamento dell'esecuzione per 
``sbacare'' il programma.

\section{Ricorsione}
Alcuni linguaggi di programmazione non consentono di programmare facilmente la 
ricorsione (quando si diffuse il LISP tutti gli altri non la consentivano). Se 
si vuole definire il fattoriale in un linguaggio non predisposto per la 
ricorsione, occorre organizzare la memoria affinch\`e ci\`o sia possibile: 
definire una pila con la quale realizzare i passi induttivi, etc. Invece con il
LISP si pu\`o scrivere: 
\begin{center}
Factorial ($n$) $=$ (COND((ZEROP $n$)$1$))
\end{center} 
ZEROP $n$ controlla che $n$ sia $0$, se si ``restituisce'' $1$, altrimenti:
\begin{center}
Factorial ($n$) $=$ (TIMES n(Factorial(SUB $1 \ n$)))
\end{center} 
TIMES $n$ $m$ moltiplica $n$ per $m$, SUB $n$ $m$ sottrae $n$ ad $m$.

Tali istruzioni possono quindi essere messe in una macro LAMBDA e con una 
definizione rese riutilizzabili.

Il motivo per cui il LISP non necessita di programmare la memoria \`e perch\`e
esso valuta le funzioni (sia quelle predefinite che quelle create dal 
programmatore) semplicemente legando gli indirizzi dei parametri formali a 
quelli degli argomenti. Ci\`o \`e reso possibile dall'organizzazione della 
memoria virtuale LISP che \`e basata su alberi fatti con puntatori (indirizzi).
\end{document}
