\begin{comment}

\begin{codice}
\begin{verbatim}

\end{verbatim}
\end{codice}

\end{comment}

\chapter{Esercitazioni di Laboratorio Computazionale Numerico.}

\section{Esercitazione I.}

\begin{enumerate}
\item Supponendo che le variabili $a$, $b$, $c$, $d$, $e$, $f$, $g$ siano 
scalari,  scrivere istruzioni di assegnazione in Matlab per calcolare il 
valore delle seguenti espressioni:
\[
x = 1 + \frac{a}{b} + \frac{c}{f^2} \qquad s = \frac{b-a}{d-c} \qquad
z = \left(1-\frac{1}{e^5}\right)^{-1}
\]
\[
r = \frac{1}{\frac{1}{a}+\frac{1}{b}+\frac{1}{c}+\frac{1}{d}} \qquad
y = ab \cdot \frac{1}{c} \cdot \frac{f^2}{2} \qquad 
t = 7\left(g^{\frac{1}{3}}\right) + 4g^{0.58}
\]
con $a = 1.12$, $b = 2.34$, $c = 0.72$, $d = 0.81$, $e = 3$, $f = 19.83$, 
$g = 20$. Visualizzare i risultati in format short e format long.

\begin{svol}
\begin{codice}
\begin{verbatim}
>> a = 1.12;
>> b = 2.34;
>> c = 0.72;
>> d = 0.81;
>> e = 3;
>> f = 19.83;
>> g = 20;
>> 
>> x = 1 + (a/b) + (c/f^2)

x =

    1.4805

>> s = (b-a)/(d-c);
>> z = (1-1/e^5)^-1;
>> r = 1/(1/a + 1/b + 1/c + 1/d);
>> y = a*b*(1/c)*((f^2)/2);
>> t = 7*(g^(1/3))+4*g^(0.58);
>> 
>> x, s, z, r, y, t

x =

    1.4805


s =

   13.5556


z =

    1.0041


r =

    0.2536


y =

  715.6766


t =

   41.7340

>> format long
>> 
>> x, s, z, r, y, t

x =

   1.480463473251643


s =

  13.555555555555541


z =

   1.004132231404959


r =

   0.253571274994625


y =

     7.156765979999999e+02


t =

  41.733956653314806

>> 
\end{verbatim}
\end{codice}
\end{svol}

\item Dopo aver cancellato le variabili del workspace, individuare, 
descrivere e correggere gli errori nelle seguenti istruzioni di assegnazione:
\begin{itemize}
\item[-]\verb a=2y+(((3+1)9) ,
\item[-]\verb b==2*sin[3] ,
\item[-]digitare \verb c=e^0.5 \ per calcolare  $e^{0.5}$ con  $e$
numero di Nepero,
\item[-]per il calcolo di $log(4 −\frac{8}{4-2})$ digitare
 \verb d=log(4-8/4*2) .
\end{itemize}

\begin{svol}
\begin{codice}
\begin{verbatim}
>> clear x s z r y t
>> 
>> a = 2y+(((3+1)9)
??? a = 2y+(((3+1)9)
         |
Error: Unexpected MATLAB expression.
\end{verbatim}
\end{codice} 
In questo comando abbiamo due tipi di errori: innanzitutto le moltiplicazioni
(ove \verb 2y \ si intende ovviamente $2 \cdot y$) si effettuano con 
l'operatore\verb * , inoltre le parentesi non sono bilanciate.
\begin{codice}
\begin{verbatim}
>> b==2*sin[3]
??? b==2*sin[3]
            |
Error: Unbalanced or unexpected
parenthesis or bracket.
 
\end{verbatim}
\end{codice} 
La funzione $\sin$, come tutte le funzioni in Matlab, utilizza le parentesi
rotonde (le quadre sono utilizzate per altri scopi). 
\begin{codice}
\begin{verbatim}
>> c = e^0.5

c =

   1.732050807568877

\end{verbatim}
\end{codice}
\verb e \ in questo caso è uno scalare definito da utente, il numero di
Nepero in Matlab è calcolato dalla dunzione \verb exp() \ con argomento $1$:
\verb exp(1) .
\begin{codice}
\begin{verbatim}
>> d = log(4-8/4*2)

d =

  -Inf

>> 
\end{verbatim}
\end{codice}
Gli operatori \verb / \ e \verb * \ hanno la stessa precedenza, quindi la
sopracitata istruzione calcola $\log(4 - \frac{8}{4}\cdot 2)$ eseguendo in 
ordine (da sinistra verso destra) le operazioni. L'istruzione corretta
è quindi \verb d=log(4-8/(4*2)) . 
\end{svol}

\item Mediante una sequenza di istruzioni di assegnazione in Matlab:
\begin{itemize}
\item[--]
Calcolare il raggio di una sfera che ha un volume del $30\%$ più grande di 
una sfera di raggio $5$ cm.
\item[--]
Considerando le seguenti approssimazioni polinomiali della funzione $e^x$ :
\[
e^x \simeq p_1(x) := 1 + x, \quad
e^x \simeq p_2(x) := 1 + x +\frac{x^2}{2}\]
si calcolino l’errore assoluto
\[e_a (x) = |e^x − p_i (x)| \quad i = 1, 2\]
e l’errore relativo
\[e_r(x) = \frac{|e^x − p_i(x)|}{e^x} \quad i = 1, 2\]
in $x = 0,1$.
\item[--] Calcolare le radici delle equazioni:
\[2t2 − 4t − 1 = 0 \quad
x4 + 2x2 − 3 = 0 \quad
x3 = 2197.\]
\end{itemize}

\begin{svol}
\begin{itemize}
\item[--]
\begin{codice}
\begin{verbatim}
>> v1 = (4/3)*pi*(5^3);
>> v2 = v1*(130/100);
>> r2 = ((3/4)*(1/pi)*v2)^(1/3)

r2 =

   5.456964415305529

>> 
\end{verbatim}
\end{codice}
\item[--]
\begin{codice}
\begin{verbatim}
>> p10 = 1 + 0;
>> p11 = 1 + 1;
>> p20 = 1 + 0 + (0^2)/2;
>> p21 = 1 + 1 + (1^2)/2;
>> 
>> e_a10 = abs(exp(0) - p10)

e_a10 =

     0

>> e_a11 = abs(exp(1) - p11)

e_a11 =

   0.718281828459046

>> e_a20 = abs(exp(0) - p20)

e_a20 =

     0

>> e_a21 = abs(exp(1) - p21)

e_a21 =

   0.218281828459046

>> e_r10 = abs(exp(0) - p10)/(exp(0))

e_r10 =

     0

>> e_r11 = abs(exp(1) - p11)/(exp(1))

e_r11 =

   0.264241117657115

>> e_r20 = abs(exp(0) - p20)/(exp(0))

e_r20 =

     0

>> e_r21 = abs(exp(1) - p21)/(exp(1))

e_r21 =

   0.080301397071394

>> 
\end{verbatim}
\end{codice}
\item[--] Utilizzando $x_{1,2}=\frac{-b \pm \sqrt{b^2-4ac}}{2a}$ e alcune
operazioni algebriche:

\begin{codice}
\begin{verbatim}
>> t1 = (4 + sqrt((-4)^2-4*2*(-1)))/(2*2);
>> t2 = (4 - sqrt((-4)^2-4*2*(-1)))/(2*2);
>> t = [t1 t2]

t =

   2.224744871391589  -0.224744871391589

>> 
>> y1 = (-2 + sqrt((2)^2-4*2*(-3)))/(2*2);
>> y2 = (-2 - sqrt((2)^2-4*2*(-3)))/(2*2);
>> x1 =  sqrt(y1);
>> x2 = -x1;
>> x = [x2 x1]

x =

  -0.907124939317785   0.907124939317785

>> 
>> x = (2197)^(1/3)

x =

  12.999999999999998

>> 
\end{verbatim}
\end{codice}
\end{itemize}
\end{svol}

\item
Generare il vettore riga e il vettore colonna $y$ di elementi equidistanti 
$1, 2, ..., 10$ e $10, 9, ..., 1$ rispettivamente e farne il prodotto scalare.
Generare inoltre il vettore colonna $z$ costituito dai valori della funzione 
seno in $11$ elementi equidistanti nell’intervallo $[0, 1]$.

\end{enumerate}
