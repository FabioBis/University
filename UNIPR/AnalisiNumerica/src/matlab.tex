\hyphenation{
Matlab
}

\begin{comment}

\begin{codice}
\begin{verbatim}

\end{verbatim}
\end{codice}

\end{comment}

\chapter{Introduzione a Matlab.}
MATLAB (abbreviazione di Matrix Laboratory) è un ambiente per il calcolo 
numerico e l'analisi statistica che comprende anche l'omonimo linguaggio di 
programmazione creato dalla The MathWorks. MATLAB consente di manipolare 
matrici, visualizzare funzioni e dati, implementare algoritmi, creare 
interfacce utente, e interfacciarsi con altri programmi.

\begin{notabene}Il corso e \emph{sopratutto l'esame} di Laboratorio di 
Calcolo Numerico vertirà su esercizi da svolgersi in ambiente Matlab, per
esercitarsi è possibile usufruire dei computer in laboratorio. Per coloro che
volessero (o debbano) esercitarsi a casa, poiché Matlab è un software
proprietario, l'unico modo (legale) per ottenere il programma è l'acquisto.
Tuttavia esiste un'alternativa ``free'', Octave, che permette di fare gran parte
delle operazioni di Matlab con gli stessi comandi. D'ora in poi, nel seguente
testo, per specificare differenze o particolarità riguardanti Octave 
utilizzeremo un ``ambiente'' apposito come il seguente.
\end{notabene}

\begin{octave}
GNU Octave è un'applicazione software per l'analisi numerica parzialmente
 compatibile con MATLAB. Octave è rilasciato sotto licenza GPL, e quindi può 
essere liberamente copiato e usato. Il programma gira sotto sistemi Unix e 
Linux, oltre che su Windows e MAC OS X.

Ha un insieme di funzionalità fornite per il calcolo matriciale come rango e 
determinante o specialistiche come decomposizione ai valori singolari (SVD), 
fattorizzazione LU; sebbene le consenta la soluzione numerica di sistemi 
lineari non svolge calcolo simbolico o altre attività tipiche di un sistema 
di algebra computazionale.
\end{octave}

\section{Introduzione.}
Questo testo è di fatto una raccolta di appunti del corso di Laboratorio di
Calcolo Numerico del dipartimento di Matematica dell'Università di Parma, 
non ha lo scopo di sostituirne le lezioni o i libri di testo
consigliati. Pertanto sconsiglio di utilizzarlo come unica fonte di studio
o di esercitazione pratica.

Inoltre non verranno definite tutte le funzioni o risolti tutti gli esercizi
proposti durante il corso ed alcune altre cose saranno date per scontate
come ad esempio la descrizione o definizione di:
\begin{enumerate}
\item[-]Command Window,
\item[-]Command History,
\item[-]Workspace,
\item[-]alcune funzionio comandi ben descritti sul libro di testo.
\end{enumerate}

\section{Help.}
Il primo comando utile di Matlab è \verb1help1, serve, dato
un comando o una funzione, per ottenerne le istruzioni d'uso:
\begin{codice}
\begin{verbatim}
>> help nomecomando
>> help nomefunzione
\end{verbatim}
\end{codice}

\section{Variabili ed ambiente di lavoro.}
In Matlab tutte le variabili sono array, nello specifico una variabile singola
è un array di dimensione $1$. Nell'ambiente di lavoro (workspace) le variabili
vengono memorizzate e vi risiedono fino alla fine della sessione di lavoro o
fino a quando non viene utilizzato il comando \verb1clear1 che consente di
cancellarle, non vengono cancellati lo schermo e la history dei comandi.
Per cancellare il contenuto dello schermo si utilizza il comando \verb1clc1.
\begin{codice}
\begin{verbatim}
>> clear
\end{verbatim}
\end{codice}

Per definire (dichiarazione e inizializzazione) una variabile è sufficiente
scrivere la seguente istruzione:
\begin{codice}
\begin{verbatim}
>> x = 1

x =

     1
\end{verbatim}
\end{codice}
In questo modo verrà salvata nel Workspace la variabile di nome $x$, valore 
$1$, size $1\times 1$ e valore minimo e massimo dell'array. Di default gli 
attributi visibili delle variabili del Workspace dovrebbero essere questi, è
comunque possibile cambiarli dal menù a tendina ``view/Choose Columns''.

E' possibile evitare di visualizzare a schermo l'assegnamento della variabile
semplicemente utilizzando il carattere ``;'' a fine istruzione. Questo è
molto utile per array o matrici di grandi dimensioni.

\begin{codice}
\begin{verbatim}
>> v = [ 1 2 3 4 5 6 7 8 9 ]

v =

     1     2     3     4     5     6     7     8     9

\end{verbatim}
\end{codice}

\begin{codice}
\begin{verbatim}
>> v = [ 1 2 3 4 5 6 7 8 9 ];
>> 
\end{verbatim}
\end{codice}

Esiste una variabile speciale in Matlab chiamata \verb1ans1, questa è generata
automaticamente se non viene esplicitamente dato un nome ad una variabile,
tipicamente quando non viene assegnato il risultato di un operazione.
Proviamo a moltiplicare il nostro vettore $v$ per la variabile $x$ senza
assegnamento:
\begin{codice}
\begin{verbatim}
>> x*v

ans =

     1     2     3     4     5     6     7     8     9
\end{verbatim}
\end{codice}
Utilizzare la variabile \verb1ans1 non è un errore, occorre però fare 
attenzione poiché questa può essere sovrascritta durante il lavoro.

Nell'ambiente di lavoro è inoltre possibile cambiare il formato di 
rappresentazione dei numeri, di default l'ambiente dovrebbe essere inpostato
sul formato \verb1short1 ma è possibile cambiarlo con il comando \verb1format1.
\begin{codice}
\begin{verbatim}
>> x = 1/2

x =

    0.5000

>> format long
>> x = 1/2

x =

   0.500000000000000

>> 
\end{verbatim}
\end{codice}
Inizialmente $x$ ha quattro cifre dopo la virgola, impostando il formato 
\verb1long1 diventano quindici.
\begin{octave}In Octave le cifre del formato short sono cinque dopo la virgola,
si noti inoltre la differenza di stampa nella Command Window (che in Octave si
chiama Octave Terminal).
\begin{codice}
\begin{verbatim}
>>> x = 1/2
x =  0.50000
>>> format long
>>> x = 1/2
x =  0.500000000000000
\end{verbatim}
\end{codice}
\end{octave}

Matlab lavora con un sistema floating point in doppia precisione (si possono
aumentare i numeri dopo la virgola).

\subsection{Variabili speciali.}
In Matlab esistono alcune variabili speciali:
\begin{center}
\begin{tabular}{ll}
\verb1i1, \verb1j1 & unità immaginarie ($i^2 = -1, \ j^2 = -1$), \\
\verb1pi1 & pi greco, \\
\verb1realmax1 & il più grande numero di macchina, \\
\verb1realmin1 & il più piccolo numero di macchina, \\
\verb1Inf1 & $\infty$, ovvero un numero maggiore di \verb1realmax1, \\
\verb1nan1 & Not a number ($0/0$, etc.), \\
\verb1eps1 & epsilon macchina.
\end{tabular}\end{center}

\begin{codice}
\begin{verbatim}
>> pi

ans =

    3.1416
\end{verbatim}
\end{codice}

\subsection{Funzioni Built-In.}
In Matlab è possibile utilizzare una lista di funzioni ``pre-costruite'' come
\verb1sin1, \verb1cos1, etc... con la sintassi seguente:
\begin{center}
\begin{tabular}{ll}
\verb1sin(x)1 & calcola il seno della variabile \verb1x1, \\
\verb1cos(x)1 & calcola il coseno delella variabile \verb1x1, \\
\verb1sqrt(x)1 & calcola la radice quadrata di \verb1x1, \\
\verb1abs(x)1 & calcola il valore assoluto di \verb1x1, \\
\verb1log(x)1 & calcola il logaritmo naturale di \verb1x1, \\
\verb log10(x) & calcola il logaritmo di\verb1x1in base $10$.
\end{tabular}\end{center}

\begin{codice}
\begin{verbatim}
>> log(2)

ans =

    0.6931

>> log10(2)

ans =

    0.3010
\end{verbatim}
\end{codice}

Molte altre funzioni sono disponibili, per poterle visualizzare tutte è
possibile utilizzare il browser apposito cliccando sulla barra di navigazione
su Help/Function Browser, oppure premendo MAIUSC+F1.

\begin{octave}
La maggior parte delle funzioni sono presenti su Octave con lo stesso nome.
\begin{codice}
\begin{verbatim}
>>> log(2)
ans =  0.69315
\end{verbatim}
\end{codice}
Però, come già sottolineato, il formato short differisce per una cifra 
decimale, è quindi consigliabile usare il formato long.
Inoltre in Octave per visualizzare la lista è necessario andare sull'Help,
sempre sulla barra di navigazione: Help/Octave Help (anche premendo F1) e 
quindi in fondo all'indice delle pagine di aiuto, dopo le appendici,
cliccare su \emph{function index}.
\end{octave}

Per avere informazioni su funzioni specifiche è possibile usare
due comandi: \verb1help1 e \verb1lookfor1, del primo abbiamo già parlato, il
secondo invece è molto utile poiché permette di verificare tutti i comandi
che si riferiscano ad una parola chiave. La sintassi è: \verb1lookfor nome1,
dove \verb1nome1 può specificare il nome di un comando o di una parola chiave,
in inglese.
\begin{codice}
\begin{verbatim}
>> lookfor absolute
abs                            - Absolute value.
genelowvalfilter               - filters genes with low absolute expression levels.
mamadnorm                      - normalizes microarray data by median absolute deviation (MAD).
abs2active                     - Convert constraints from absolute format to active format.
active2abs                     - Convert constraints from active format to absolute format.
imabsdiff                      - Absolute difference of two images.
mae                            - Mean absolute error performance function.
dmae                           - Mean absolute error performance derivative function.
circlepick                     - Pick bad triangles using an absolute tolerance
mad                            - Mean/median absolute deviation.
>> 
\end{verbatim}
\end{codice}
 
\subsection{Modificare, Salvare e Caricare il Workspace.}
Nel Workspace sono presenti tutte le variabili che noi dichiariamo con la
semplice sintassi descritta. Ogni volta che eseguiamo un assegnamento
modifichiamo il Workspace, nello specifico modifichiamo la variabile a 
sinistra dell'assegnamento, se una funzione non ha tale variabile di default
viene modificata \verb1ans1.
\begin{exe}
Modifica del Workspace.
\begin{codice}
\begin{verbatim}
>> x = 1

x =

     1
>> x = [1 2 3]

x =

     1     2     3

\end{verbatim}
\end{codice}
In questo esempio la variabile \verb1x1 viene inizializzata come un array
di un solo elemento con valore $1$, il comando successivo la modifica in un
array di $3$ elementi. D'ora in avanti, durante il lavoro, ogni volta che
utilizzeremo \verb1x1 questa sarà come l'ultima modifica effetuata.
\end{exe}

Per evitare di cancellare dati importanti è consigliabile dare nomi diversi
alle variabili e sopratutto significativi, quando le variabili in gioco
cominciano a diventare numerose è possibile usare il comando \verb1exist1
per verificare l'esistenza o meno.
\begin{exe}
Ricerca di una variabile.
\begin{codice}
\begin{verbatim}
>> exist x

ans =

     1

>> exist y

ans =

     0

>> 
\end{verbatim}
\end{codice}
In questo caso \verb1x1 esiste ($1 = $ vero) mentre \verb1y1 no.
\end{exe}

E' possibile che si debba interrompere una sessione di lavoro, o che questa
debba essere suddivisa in intervalli temporali per vari motivi. Ovviamente
non occorre ogni volta che si riavvia Matlab reinserire tutte le variabili
manualmente per ricominciare il lavoro, con il comando \verb1save nomefile1 è
infatti possibile salvare l'intero contenuto del Workspace all'interno del 
file specificato. Per ricarice i dati salvati si usa il comando 
\verb1load nomefile1.
\begin{ese}
Si provi la seguente lista di comandi:
\begin{codice}
\begin{verbatim}
>> x = 1;
>> save walker
>> clear
>> exist x
>> load walker
>> exist x
\end{verbatim}
\end{codice}
\end{ese}

\section{Manipolare i dati.}
Come abbiamo già detto, i dati in Matlab sono tutti espressi in array. Con
la sintassi \verb1>> x = 2;1 si genera un array monodimenzionale di un solo
elemento ($1 \times 1$). Ma come si creano vettori di più elementi?
\begin{codice}
\begin{verbatim}
>> x = [1 2 3 4 5];
>> y = [1,2,3,4,5]

y =

     1     2     3     4     5

\end{verbatim}
\end{codice}
Questi due comandi generano due vettori riga (\verb1x1 e \verb1y1) 
completamente uguali
a parte per il nome, ovvero all'interno di parentesi elencare numeri
--elementi-- separati da uno spazio o una virgola genera un vettore riga.

Per generare un vettore colonna invece gli elmenti devono essere separati da
un ``a capo'' o da un punto e virgola.
\begin{codice}
\begin{verbatim}
>> x = [1 
2 
3 
4 
5];
>> y = [1;2;3;4;5]

y =

     1
     2
     3
     4
     5

>> 
\end{verbatim}
\end{codice}

Per generare una matrice si devono mischiare le due modalità, ovvero inserire
una colonna di righe o una riga di colonne (è possibile usare anche
le variabili come righe o colonne).
\begin{codice}
\begin{verbatim}
>> Y = [1 2 ; 3 4]

Y =

     1     2
     3     4
\end{verbatim}
\end{codice}


\subsection{Trasposta.}
Dato un array (vettore o matrice) è possibile calcolarne semplicemente la
\emph{trasposta} applicando un apice (') alla fine della definizione
del vettore o al nome del vettore.
\begin{codice}
\begin{verbatim}
>> y = [1;2;3;4;5]'

y =

     1     2     3     4     5

>> x'

ans =

     1     2     3     4     5
\end{verbatim}
\end{codice}

\subsection{Accesso agli elementi e dimensioni}
Per accedere ad un elemento specifico di un vettore o di una matrice è 
sufficiente utilizzare le parentesi rotonde applicate al nome dell'array
indicando l'indice o gli indici
dell'elemento desiderato.

\begin{codice}
\begin{verbatim}
>> x(2)

ans =

     2

>> X(2,1)

ans =

     0
\end{verbatim}
\end{codice}

Dato un array è possibile conoscerne la lunghezza con il comando 
\verb1length nomevettore1, oppure  le dimensioni con il comando
\verb1size nomevettore1.

\begin{codice}
\begin{verbatim}
>> length x

ans =

     1

>> size X

ans =

     1     1
\end{verbatim}
\end{codice}

\subsection{linspace(a,b,n).}
\begin{codice}
\begin{verbatim}
>> x = linspace(a,b,n);
\end{verbatim}
\end{codice}
Crea un vettore equispaziato da $\verb1a1$ a $\verb1b1$ di $n$ elementi.
Il parametro \verb1n1 è facoltativo, se non presente di default Matlab
imposta $100$ elementi.

\subsection{logspace(a,b,n).}
\begin{codice}
\begin{verbatim}
>> x = linspace(a,b,n);
\end{verbatim}
\end{codice}
Crea un vettore di $n$ elementi intervallati logaritmicamente da $10^{\verb1a1}$
 a 
$10^{\verb1b1}$. Il parametro \verb1n1 è facoltativo, se non presente di default 
Matlab imposta $50$ elementi. Utile per creare grafici in scala logaritmica
per dati di grandi dimensioni.

\subsection{Assegnazione a blocchi.}
Matlab da la possibilità di eseguire operazioni fondamentali senza il bisogno
di accedere agli elementi di vettori e matrici direttamente magari usando
comandi come cicli \verb1for1 o \verb1while1. Ad esempio è possibile creare 
matrici con assegnazioni a blocchi.
\begin{codice}
\begin{verbatim}
>> x = [1 2];
>> y = [3 4];
>> z = [x y x]

z =

     1     2     3     4     1     2

>> B = [x;y]

B =

     1     2
     3     4
\end{verbatim}
\end{codice}
\begin{notabene}
I blocchi devono essere di dimensioni compatibili.
\end{notabene}
\begin{codice}
\begin{verbatim}
>> C = [y' B; 5 6 7]

C =

     3     1     2
     4     3     4
     5     6     7
\end{verbatim}
\end{codice}
\begin{codice}
\begin{verbatim}
>> C = [y B; 5 6 7]
??? Error using ==> horzcat
CAT arguments dimensions are not consistent.
 
\end{verbatim}
\end{codice}

\subsection{Notazione ``due punti'' o ``colon''.}
E' possibile usare, alternativamente a \verb1linspace1 la notazione ``colon''
per creare vettori equispaziati, utile sopratutto per array di piccole 
dimensioni, la sintassi è la seguente:
\begin{center}
\verb1vettore = inizio : passo : fine1
\end{center}
Il \verb1passo1 se non specificato di default è $1$, ovvero il vettore che ne 
risulta avrà come valore iniziale \verb1inizio1 e di seguito gli altri 
elementi saranno incrementati di $1$ fino all'ultimo elemento (\verb1fine1),
un valore negativo invece crea il vettore dal valore più alto a quello più
basso. 
\begin{codice}
\begin{verbatim}
>> x = (2:5)

x =

     2     3     4     5

>> y = (5:2)

y =

   Empty matrix: 1-by-0

>> z = (5:-1:2)

z =

     5     4     3     2
\end{verbatim}
\end{codice}
\begin{notabene}
Si noti che \verb1y1 è un vettore vuoto di dimensioni $1 \times 0$
\end{notabene}

Un altro utilizzo per la notazione ``colon'' è per selezionare un'intera
colonna o una riga in una matrice.
\begin{codice}
\begin{verbatim}
>> C(:,1)

ans =

     3
     4
     5

>> C(1,:)

ans =

     3     1     2
\end{verbatim}
\end{codice}
Oppure per la selezione di un ``range'' all'interno di un vettore o una
matrice. 
\begin{codice}
\begin{verbatim}
>> C(1:2,2:3)

ans =

     1     2
     3     4

\end{verbatim}
\end{codice}
Qui ad esempio abbiamo estratto la sottomatrice di \verb1C1 data dalla
prima alla seconda riga e dalla seconda alla terza colonna.
\begin{codice}
\begin{verbatim}
>> A = [1 2 3; 4 5 6; 7 8 9]

A =

     1     2     3
     4     5     6
     7     8     9

\end{verbatim}
\end{codice}
L'ultima utilità che vediamo per la notazione ``colon'' è quella per la 
modifica di una matrce o un vettore, partiamo dalla matrice \verb1A1 di cui
sopra ed applichiamo la notazione per alcune modifiche.
\begin{codice}
\begin{verbatim}
>> A(1,:) = (2:2:6)

A =

     2     4     6
     4     5     6
     7     8     9

\end{verbatim}
\end{codice}
Qui alla prima riga abbiamo sostituito il vettore dato da \verb12:2:61.
\begin{codice}
\begin{verbatim}
>> A(:,3) = (1:3)'

A =

     1     2     1
     4     5     2
     7     8     3


\end{verbatim}
\end{codice}
Alla terza colonna abbiamo sostituito il vettore colonna \verb (1:3)' .

\begin{codice}
\begin{verbatim}
>> A(:,1) = []

A =

     2     3
     5     6
     8     9

\end{verbatim}
\end{codice}
Come si nota, è possibile anche eliminare un intero vettore colonna o riga,
modificando la struttura della matrice.

\subsection{zeros(), ones(), rand() e eye()}
Alcune funzioni utili per costruire vettori e matrici particolari sono:
\begin{center}
\begin{tabular}{ll}
\verb1zeros(righe, colonne)1& crea un array contenete solo elementi \verb101, \\
\verb1ones(righe, colonne)1 & crea un array contenete solo elementi \verb 1 , \\
\verb1rand(righe, colonne)1 & crea un array contenete elementi casuali, \\
\verb1eye(righe, colonne)1 & crea una matrice identità.
\end{tabular}
\end{center}
I parametri \verb1righe1 e \verb1colonne1 indicano il numero di righe e di 
colonne della matrice risultante.
\begin{codice}
\begin{verbatim}
>> zeros(2,2)

ans =

     0     0
     0     0

>> ones(2,2)

ans =

     1     1
     1     1

>> rand(2,2)

ans =

    0.8147    0.1270
    0.9058    0.9134

>> eye(2,4)

ans =

     1     0     0     0
     0     1     0     0

\end{verbatim}
\end{codice}
Come si può notare è possibile dare dimensioni differenti dalla canonica
forma quadrata, anche nella matrice identità.

\subsection{Operatori su vettori e matrici.}
Sia \verb1x1 un vettore di $n$ elementi tale che $x_i \in \verb1x1$ per ogni
$i \in [0,n-1]$, sia \verb1A1 una matrice di $n \times m$ elementi.
In Matlab sono definite le seguenti funzioni:

\begin{center}
\begin{tabular}{ll}
\verb1a = sum(x)1 & genera lo scalare $\verb1a1 = \sum_ix_i$, \\
\verb1a = prod(x)1 & genera lo scalare $\verb1a1 = \prod_ix_i$, \\
\verb1a = max(x)1 & genera lo scalare $\verb1a1 = \textrm{max}\ x_i, \quad i 
= 1, \ldots n$, \\
\verb1a = min(x)1 & genera lo scalare $\verb1a1 = \textrm{min}\ x_i, \quad i 
= 1, \ldots n$, \\
$\verb1a = norm(x)1$ & genera lo scalare $\verb1a1 = \|x\|_2$, \\
$\verb1a = norm(x,1 \verb 1) $ & genera lo scalare $\verb1a1 = \|x\|_1$, \\
\verb1a = norm(x,inf)1 & genera lo scalare $\verb1a1 = \|x\|_{\infty}$, \\
\verb1A = diag(x)1 & crea una matrice con diagonale il vettore \verb1x1, \\
\verb1a = norm(A)1 & genera lo scalare $\verb1a1 = \|A\|_2$, \\
\verb1a = norm(A,1)1 & genera lo scalare $\verb1a1 = \|A\|_1$, \\
\verb1a = norm(A,inf)1 & genera lo scalare $\verb1a1 = \|A\|_{\infty}$, \\
\verb1x = sum(A)1 & genera il vettore riga $\verb1x1 = x_{i,j} = \sum_ia_{i,j},
\quad j = 1, \ldots n$, \\
\verb1x = max(A)1 & genera il vettore riga $\verb1x1 = x_{i,j} = \textrm{max}
 \ a_{i,j}, \quad j = 1, \ldots n$, \\
\verb1x = min(A)1 & genera il vettore riga $\verb1x1 = x_{i,j} = \textrm{min}
 \ a_{i,j}, \quad j = 1, \ldots n$, \\
\verb1x = diag(A)1 & genera il vettore colonna dato dalla diagonale di 
\verb1A1, \\
\verb1B = abs(A)1 & genera la matrice dei valori assoluti, \\
\verb1U = triu(A)1 & genera la matrice triangolare superiore con gli elementi\\
& non nulli uguali a quelli di \verb1A1.\\
\verb1U = tril(A)1 & genera la matrice triangolare inferiore con gli elementi\\
& non nulli uguali a quelli di \verb1A1.
\end{tabular}
\end{center}

\subsection{Operatori algebrici sugli array.}
Siano \verb1A1 e \verb1B1 array (matrici o vettori), Matlab definisce due
tipi generici di operazioni: le operazioni puntuali e le operazioni tra
array. Le operazioni tra array sono quelle con la consueta sintassi, le 
operazioni puntuali invece hanno come sintazzi un punto prima dell'operazione
desiderata. Ad esempio un prodotto matriciale riga per colonna si effettua
con il seguente comando: \verb1A*B1, se invece avessimo voluto moltiplicare
elemento per elemento (nella stessa posizione) il comando sarebbe stato:
\verb1A.*B1.

\begin{codice}
\begin{verbatim}
>> A = 2*ones(2,2)

A =

     2     2
     2     2

>> A.^2

ans =

     4     4
     4     4

>> A^2

ans =

     8     8
     8     8
\end{verbatim}
\end{codice}
Si noti che \verb1A.^21 equivale al comando \verb1A.*A1, ovvero moltiplica
ciascun elemento per se stesso, invece \verb1A^21 equivale a \verb1A*A1,
ovvero prodotto riga per colonna.

Come il prodotto è possibile utilizzare tutte le operazioni su vettori,
matrici, scalari, etc.

\chapter{Lavorare con Matlab.}

\section{Funzioni dell'algebra lineare.}
Come si è visto nell'introduzione a Matlab, i dati su cui si lavora sono
matrici e vettori. In questa sezione vedremo quindi l'utilizzo delle principali
funzioni Matlab basate sull'algebra lineare.

\begin{itemize}
\item \verb1d = det(A) 1  calcola il determinante di $A$.
\item \verb1B = inv(A) 1  calcola la matrice inversa di $A$.
\item \verb1H = hilb(n) 1  costruisce la matrice di Hilbert: $H(i,j) = 
\frac{1}{i+j-1}$.
\item \verb1V = vander(v) 1 costruisce la matrice di Vandermonde di ordine
$n = \verb1length(v)1$: $V(i,j) = v(i)^{(n-j)}$.
\item \verb1[M,D] = eig(A) 1 calcola gli autovalori e autovettori di $A$.
\item \verb1 A\b 1 calcola la soluzione del sistema lineare $Ax = b$ con
l'eliminazione di Gauss.
\end{itemize}

\begin{exe}Matrice di Hilbert e suo determinante.

\begin{codice}
\begin{verbatim}
>> format rat
>> A = hilb(4)

A =

       1              1/2            1/3            1/4     
       1/2            1/3            1/4            1/5     
       1/3            1/4            1/5            1/6     
       1/4            1/5            1/6            1/7     

>> m1 = A(1,1)

m1 =

       1       

>> m2 = A(1:2,1:2), dm2 = det(m2)

m2 =

       1              1/2     
       1/2            1/3     


dm2 =

       1/12    

>> m3 = A(1:3,1:3), dm3 = det(m3)

m3 =

       1              1/2            1/3     
       1/2            1/3            1/4     
       1/3            1/4            1/5     


dm3 =

       1/2160  

>> m4 = det(A)

m4 =

       1/6048000

>> 
\end{verbatim}
\end{codice}
In questo esempio vediamo un nuovo formato: \verb1format rat1 che visualizza la
notazione razionale dei nostri dati.

Le funzioni \verb1hilb()1 e \verb1det()1, si comportano esattamente come ci si
aspetta: con la prima si costruisce una matrice di ordine $4$, mentre con la 
seconda un \emph{vettore} $1 \times 1$ il cui unico elemento è il valore
del determinante.
\end{exe}

\begin{exe}Matrice di Vandermonde e calcolo dell'inversa.

\begin{codice}
\begin{verbatim}
>> v = [1:4]; V = vander(v)

V =

       1              1              1              1       
       8              4              2              1       
      27              9              3              1       
      64             16              4              1       

>> inv_V = inv(V)

inv_V =

      -1/6            1/2           -1/2            1/6     
       3/2           -4              7/2           -1       
     -13/3           19/2           -7             11/6     
       4             -6              4             -1       

>> 
\end{verbatim}
\end{codice}
\verb1v1 è il vettore $[1,2,3,4]$, viene quindi creata la matrice $V$ di
Vandermonde come ci si aspetta dal comando \verb1vander()1, quindi con la 
funzione \verb1inv()1 si calcola l'inversa.
\end{exe}

\begin{exe}Autovalori e autovettori.

\begin{codice}
\begin{verbatim}
>> v = [0.1:0.1:0.5]; [M,D] = eig(vander(v))

M =

    -641/1844      -903/1345      1206/1345       521/892      -1236/1405  
    -521/1355      -397/702        739/2270      -174/235        411/1310  
   -3585/8339      -963/2413     -1189/10328      129/394        615/1933  
    -605/1244      -249/1691     -1018/3795      -243/4162     -1259/7785  
    -976/1753        24/109        194/2795        87/25672       95/5587  


D =

     939/535          0              0              0              0       
       0           -452/1511         0              0              0       
       0              0            169/3621         0              0       
       0              0              0              8/32799        0       
       0              0              0              0           -100/20753 

>> 
\end{verbatim}
\end{codice}

\end{exe}

\begin{exe}Risoluzione di un sistema lineare.

\begin{codice}
\begin{verbatim}
>> A = vander(1:3); b=[3;7;13];
>> x = A\b

x =

       1       
       1       
       1       

>> A*x

ans =

       3       
       7       
      13       

\end{verbatim}
\end{codice}

\end{exe}

\section{File Diario.}
Il comando \verb1diary1 permette di registrare l'attività svolta in una sessione
di lavoro con Matlab. L'utilizzo è il seguente:
\begin{itemize}
\item \verb1diary 1 registra una sessione di lavoro e la salva in un file di 
nome diary.
\item \verb1diary nomefile 1 registra una sessione di lavoro e la salva in un 
file di nome \verb1nomefile1.
\item \verb1diary off 1 disattiva la registrazione.
\item \verb1diary on 1 riattiva la registrazione. 
\end{itemize}
I file vengono salvati in formato ASCII.
\begin{exe}Utilizzo del comando \verb1diary1.

\begin{codice}
\begin{verbatim}
>> diary esempio
>> x = linspace(0,pi,10);
>> y = (15120 -6900*x.^2 + 313*x.^4)./(15120+660*x.^2 +13*x.^4);
>> [x',y']

ans =

       0              1       
     355/1017       857/912   
     710/1017      1313/1714  
     355/339     441131/882261
    1420/1017       207/1192  
    1775/1017      -109/628   
     710/339       -546/1093  
    1102/451       -554/725   
    2840/1017    -15007/16079 
     355/113      -4069/4143  

>> diary off
>> % fine registrazione
>> 
>> type esempio

x = linspace(0,pi,10);
y = (15120 -6900*x.^2 + 313*x.^4)./(15120+660*x.^2 +13*x.^4);
[x',y']

ans =

       0              1       
     355/1017       857/912   
     710/1017      1313/1714  
     355/339     441131/882261
    1420/1017       207/1192  
    1775/1017      -109/628   
     710/339       -546/1093  
    1102/451       -554/725   
    2840/1017    -15007/16079 
     355/113      -4069/4143  

diary off

>> 
\end{verbatim}
\end{codice}
Il comando \verb1type nomefile1 infine permette di visualizzare il contenuto
di \verb1nomefile1.
\end{exe}

\section{Tipi di file Matlab.}
In Matlab ci sono tre tipi di file:
\begin{itemize}
\item Mat File, con estensione \verb1.mat1.
\begin{codice}
\begin{verbatim}
>> save esempio
\end{verbatim}
\end{codice}
salva il file \verb1esempio.mat1, contente l'ambiente di lavoro Matlab.
\item M-file, con estensione \verb1.m1, per memorizzare programmi e funzioni
Matlab.
\item File dati, con estensione \verb1.txt1, in formato ASCII.
\end{itemize}

In Matlab possiamo operare in due modi, in modalità interattiva (a riga di 
comando dalla shell) oppure tramite l'esecuzione di programmi o script.

\section{Creare e utilizzare M-file.}
Gli M-file possono contenere script o funzioni. La principale differenza 
``semantica'' tra lo script e la function è che nel primo le variabili sono
globali, e quindi vengono salvate nel workspace, nelle function invece le
variabili sono memorizzate localmente e non vanno a sporcare l'ambiente di 
lavoro.

Inoltre una function ha delle variabili in input ed output, mentre lo script
modifica e crea variabili date nel workspace, in una function l'unico
modo di modificare variabili è quello di usare il meccanismo di output.

Per creare un M-file è necessario:
\begin{enumerate}
\item selezionare \verb1New1 dal menù \verb1File1 e poi \verb1M-file1;
\item digitare i comandi che compongono lo script o la function;
\item selezionare \verb1save1 dal menù \verb1File1 della nuova finestra;
\item dare un nome coerente al file se si tratta di uno script, o il nome
esatto della function.
\end{enumerate}

\subsection{Function.}
Le function Matlab sono utili quando occorre ripetere più volte una serie
di comandi, in modo particolare quando questi comandi hanno un senso
particolare, ad esempio un algoritmo.

Per quanto riguarda la sintassi, la prima riga dell'M-file deve contenere la
parola chiave \verb1function1 seguita dai parametri di output racchiusi tra
parentesi quadre, il nome della funzione ed i nomi delle variabili in input
tra parentesi rotonde.

\begin{exe}
Il codice seguente, salvato in un file \verb1risolvi.m1, implementa una
funzione che genera il vettore soluzione \verb1x1 del problema lineare
$ Ax = b$.
\begin{codice}
\begin{verbatim}
function [x] = risolvi(A,b)
% risolvi risolve il generico problema Ax = b
% con metodo di Gauss.

x = A\b;

end
\end{verbatim}
\end{codice}
Il suo utilizzo è il seguente.
\begin{codice}
\begin{verbatim}
>> A = rand(4);
>> b = rand(4,1);
>> x1 = A\b

x1 =

   17.2819
    0.8395
  -15.9067
    1.0883

>> x = risolvi(A,b)

x =

   17.2819
    0.8395
  -15.9067
    1.0883

>> 
\end{verbatim}
\end{codice}
Questo è solo un esempio di come possono essere utilizzate le funzioni, in 
genere convengono quando l'algoritmo richiede più linee di codice.
\end{exe}

\begin{exe}
La seguente funzione, invece è più complessa, ha come parametri di output
due variabili: $A_n$ e $P$. Di conseguenza il metodo di invocazione è differente
e, come si può notare, ha molte più righe di codice ed ha più senso di essere
scritta come funzione di quella vista in precedenza.
\begin{codice}
\begin{verbatim}
function [An,P] = permuta(A)
% Data una matrice A analizza l'elemento
% di indice (1,1) e se è uguale a zero effettua uno scambio di righe.
% Utilizza la tecnica del pivoting parziale.

n = size(A);
An = A;
P = eye(n(1));

if(A(1,1)==0)
   [ma, j] = max(abs(A(:,1)));
   if(ma ~= 0)
    An(1,:) = A(j,:); 
    An(j,:) = A(1,:);
    P(1,:) = P(j,:);
    P(j,:)= eye(1,n);
   end
end
\end{verbatim}
\end{codice}
Ecco come si utilizza, da notare che l'output è un array di due elementi.

\begin{codice}
\begin{verbatim}
>> A = [ 0 1 1 2 4
1 0 3 3 3
5 5 5 5 5]

A =

     0     1     1     2     4
     1     0     3     3     3
     5     5     5     5     5

>> [An, P] = permuta(A)

An =

     5     5     5     5     5
     1     0     3     3     3
     0     1     1     2     4


P =

     0     0     1
     0     1     0
     1     0     0

\end{verbatim}
\end{codice}
I dati di input, se si verifica il workspace, sono inalterati. Gli unici dati
aggiunti sono quelli di output, delle variabili interne alla funzione non vi
è traccia.
\end{exe}

\subsection{Script.}
Gli script Matlab sono di fatto file di comandi, contengono una sequenza di 
comandi Matlab, comprese anche funzioni definite da utente.

Tutte le variabili contenute negli script vengono salvate nel workspace.
Vediamo dunque un esempio di prima in forma di script.
\begin{exe}Il seguente codice è inserito nel file \verb1esempio.m1.
\begin{codice}
\begin{verbatim}
A = rand(4)
b = rand(4,1)
x = A\b
\end{verbatim}
\end{codice}
Per eseguirlo è sufficiente digitare \verb1esempio1, ovvero il nome dello script
salvato nella directory corrente.
\begin{codice}
\begin{verbatim}
>> esempio

A =

    0.3804    0.5308    0.5688    0.1622
    0.5678    0.7792    0.4694    0.7943
    0.0759    0.9340    0.0119    0.3112
    0.0540    0.1299    0.3371    0.5285


b =

    0.1656
    0.6020
    0.2630
    0.6541


x =

   -0.7515
    0.0062
    0.5057
    0.9902
\end{verbatim}
\end{codice}
\end{exe}
