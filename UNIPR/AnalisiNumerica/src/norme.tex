\hyphenation{
Barrow
Binet
Chebyshev
Cholesky
Cramer
Gauss
Faber
Frobenius
Hausdorff
Householder
Laplace
Lebesque
Newton
Rolle
Runghe
Sturm
tras-for-ma-zio-ne
Torricelli
}

%\addtocontents{toc}{\protect\contentsline{chapter} {Appendice}{}}
\chapter{Norme.}
\section{Norma di un vettore.}
La norma di un vettore è un'applicazione $\|\cdot\| \colon \rr^n \to \rr^+$
tale che:
\begin{itemize}
\item[$(1)$]$\|x\| \geq 0, \quad \forall x \in \rr^n$.
\item[$(2)$]$\|x\| = 0 \Leftrightarrow x = 0$.
\item[$(3)$]$\|ax\| = |a|\cdot\|x\|, \quad \forall x \in \rr^n$.
\item[$(4)$]$\forall x,y \in \rr^n \quad \|x+y\| \leq \|x\| + \|y\|$.
\end{itemize}

\begin{defi}Si definisce la
norma $p$ con $1 \leq p < +\infty$ tale che:
\[\|x\|_p = \left(\sum_{i=1}^n|x_i|^p \right)^{\frac{1}{p}}.\]
\[\|x\|_\infty = \max_{1 \leq i \leq m}|x_i|.\]
\end{defi}

\begin{exe}
Sia $x = (1, -2)$.
\[\|x\|_1 = 3, \quad \|x\|_2 = \sqrt{5}, \quad \|x\|_\infty = 2.\]
\end{exe}

\begin{defi}Due norme si dicono \emph{topologicamente equivalenti} se
esistono due costanti $\alpha, \beta \in \rr, 0 < \alpha \leq \beta$ tali che:
\[
\alpha \|x\|'' \leq \|x\|' \leq \beta\|x\|'.
\] 
\end{defi}

\begin{teo}
$\forall x \in \rr^n$ si ha:
\begin{enumerate}
\item $\|x\|_\infty \leq \|x\|_2 \leq \sqrt{n}\|x\|_\infty$.
\item $\|x\|_2 \leq \|x\|_1 \sqrt{n}\|x\|_2$.
\item $\|x\|_\infty \leq \|x\|_1 \leq n\|x\|_\infty$.
\end{enumerate}
\end{teo}

\section{Norma di una matrice.}
La norma di una matrice è un'applicazione analoga a quella sui vettori, 
valgono tutti i punti di cui sopra e in aggiunta:
\begin{itemize}
\item[$(5)$] $\|A\cdot B\| \leq \|A\|\cdot\|B\|$.
\end{itemize}

\begin{exe}Esempio di norma che non soddisfa la proprietà $(5)$:
\[
A = B = \left[\begin{array}{cc}1 & 1 \\ 0 & 1\end{array}\right], \quad
C = AB = \left[\begin{array}{cc}1 & 2 \\ 0 & 1\end{array}\right].
\]
\[
\max_{i,j}|c_{i,j}| = 2, \quad \max_{i,j}|a_{i,j}| = \max_{i,j}|b_{i,j}| = 1.
\]
\[2 \leq 1 \ \longrightarrow\ \textrm{Assurdo.} \]
\end{exe}

\begin{defi}
Si dice \emph{norma naturale} o \emph{norma indotta da un vettore} della
matrice $A$ il numero reale $\|A\|$ tale che:
\[
\|A\|= \sup_{x \neq 0}\frac{\|Ax\|}{\|x\|}.
\]
\end{defi}
\subsubsection{Proprietà:}
\begin{itemize}
\item[$(1)$]$\|I\| =1$.
\item[$(2)$]$\|Ax\| \leq \|A\|\|x\|$.
\item[$(3)$]$\rho(A)\leq \|A\|$.
\end{itemize}

$\rho(A)$ è il massimo autovalore di $A$ in modulo.

\begin{flushleft}
Nel condizionamento:
\[
1 = \|I\| = \|A\cdot A^{-1}\| \leq \|A\|\cdot\|A\| = 
\textrm{cond(}A\textrm{)}.
\]
\end{flushleft}

\begin{prop}
La norma naturale indotta da $\|\cdot\|_1$ è tale che:
\[
\left\|A\right\|_1 = \max_j \sum_{i = 1}^n|a_{i,j}|, \qquad A \in \rr^{n\times n}.
\]
\end{prop}
\begin{dimo}
\[ \left\|Ax\right\|_1  =  \left\| \sum_{j=1}^n a_{i,j}x_j\right\|_1 \]
\[ =  \sum_{i=1}^n\left|\sum_{j=1}^n a_{i,j}x_j\right| \leq  
\sum_{i=1}^n\sum_{j=1}^n |a_{i,j}||x_j| \]
\[ =  \sum_{j=1}^n|x_j|\sum_{i=1}^n |a_{i,j}|
\leq  \sum_{i=1}^n|x_j| \max_j\sum_{i=1}^n |a_{i,j}| \]
\[ =  \|x\|_1 (\max_j\sum_{i=1}^n |a_{i,j}|). \]

\[
\longrightarrow \ \frac{\|Ax\|_1}{\|x\|_1} \leq \sum_{i=1}^n |a_{i,j}|.
\]
\end{dimo}


\subsubsection{Norme indotte:}
\begin{enumerate}
\item $\|A\|_1 = \max_j \sum_{i=1}^n|a_{i,j}|$.
\item $\|A\|_2 = \sqrt{\rho(A^TA)}$. 
\item $\|A\|_\infty = \max_i \sum_{j=1}^n|a_{i,j}|$.
\end{enumerate}

\subsection{Norma di Frobenius.}
\begin{defi}
Sia $A \in \rr^{n \times n}$, la \emph{norma di Frobenius} è definita come 
segue:
\[
\|A\|_F = \left( \sum_{i=1}^n\sum_{j=1}^na_{i,j}^2\right)^{\frac{1}{2}}.
\]
\end{defi}
E' una norma compatibile con la norma due di vettore.
\[
\|A\|_2 \leq \|A\|_F\|x\|_2.
\]
La sintassi del comando Matlab, data una matrcie quadrata $X$, è la seguente:
\begin{codice}
\begin{verbatim}
>>> norm(X,'fro')
\end{verbatim}
\end{codice}

\subsubsection{Proprietà:}

\begin{enumerate}
\item $\frac{1}{\sqrt{n}}\|A\|_\infty \leq \|A\|_2 \leq \sqrt{n}\|A\|_\infty$.
\item $\frac{1}{\sqrt{n}}\|A\|_1 \leq \|A\|_2 \leq \sqrt{n}\|A\|_1$.
\item $\max_{i,j} |a_{i,j}| \leq \|A\|_2 \leq n \max_{i,j}|a_{i,j}|$.
\item $\|A\|_2 \leq \sqrt{\|A\|_1 \|A\|_\infty}$.
\end{enumerate}
