\hyphenation{
un'es-pres-sio-ne
JSetL
Choco
JaCoP
}
\chapter{La Libreria JSetL}\label{jsetl}
In questo capitolo verrà descritta brevemente la libreria JSetL, soffermandosi
principalmente sull'utilizzo, ovvero sui costrutti e i metodi utilizzati
nell'implementazione concreta di JSR-331. 

JSetL è una libreria Java che combina il paradigma di programmazione orientato 
agli oggetti con i concetti dei linguaggi \emph{CLP} (\emph{Constraint Logic 
Programming}), 
come variabili logiche, elenchi,
unificazione, risoluzione di vincoli, non determinismo.

Tra le caratteristiche di interesse per l'implementazione JSR-331 troviamo le
variabili logiche (\files{IntLVar} e \files{SetLVar}), i vincoli 
(\files{Constraint}) ed il meccanismo di risoluzione dei vincoli 
(\files{SolverClass}).

\section{Variabili logiche}
JSetL supporta la nozione di variabile logica come quella che si trova in
Logica e nei linguaggi di programmazione funzionale. Le variabili logiche 
possono essere sia \emph{inizializzate} che \emph{non inizializzate}.
Il valore di una variabile logica in JSetL può essere di qualsiasi tipo.
Un valore può essere assegnato ad una variabile logica (non inizializzata) come
risultato dell'elaborazione dei vincoli che la coinvolgono.

\begin{flushleft}\textbf{Variabile logica.}\\
Una \emph{variabile logica} è un'istanza della classe \files{LVar}, creata
dalla seguente chiamata:
\begin{center}
\lstinline[language = Java]$LVar varName = new LVar(ExtVarName, VarValue);$
\end{center}
dove \files{varName} è il nome della variabile, \files{ExtVarName} è il nome
opzionale esterno e \files{VarValue} è un parametro opzionale che ne
rappresenta il valore.
\end{flushleft}

Il nome esterno è una stringa che può essere utile quando occorre stampare la
variabile ed i possibili vincoli che la coinvolgono. 

\begin{nota}
Durante la fase di sviluppo dell'implementazione la stampa è stata una parte 
fondamentale per il debugging, sia per quanto riguarda le prove d'esecuzione che
per i test con JUnit e con il TCK. Ovviamente il nome esterno non definisce
una variabile (essendo opzionale) e sarebbe possibile creare variabili con
il medesimo \files{ExtVarName} ma indipendenti. Tuttavia, per 
facilitare il riconoscimento all'occhio umano è buona norma definire nomi
significativi, per quanto riguarda i nomi delle variabili temporanee interne
si rimanda alla sezione \ref{problem}.
\end{nota}

\section{Variabili logiche intere}
Le variabili logiche intere sono quelle d'interesse al fine di implementare le 
classi  \files{Var} e \files{VarBool} per l'attuale implementazione JSR-331. 

In JSetL le variabili intere sono rappresentate
dalla classe \files{IntLVar}, che di fatto estende la
classe \files{LVar} ereditandone metodi e costruttori.

\begin{flushleft}\textbf{Variabile logica intera.}\\
Una \emph{variabile logica intera} è un'istanza della classe \files{IntLVar}, 
creata dalla seguente chiamata:
\begin{center}
\lstinline[language = Java]$IntLVar varName = new IntLVar(name);$
\end{center}
dove  \files{varName} è il nome della variabile e \files{name} è il nome
esterno opzionale.
\end{flushleft}

\subsection{Dominio}
Ogni variabile logica intera ha un dominio ad essa associato, nel caso di
\files{IntLVar} il dominio è rappresentato da un multi-intervallo di interi.

Il concetto di multi-intervallo è definito in \cite{tesiAmadini} e, in breve, è
costituito dall'unione disgiunta di intervalli di interi. Ogni volta che
la variabile in questione viene modificata tramite vincoli che la coinvolgono
il relativo dominio viene modificato di conseguenza.

\subsection{Metodi utili}
La classe \files{IntLVar} fornisce alcuni metodi di utilità, quelli più
utilizzati nell'implementazione sono:
\begin{itemize}
\item[-]\files{MultiInterval getDomain()}: restituisce il
dominio della variabile intera.
\item[-]\files{Boolean isBound()}: restituisce 
\verb true  se
il dominio è ristretto ad un solo elemento, \verb false  in caso contrario.
\end{itemize}

\subsection{Operazioni aritmetiche}
\`E possibile creare variabili vincolate intere anche partendo da una
variabile d'invocazione (\files{this}), mediante le comuni operazioni
aritmetiche. Siano $X_0$ la variabile d'invocazione \files{x0}, $\mathcal{C}_0$ 
i vincoli ad essa associati, $X_1$ la variabile data dal parametro \files{x1}, 
$\mathcal{C}_1$ i vincoli ad essa associati, $X_2$ la nuova variabile \files{x2}
ottenuta dalle seguenti operazioni:
\begin{itemize}
\item[-]\files{IntLVar sum(Integer k)}: crea una variabile logica intera 
vincolata \files{x2} tale che se \files{x2 = x0.sum(k)}, allora:
\[
X_2 = X_0 + k \wedge \mathcal{C}_0.
\]
\item[-]\files{IntLVar sum(IntLVar x1)}: crea una variabile logica intera 
vincolata tale che se \files{x2 = x0.sum(x1)}, allora:
\[
X_2 = X_0 + X_1 \wedge \mathcal{C}_0 \wedge \mathcal{C}_1.
\]
\item[-]\files{IntLVar sub(Integer k)}: crea una variabile logica intera 
vincolata tale che se \files{x2 = x0.sub(k)}, allora:
\[
X_2 = X_0 - k \wedge \mathcal{C}_0.
\]
\item[-]\files{IntLVar sub(IntLVar x1)}: crea una variabile logica intera 
vincolata tale che se \files{x2 = x0.sub(x1)}, allora:
\[
X_2 = X_0 - X_1 \wedge \mathcal{C}_0 \wedge \mathcal{C}_1.
\]
\item[-]\files{IntLVar mul(Integer k)}: crea una variabile logica intera 
vincolata tale che se \files{x2 = x0.mul(k)}, allora:
\[
X_2 = X_0 \cdot k \wedge \mathcal{C}_0.
\]
\item[-]\files{IntLVar mul(IntLVar x1)}:  crea una variabile logica intera 
vincolata tale che se \files{x2 = x0.mul(x1)}, allora:
\[
X_2 = X_0 \cdot X_1 \wedge \mathcal{C}_0 \wedge \mathcal{C}_1.
\] 
\item[-]\files{IntLVar div(Integer k)}: crea una variabile logica intera 
vincolata tale che se \files{x2 = x0.div(k)}, allora:
\[
X_2 = \frac{X_0}{k} \wedge k \neq 0 \wedge \mathcal{C}_0.
\]
\item[-]\files{IntLVar div(IntLVar x1)}:  crea una variabile logica intera 
vincolata tale che se \files{x2 = x0.div(x1)}, allora:
\[
X_2 = \frac{X_0}{X_1} \wedge X_1 \neq 0 \wedge \mathcal{C}_0 \wedge 
\mathcal{C}_1.
\]
\end{itemize}

\subsection{Vincoli}
La classe \files{IntLVar} fornisce metodi per la creazione dei comuni vincoli
che rappresentano le relazione su interi ($<$, $=$, $\neq$, \ldots). 
Inoltre fornisce altri
vincoli comunemente usati come \emph{AllDifferent}, vincoli di dominio e di
appartenenza.

Di seguito si elencano e descrivono brevemente i vincoli utilizzati 
nell'implementazione, siano
$X_0$ la variabile d'invocazione \files{this}, $\mathcal{C}_0$ i vincoli ad 
essa associati, $X_1$ la variabile data dal parametro \files{x1} e 
$\mathcal{C}_1$ i relativi vincoli associati:
\begin{itemize}
\item[-]\files{Constraint eq(Integer k)}: 
crea un nuovo vincolo $\mathcal{C}$ tale che:
\[
\mathcal{C} \leftarrow X_0 = k \wedge \mathcal{C}_0.
\]
\item[-]\files{Constraint eq(IntLVar x1)}: crea un nuovo vincolo $\mathcal{C}$ 
tale che:
\[
\mathcal{C} \leftarrow X_0 = X_1 \wedge \mathcal{C}_0 \wedge \mathcal{C}_1.
\]
\item[-]\files{Constraint neq(Integer k)}: 
crea un nuovo vincolo $\mathcal{C}$ tale che:
\[
\mathcal{C} \leftarrow X_0 \neq k \wedge \mathcal{C}_0.
\]
\item[-]\files{Constraint neq(IntLVar x1)}: crea un nuovo vincolo $\mathcal{C}$ 
tale che:
\[
\mathcal{C} \leftarrow X_0 \neq X_1 \wedge \mathcal{C}_0 \wedge \mathcal{C}_1.
\]
\item[-]\files{Constraint le(Integer k)}: 
crea un nuovo vincolo $\mathcal{C}$ tale che:
\[
\mathcal{C} \leftarrow X_0 \leq k \wedge \mathcal{C}_0.
\]
\item[-]\files{Constraint le(IntLVar x1)}: crea un nuovo vincolo $\mathcal{C}$ 
tale che:
\[
\mathcal{C} \leftarrow X_0 \leq X_1 \wedge \mathcal{C}_0 \wedge \mathcal{C}_1.
\]
\item[-]\files{Constraint lt(Integer k)}: 
crea un nuovo vincolo $\mathcal{C}$ tale che:
\[
\mathcal{C} \leftarrow X_0 < k \wedge \mathcal{C}_0.
\]
\item[-]\files{Constraint lt(IntLVar x1)}: crea un nuovo vincolo $\mathcal{C}$ 
tale che:
\[
\mathcal{C} \leftarrow X_0 < X_1 \wedge \mathcal{C}_0 \wedge \mathcal{C}_1.
\]
\item[-]\files{Constraint ge(Integer k)}: 
crea un nuovo vincolo $\mathcal{C}$ tale che:
\[
\mathcal{C} \leftarrow X_0 \geq k \wedge \mathcal{C}_0.
\]
\item[-]\files{Constraint ge(IntLVar x1)}: crea un nuovo vincolo $\mathcal{C}$ 
tale che:
\[
\mathcal{C} \leftarrow X_0 \geq X_1 \wedge \mathcal{C}_0 \wedge \mathcal{C}_1.
\]
\item[-]\files{Constraint gt(Integer k)}: 
crea un nuovo vincolo $\mathcal{C}$ tale che:
\[
\mathcal{C} \leftarrow X_0 > k \wedge \mathcal{C}_0.
\]
\item[-]\files{Constraint gt(IntLVar x1)}: crea un nuovo vincolo $\mathcal{C}$ 
tale che:
\[
\mathcal{C} \leftarrow X_0 > X_1 \wedge \mathcal{C}_0 \wedge \mathcal{C}_1.
\]
\end{itemize}

Sia \files{vars} un vettore di $n+1$ variabili logiche intere tale che $X_0$
rappresenti \files{vars[0]}, $X_1$ rappresenti \files{vars[1]}, etc.
Il vincolo ``AllDifferent'' è descrivibile come:
\begin{itemize}
\item[-]\files{Constraint AllDifferent(IntLVar[] vars)}: crea un nuovo vincolo 
$\mathcal{C}$ tale che:
\[
\mathcal{C} \leftarrow X_0 \neq X_1 \wedge X_0 \neq X_2 \wedge \cdots \wedge X_0
\neq X_n \wedge X_1 \neq X_2 \wedge \cdots \wedge X_1 \neq X_n \wedge \cdots
\]
è possibile scrivere in modo più compatto il vincolo:
\[
\mathcal{C} \leftarrow \left(\bigwedge_{0 \leq i < j \leq n} X_i \neq X_j\right)
\wedge \bigwedge_{i = 0}^{n}\mathcal{C}_i.
\]
\end{itemize}

\section{Variabili logiche su insiemi di interi}
Le variabili logiche di interesse al fine di implementare la classe 
\files{VarSet} per l'attuale implementazione JSR-331 sono
quelle di insiemi di interi. 

In JSetL le variabili di insiemi di interi sono rappresentate dalla classe
\files{SetLVar}, che estende la classe \files{LVar} ereditandone metodi e 
costruttori.

\begin{flushleft}\textbf{Variabile logica su insiemi di interi.}
Una variabile logica su \emph{insiemi di interi} (più brevemente un 
\emph{insieme di interi}) è un'istanza di
\files{SetLVar}, 
creata dalla seguente chiamata:
\begin{center}
\lstinline[language = Java]!SetLVar varName = new SetLVar(name);!
\end{center}
dove  \files{varName} è il nome della variabile e \files{name} è il nome
opzionale esterno.
\end{flushleft}

\subsection{Dominio}
Il dominio di un \files{SetLVar} è rappresentato da un'istanza di
\files{SetInterval} che di fatto è un reticolo\footnote{Un reticolo è un 
insieme parzialmente ordinato in cui ogni coppia di elementi ha sia un estremo 
inferiore che un estremo superiore.} di insiemi di interi 
(\files{Set<Integer>}).

Il dominio di una variabile di insiemi di interi può essere specificato alla
creazione della variabile, ed è modificato automaticamente quando vengono
risolti i vincoli ad essa associati.

Quando il dominio di una variabile si restringe fino ad essere rappresentato da
un singoletto, la variabile viene detta \emph{bound}, se
invece il dominio si restringe all'insieme vuoto significa che i vincoli ad
essa associati sono insoddisfacibili.

Sia \files{s} un'istanza della classe \files{MultiInterval} che rappresenta un
insieme di interi $S$, la variabile logica che ha come dominio $S$ è creata 
dalla seguente chiamata:
\begin{center}
\lstinline[language = Java]!SetLVar varName = new SetLVar(s);!
\end{center}

\subsection{Metodi utili}
La classe \files{SetLVar} fornisce alcuni metodi utili ereditati dalla classe
\files{LVar}, adattati per supportare i tipi di ritorno 
\files{MultiInterval}, \files{SetLVar} e per i vincoli su insiemi e domini.
\begin{itemize}
\item[-]\files{Constraints getConstraints()}: restituisce la congiunzione di
tutti i vincoli associati alla variabile;
\item[-]\files{MultiInterval getDomain()}: restituisce il
dominio della variabile;
\item[-]\files{Boolean isBound()}: restituisce 
\files{true}  se
il dominio è ristretto ad un solo elemento, \files{false}  in caso contrario.
\end{itemize}

\subsection{Operazioni insiemistiche}
Come nel caso delle variabili intere, è possibile creare nuove
variabili insiemistiche utilizzando metodi che riproducono le comuni operazioni
sugli insiemi. Sia $S_0$ l'insieme rappresentante la variabile d'invocazione
\files{s},
$\mathcal{C}_0$ i vincoli ad essa associati, siano $S_1$, $S_2$ gli insiemi
rappresentanti i parametri \files{SetLVar x} e \files{y}, $\mathcal{C}_1$ e
$\mathcal{C}_2$  i vincoli ad esse associati, si definiscono le seguenti 
operazioni:
\begin{itemize}
\item[-]\files{SetLVar compl()}: 
crea un nuovo insieme di interi tale che, se \files{x = s.compl()}
allora:
\[
S_1 = S_0^c \wedge \neg  \mathcal{C}_0;
\]
\item[-]\files{SetLVar intersect(SetLVar y)}: crea un nuovo insieme di interi
tale che, se \files{x = s.intersect(y)} allora:
\[
S_1 = S_0 \cap S_2 \wedge \mathcal{C}_0 \wedge \mathcal{C}_2;
\]
\item[-]\files{SetLVar union(SetLVar y)}: crea un nuovo insieme di interi
tale che, se \files{x = s.union(y)} allora:
\[
S_1 = S_0 \cup S_2 \wedge \mathcal{C}_0 \vee \mathcal{C}_2;
\]
\item[-]\files{SetLVar diff(SetLVar y)}: crea un nuovo insieme di interi
tale che, se \files{x = s.diff(y)} allora:
\[
S_1 = S_0 \setminus S_2 \wedge \mathcal{C}_0 \wedge \neg \mathcal{C}_2.
\]
\end{itemize}


%% Vincoli
\section{Vincoli}
JSetL fornisce vincoli per specificare condizioni su variabili logiche e 
insiemi. Questi vincoli sono manipolati da un \emph{risolutore di vincoli} 
(un'istanza della classe \files{SolverClass}) che implementa le specifiche
strategie di risoluzione.

Il \emph{dominio dei vincoli} in JSetL è il dominio $\mathcal{SET}$ definito
in \cite{artClp}, esteso con alcuni nuovi vincoli per la comparazione tra 
interi.
\begin{flushleft}\textbf{Vincolo atomico}.\\
Un \emph{vincolo atomico} in JSetL è un'espressione in una delle 
seguenti forme:
\begin{itemize}
\item[-] $e_1.\texttt{op}(e_2)$;
\item[-] $e_1.\texttt{op}(e_2, e_3)$;
\end{itemize}
dove \files{op}  è uno dei metodi predefiniti (\files{eq, neq, in, nin},
\ldots) ed $e_1$, $e_2$, $e_3$ sono espressioni il cui tipo dipende da 
\files{op}.
\end{flushleft}

Il significato di questi metodi è associato in modo naturale al loro stesso 
nome, ovvero \files{eq} e \files{neq} stanno ad indicare l'uguaglianza e la
disuguaglianza rispettivamente (dall'inglese equal e not equal),
\files{in} e \files{nin} per l'appartenenza e la non appartenenza, etc.

\begin{flushleft}\textbf{Vincolo}.\\
Un \emph{vincolo} è un vincolo atomico o (ricorsivamente) un'espressione
del tipo:
\[
c_1.\files{and}(c_2.\files{and}(\cdots c_{n-1}.\files{and}(c_n) \cdots)
\]
\[
\textrm{oppure}
\]
\[
c_1.\files{and}(c_2). \ldots .\files{and}(c_n)
\]
dove $c_1, c_2, \ldots, c_n$ sono vincoli atomici.
\end{flushleft} 
Il significato di $c_1.\files{and}(c_2). \ldots .\files{and}(c_n)$ è la 
congiunzione logica $\mathcal{C}_1 \wedge \mathcal{C}_2 \wedge \cdots 
\wedge \mathcal{C}_n$, in cui i $\mathcal{C}_i$ sono le espressioni logiche
rappresentanti i vincoli atomici $c_i$.

\begin{flushleft}\textbf{Constraint.}\\
Un \emph{Constraint} JSetL è un'istanza della classe \files{Constraint}, creata
dalla seguente chiamata:
\begin{center}
\lstinline[language = Java]
$Constraint constraintName = new Constraint();$
\end{center}
dove \files{constraintName} è il nome del vincolo.
\end{flushleft}

\subsection{Operazioni logiche}
\`E possibile creare vincoli anche partendo da un
vincolo d'invocazione (\files{this}), mediante le comuni operazioni
logiche. Siano  $\mathcal{C}_1$, 
$\mathcal{C}_2$ i vincoli rappresentanti le istanze dei \files{Constraint c1}
e \files{c2}. Si definiscono le seguenti operazioni logiche:
\begin{itemize}
\item[-]\files{Constraint c = c1.and(c2)}: crea un nuovo vincolo $\mathcal{C}$
tale che:
\[
\mathcal{C} \leftarrow \mathcal{C}_1 \wedge \mathcal{C}_2;
\]
\item[-]\files{Constraint c = c1.or(c2)}: crea un nuovo vincolo $\mathcal{C}$
tale che:
\[
\mathcal{C} \leftarrow \mathcal{C}_1 \vee \mathcal{C}_2;
\]
\item[-]\files{Constraint c = c1.orTest(c2)}: crea un nuovo vincolo 
$\mathcal{C}$ tale che:
\[
\mathcal{C} \leftarrow \mathcal{C}_1 \vee \mathcal{C}_2;
\]
\item[-]\files{Constraint c = c1.notTest()}: crea un nuovo vincolo 
$\mathcal{C}$ tale che:
\[
\mathcal{C} \leftarrow \neg \mathcal{C}_1;
\]
\item[-]\files{Constraint c = c1.impliesTest(c2)}: crea un nuovo vincolo 
$\mathcal{C}$
tale che:
\[
\mathcal{C} \leftarrow \mathcal{C}_1 \Rightarrow \mathcal{C}_2.
\]
\end{itemize}
Il vincoli \files{or} viene risolto in modo non deterministico\footnote{Per 
risoluzione non deterministica si intende che il vincolo crea un punto
di scelta copiando il contenuto del constraint store. In pratica aprendo più
computazioni parallele.}, mentre \files{and}  e i vincoli i cui nomi finiscono 
con la
sottostringa ``\files{Test}'' (\files{impliesTest} ad esempio) vengono risolti
in modo determistico\footnote{Viceversa il metodo deterministico non crea
punti di scelta e si limita a propagare il vincolo finchè non viene effettuato 
il labeling sulle variabili è un metodo più efficiente, ma non garantisce
una soluzione senza labeling.}.

%% Soluzione
\section{Meccanismi di risoluzione}
I vincoli creati utilizzando la classe \files{Constraint} o mediante i metodi
propri delle variabili logiche vengono passati ad un solver, un'istanza 
della classe \files{SolverClass}, che li aggiunge al proprio \emph{constraint
store}, il quale contiene tutti i vincoli attivi.

La risoluzione dei vincoli è attuata mediante la chiamata di uno dei metodi
della classe \files{SolverClass} come \files{solve()} o \files{nextSolution()}.

\subsection{Constraint store}
Il \files{constraint store} di un'istanza del solver contiene tutti i 
vincoli attivi per il programma in esecuzione. JSetL fornisce alcuni
metodi con i quali aggiungere nuovi vincoli ad uno specifico constraint store,
visualizzarne il contenuto o rimuovere tutti i vincoli presenti.

Per aggiungere un vincolo \files{c} al constraint store \files{S} è 
sufficiente utilizzare il metodo \files{add} nel seguente modo:
\begin{center}
\files{S.add(c);}
\end{center}
La collezione dei vincoli nello store è considerata come una congiunzione di
vincoli atomici, se $\Gamma$ è il contenuto del constraint store del
solver \files{S}, con la chiamata \files{S.add(c)} si ottiene il vincolo
$\Gamma \wedge \mathcal{C}$, dove $\mathcal{C}$ rappresenta 
\files{c}. 

\begin{ese}
Siano \files{x}, \files{y} e \files{z} variabili logiche, sia \files{solver}
un'istanza della classe \files{SolverClass}.
\begin{lstlisting}[language = Java, frame = single]
    solver.add(x.eq(3));    // x = 3
    solver.add(y.neg(x));   // y != x
    solver.add(x.eq(z));    // x = z
\end{lstlisting}
Il vincolo aggiunto allo store è $x = 3 \wedge y \neq x \wedge x = z$. Il
medesimo risultato si sarebbe potuto ottenere con la seguente chiamata:
\begin{center}
\lstinline$solver.add(x.eq(3).and(y.neq(x)).and(x.eq(z)));$
\end{center}
\end{ese}

\subsection{Risoluzione dei vincoli}
La risoluzione dei vincoli in JSetL è basata sulla riduzione di ogni 
congiunzione di vincoli atomici ad una forma semplificata denominata 
\emph{solved form}.

Senza entrare nel dettaglio, si può affermare che una solved form è una
congiunzione di vincoli irriducibili e che, se non vuota, rappresenta una 
soluzione per i vincoli aggiunti allo store.

Il metodo (inteso come funzione Java) principale per ottenere una soluzione
da un'istanza \files{solver} di \files{SolverClass} è \files{solve}, il quale
lancia un'eccezione se i vincoli non sono soddisfacibili, altrimenti trasforma
il constraint store in modo non deterministico affinchè i vincoli
assumano una  forma semplificata.

\subsection{Labeling}
Come spesso si verifica nei risolutori di vincoli su domini finiti, il
solver di JSetL non può considerarsi \emph{completo} nel momento in cui i
vincoli del constraint store coinvolgono istanze di \files{IntLVar} o
\files{SetLVar}. Per controllare la soddisfacibilità dello store e per
trovare una o tutte le soluzioni, in questo caso, può rendersi
necessario introdurre strategie di ricerca, una di queste è il
\emph{labeling}.

Il labeling si effettua su variabili logiche, alle quali viene cercato di
assegnare un valore del proprio dominio.
Ovviamente considerare ogni possibile valore del dominio risulterebbe molto
gravoso in termini di computazione ed efficienza, sono state introdotte quindi
delle euristiche per ridurre lo spazio di ricerca:
\begin{itemize}
\item[-]\emph{Scelta delle variabili}: determina in quale ordine vengono 
scelte le variabili a cui si cerca di assegnare un valore.
\item[-]\emph{Scelta dei valori}: determina quale valore assegnare alle 
variabili.
\end{itemize}

Di seguito si elencano le possibili euristiche attualmente presenti in JSetL:
\begin{lstlisting}[language = Java,
                   caption = {euristiche dei valori.}]
public enum ValHeuristic {
	GLB,            // da sinistra a destra.
	LUB,            // da destra a sinistra.
	MID_MOST,       // dal centro.
	MEDIAN,         // dal valore medio.
	EQUI_RANDOM,    // metodi casuali.
	RANGE_RANDOM,
	MID_RANDOM
}
\end{lstlisting}
\begin{lstlisting}[language = Java,
                   caption = {euristiche delle variabili.}]
public enum VarHeuristic {
	RIGHT_MOST,   // da destra a sinistra.
	LEFT_MOST,    // da sinistra a destra.
	MID_MOST,     // dal centro.
	MIN,          // dal dominio con valore minimo.
	MAX,          // dal dominio con valore massimo.
	FIRST_FAIL,   // dal dominio minimo.
	RANDOM        // casuale.
}
\end{lstlisting}

Per applicare un'euristica occorre aggiungerla sotto forma di vincolo
all'interno del constraint store. Siano \files{x}, \files{y} variabili logiche,
sia \files{vars} un array di variabili logiche (omogenee),
le sintassi possibili sono le seguenti:
\begin{lstlisting}[caption = {opzioni di labeling.}]
solver.add(x.label());

solver.add(y.label(ValHeuristic.LUB));

LabelingOptions lop = new LabelingOptions();
lop.val = ValHeuristic.MEDIAN;
lop.var = VarHeuristic.MID_MOST;
solver.add(label(vars, lop));
\end{lstlisting}
Come si vede è possibile utilizzare il metodo \files{label()} senza parametri,
che utilizza un'euristica di default, oppure è possibile specificare 
un metodo per le scelte mediante la chiamata con parametro 
\files{label(ValHeuristic)}. L'ultima opzione è quella di utilizzare una 
struttura di supporto chiamata \files{LabelingOptions}, che consente di 
specificare 
entrambe le scelte per un array di variabili (ma è anche disponibile il metodo 
per una variabile singola).
