\documentclass[a4paper, 11pt]{article}
\usepackage[italian]{babel}
\usepackage[utf8]{inputenc}
\usepackage{makeidx}
\usepackage{amsfonts}
\usepackage{amsmath}
\usepackage{amsthm}
\usepackage{amssymb}
\usepackage{graphicx}
\usepackage{verbatim}
\usepackage[symbol]{footmisc}
\usepackage[T1]{fontenc}
\usepackage{listings}
\usepackage[usenames,dvipsnames]{xcolor}
\usepackage{enumerate}
\usepackage{multirow}
\usepackage{algpseudocode}
\usepackage{algorithm}

\usepackage{hyperref}

\makeatletter
\renewcommand*{\ALG@name}{Pseudocodice}
\makeatother

\definecolor{orange}{rgb}{1,0.5,0}

\tolerance=1
\emergencystretch=\maxdimen
\hyphenpenalty=10000
\hbadness=10000

\newcommand{\files}[1]{\texttt{#1}}


\newcommand{\zz}{\mathbb{Z}}
\newcommand{\nn}{\mathbb{N}}
\newcommand{\rr}{\mathbb{R}}

\frenchspacing

\newenvironment{dimo}{\textbf{Dimostrazione.\ }}{\samepage
\begin{flushright}$\blacksquare$\end{flushright}}

\newtheorem{teo}{Teorema}[section]
\newtheorem{prop}[teo]{Proposizione}
\newtheorem{cor}[teo]{Corollario}
\newtheorem{lemma}[teo]{Lemma}


\theoremstyle{definition}
\newtheorem{defi}[teo]{Definizione}
\newtheorem{osse}[teo]{Osservazione}
\newtheorem*{exe}{Esercizio}
\newtheorem*{ese}{Esempio}
\newtheorem*{notabene}{Nota Bene}
\newtheorem*{nota}{Nota}

\theoremstyle{remark}
\newtheorem*{notazione}{Notazione}

% General command to set parameter(s).
\lstset{
 %belowcaptionskip=1\baselineskip,
 basicstyle=\footnotesize\ttfamily,
 keywordstyle=\bfseries\color{Purple},
 commentstyle=\itshape\color{ForestGreen},
 identifierstyle=\color{blue},
 stringstyle=\color{orange},
 showstringspaces=false,
 otherkeywords={},
 %breakindent = 20pt,
 breakautoindent = True,
 postbreak=\space,
 breaklines,
 tabsize = 2,
 language = C,
 %frameround = tttt,
 frame = L %shadowbox
}
%\renewcommand{\lstlistingname}{}

\author{Fabio Biselli -- Mat: 0000734326}
\title{Algoritmi Avanzati -- Relazione di Progetto: ``Skyline''}
\date{}

\begin{document}
\maketitle

\section{Introduzione}
Il progetto proposto riguarda l'implementazione di un algoritmo parallelo
per determinare lo \emph{skyline} di un isieme di punti nello spazio,
su architetture a memoria distribuita utilizzando il linguaggio \texttt{C} e
la libreria \texttt{MPI}. Il processo principale deve leggere le coordinate
dei punti in input da un file di testo \texttt{in.txt} composto da $N + 1$ righe
in cui la prima specifica il numero di punti ($N$), e le successive contengono
le coordinate di ogni punto. Al termine dell'esecuzione l'insieme dello
skyline deve essere salvato con le medesime caratteristiche nel file
\texttt{out.txt}. Si assume che i punti in input appartengano allo spazio
euclideo a tre dimensioni e che il loro numero sia sempre (molto) maggiore
del numero di processi \texttt{MPI}.

\begin{defi}
Sia $P = \left\{ p_0, p_1, \ldots, p_{N-1}\right\}$ un insieme di punti in tre
dimensioni ($p_i = \langle x_i, y_i, z_i\rangle \in \rr^3$). Si dice che $p_i$
\emph{domina} $p_j$, e scriveremo $p_i \succ p_j$,
se si verificano le seguenti condizioni:
\begin{itemize}
\item
$x_i \ge x_j, y_i \ge y_j, z_i \ge z_j$;
\item
Esiste almeno una coordinata per cui la disuguaglianza valga in senso stretto.
\end{itemize}
\end{defi}

\begin{prop}
Transitività della relazione $\succ$.
\[
\forall p, q, r \in P,\quad
p \succ q\ \text{e}\ q \succ r \Rightarrow p \succ r.
\]
\end{prop}

\begin{defi}
Sia $P$ un insieme di punti, si definisce lo \emph{skyline} di $P$ l'insieme
$Sk(P)$ composto da tutti i punti di $P$ che non sono dominati da alcun altro
punto di $P$.
\[
Sk(P) := \left\{
s \in P \colon \neg \exists\ t \in P \colon t \not\succ s
\right\}.
\]
\end{defi}
Si noti che per definizione nessun punto domina se stesso.


\section{Implementazione}
L'implementazione sequenziale (descritta nello pseudocodice \ref{seq})
suggerita per questo algoritmo prevede l'utilizzo
di un array di interi $S$ di lunghezza pari al numero dei punti in input.
Questo viene inizializzato impostando tutti i valori ad $1$. In seguito
vengono confrontati tutti i punti mediante un operatore binario $\succ$
(domina), ogni
punto viene confrontato con tutti gli altri per un totale di $O(N^2)$ chiamate.
Se $p_i \succ p_j$, il relativo valore di $p_j$ in $S$ ($S[j]$) viene impostato
a $0$.
Al termine della procedura l'array $S$ conterrà le posizioni della skyline:
se $S[i] = 1$ allora $p_i \in Sk(P)$.

\begin{algorithm}
\caption{Skyline sequenziale.}
\label{seq}
\begin{algorithmic}
\Function{Skyline}{$x[N]$, $y[N]$, $z[N]$}
\State int $S[N]$;
\For{$i \gets 0$ to $N-1$}
  \State{$S[i] \gets 1$;}
\EndFor

\For{$i \gets 0$ to $N-1$}
   \For{$j \gets 0$ to $N-1$}
      \If{$\langle x[i],y[i],z[i]\rangle \succ \langle x[j],y[j],z[j]\rangle$}
         \State $S[j] \gets 0$;
      \EndIf
   \EndFor
\EndFor
\EndFunction
\end{algorithmic}
\end{algorithm}

La soluzione parallela implementata sfrutta la transitività della relazione
\emph{domina}, in sostanza l'algoritmo è suddiviso in due fasi. Nella
prima fase i dati in input vengono partizionati in base al numero di processi
disponibili ed inviati ai processi client (server compreso),
in ogni partizione è quindi utilizzato l'algoritmo sequenziale proposto.
Ogni processo invia la propria parte di array $S$  (di dimesione fissata)
al processo ``server''.

Nella seconda fase, che riguarda il solo processo server,
vengono creati gli array dei punti ottenuti dalla prima,
a cui vengono eventualmente aggiunti quelli in avanzo dalla procedura di
partizione dei dati in input. Infatti se il numero di processi $p$ non divide
il numero di punti in input $N$ si ottiene un insieme di punti di ``resto'' che
devono essere aggiunti in questa fase. Tale numero è comunque compreso tra $1$
e $p-1$, quindi possiamo assumere che sia irrilevante in termini di complessità
computazionale. A questo punto si utilizza nuovamente l'algoritmo sequenziale
sull'insieme di punti per ottenere la skyline finale.

Si può vedere questo algoritmo come una specie di torneo, in cui si hanno un
numero $p$ di gironi di qualificazione (in cui però non è dato sapere a priori
il numero di vincitori) ed un girone finale in cui partecipano tutti i
qualificati più le ``teste di serie'' (l'eventuale resto).

\section{Analisi}
Il punto di forza di questo algoritmo è dato dalla transitività della relazione
\emph{domina} ($\succ$) che consente di ridurre notevolmente il numero totale di
confronti. Questo approccio funziona molto bene con insiemi di punti densi, in
cui il numero di ``vincitori'' per ogni partizione è basso rispetto al numero
totale, e quindi si comporta molto bene nel caso medio. D'altra parte, nel
caso pessimo (quello in cui tutti i punti in input appartengono alla skyline) si
comporta peggio di quello sequenziale, dovendo aggiungere al tempo sequenziale
il tempo relativo alla prima fase.

\subsection{Scalabilità}
L'algoritmo parallelo proposto ed implementato riduce drasticamente i tempi di
calcolo rispetto alla versione sequenziale proposta. Questo è dovuto al fatto
che il secondo agisce ``a forza bruta'', mentre il primo, oltre a suddividere
il carico della computazione su più processi, sfrutta la transitività di
$\succ$. Utilizzando l'algoritmo \texttt{MPI} implementato con \textbf{N} $= 1$
per l'analisi infatti si ottengono risultati non obiettivi, con \emph{speedup}
ed \emph{efficienza} esageratamente alte. Al fine di generare dati compatibili
si è implementato un algoritmo sequenziale che potesse sfruttare il
partizionamento.
I dati riportati nelle tabelle \ref{t3} e \ref{t4} sono stati ottenuti
confrontando i tempi dell'algoritmo seriale riportati nella tabella \ref{t1},
in cui \textbf{N} rappresenta il numero di partizioni, ed i tempi dell'algoritmo
parallelo riportati nella tabella \ref{t2}, in cui \texttt{comm\_sz} rappresenta
il numero di processi.

La tabella \ref{t3} mostra la \emph{speedup} ottenuta. Si può notare che,
fissato il numero dei processi, questa aumenti all'aumentare dei dati in input
e quindi all'aumentare della dimensione delle partizioni. A prima vista
potrebbe sembrare un comportamento anomalo, in realtà ciò è dovuto all'aumento
della densità delle partizioni. Ovviamente questo comportamento è strettamente
legato alla ``qualità'' dei punti in input, mi aspetto che per insiemi di punti
meno densi il fenomeno sia minore e che comunque, all'aumentare dei dati si
superi la soglia ottimale e lo speedup dimunisca.

Fissando ora il numero di dati input si può notare un comportamento che avvalora
la tesi di cui sopra. In generale all'aumento dei processi corrisponde un
aumento della speedup, ma abbiamo un'eccezione nella prima colonna ($3500$) in
cui lo speedup diminusice utilizzando $8$ processi, in questo caso
probabilmente la soglia ottimale del rapporto ``Numero di punti'' /
\texttt{comm\_sz} è stata superata.

Ulteriori test con \texttt{comm\_sz} $> 8$ sono stati eseguiti, ed in tutti i
casi si è notato una stabilizzazione dei tempi di risposta ed un corrispondente
degrado della speedup, ma questo potrebbe essere dovuto anche alla macchina su
cui sono stati effettuati i test (Intel® Core™ i7-3630QM CPU @
2.40GHz$\times$8).

Infine la tabella \ref{t4} mostra l'efficienza dell'algoritmo, evidenziando
il fatto che questa aumenti con l'aumento del rapporto ``Numero di punti'' /
\texttt{comm\_sz}.

\subsection{Speedup superlineare}
Un dato che balza all'occhio guardando le tabelle è la presenza di alcuni casi
di speedup superlienare e
la relativa efficienza $>1$. Solitamente, dati $p$ processi, la speedup
$S(p)$ è minore o uguale a $p$, il caso ottimo si ha quando $S(p) = p$ e di
conseguenza l'efficienza $E(p) = 1$. In genere questo fenomeno si verifica
conseguentemente all'utilizzo della cache di memoria, tuttavia osservando i dati
sembra che anche la densità dei punti, in questo specifico caso, possa dare
il proprio contributo.

\begin{table}
\centering
\setlength\tabcolsep{4pt}
\begin{minipage}{0.48\textwidth}
\centering
\begin{tabular}{c|cccc}
\multirow{2}{*}{\textbf{N}}
  & \multicolumn{4}{c}{\textbf{Numero di punti}} \\
  & $3500$ & $7500$ & $15000$ & $25000$ \\
\hline
$2$ & $0.11$ & $0.46$ & $1.81$ & $5.00$ \\
$3$ & $0.07$ & $0.31$ & $1.21$ & $3.33$ \\
$4$ & $0.05$ & $0.23$ & $0.91$ & $2.50$ \\
$8$ & $0.03$ & $0.12$ & $0.46$ & $1.25$ \\
\end{tabular}
\caption{Esecuzione seriale}
\label{t1}
\end{minipage}
\hfill
\begin{minipage}{0.48\textwidth}
\centering
\begin{tabular}{c|cccc}
\multirow{2}{*}{\texttt{comm\_sz}}
  & \multicolumn{4}{c}{\textbf{Numero di punti}} \\
  & $3500$ & $7500$ & $15000$ & $25000$ \\
\hline
$2$ & $0.05$ & $0.21$ & $0.82$ & $2.22$ \\
$3$ & $0.03$ & $0.10$ & $0.38$ & $1.04$ \\
$4$ & $0.02$ & $0.06$ & $0.22$ & $0.59$ \\
$8$ & $0.014$ & $0.03$ & $0.10$ & $0.25$ \\
\end{tabular}
\caption{Esecuzione parallela}
\label{t2}
\end{minipage}

\end{table}


\begin{table}
\centering
\setlength\tabcolsep{4pt}
\begin{minipage}{0.48\textwidth}
\centering
\begin{tabular}{c|cccc}
\multirow{2}{*}{\texttt{comm\_sz}}
  & \multicolumn{4}{c}{\textbf{Numero di punti}} \\
  & $3500$ & $7500$ & $15000$ & $25000$ \\
\hline
$2$ & $2.2$ & $2.2$ & $2.2$ & $2.25$ \\
$3$ & $2.3$ & $3.1$ & $3.18$ & $3.20$ \\
$4$ & $2.5$ & $3.83$ & $4.13$ & $4.23$ \\
$8$ & $2.14$ & $4.0$ & $4.6$ & $5.0$ \\
\end{tabular}
\caption{Speedup}
\label{t3}
\end{minipage}
\hfill
\begin{minipage}{0.48\textwidth}
\centering
\begin{tabular}{c|cccc}
\multirow{2}{*}{\texttt{comm\_sz}}
  & \multicolumn{4}{c}{\textbf{Numero di punti}} \\
  & $3500$ & $7500$ & $15000$ & $25000$ \\
\hline
$2$ & $1.10$ & $1.10$ & $1.10$ & $1.12$ \\
$3$ & $0.77$ & $1.03$ & $1.06$ & $1.06$ \\
$4$ & $0.63$ & $0.95$ & $1.03$ & $1.05$ \\
$8$ & $0.26$ & $0.50$ & $0.57$ & $0.62$ \\
\end{tabular}
\caption{Efficienza}
\label{t4}
\end{minipage}
\end{table}



\end{document}
