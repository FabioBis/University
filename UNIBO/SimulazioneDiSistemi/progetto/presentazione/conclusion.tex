%% Slide 130.
\begin{frame}
\frametitle{Conclusioni e lavori futuri}
\begin{itemize}[<+->]
\item
Studiati alcuni articoli sui DVE;
\item
\`E stato creato un nuovo modello sulla base di quello proposto;
\item
Eseguite due simulazioni con $180$ e $360$ client coinvolti;
\item
L'analisi dei risultati ha evidenziato robustezza, non avendo
ottenuto nessuna perdita di movimenti ed una buona efficienza;
\item
In futuro si potrebbe espandere il modello;
\item
Analizzare più approfonditamente il sistema di partizionamento.
\end{itemize}
\end{frame}

%% Slide 31.
\begin{frame}
\frametitle{Riferimenti}
\begin{thebibliography}{}
\bibitem{I}
P.Morillo, J.M.Orduna, M.Fernandez, and J.Duato.
\emph{Improving the Performance of Distributed Virtual Environment
  Systems}.
IEEE TRANSACTIONS ON PARALLEL AND DISTRIBUTED SYSTEMS, 16(7),
  2005.
\bibitem{z} M. Zyda, \emph{From Visual Simulation to Virtual Reality to
Games}, Computer Society, September 2005;
\bibitem{v} S.A. van Houten, P.H.M. Jacobs, \emph{An Architecture for
Distribuited Simulation Games},
Proceedings of the 2004 \emph{Winter Simulation Conference};
\bibitem{g} C. Ghosh, R.P. Wiegand, B. Goldiez, T.Dere, \emph{An
Architecture Supporting Large Scale MMOGs},
Proceedings of the 3rd International ICST Conference on Simulation Tools
and Technique, 2010.
\end{thebibliography}
\end{frame}



\appendix
\begin{frame}[fragile]
\begin{lstlisting}[language = python]
BATCH = 180

def main(args):
  x_bar = []

  # Compute mean
  for current in range(1, BATCH + 1):
    responseFile = 'responses-' + str(current) + '.csv'
    respReader = csv.reader(open('../data/' + responseFile), delimiter = ',')
    x_bar_i = 0
    n = 0
    for row in respReader:
      if row[0] != 'time':
        if float(row[0]) > 20.0 and float(row[0]) < 180.0:
          x_bar_i += float(row[1])
          n += 1
    x_bar.append(x_bar_i / n)
  mu = 0
  for x_i in x_bar:
    mu += x_i / BATCH
  print('Stima puntuale della media: ' + str(mu))

  sigma2 = 0
  for x_i in x_bar:
    sigma2 += (x_i - mu)**2
  sigma2 /= (BATCH - 1)
  print('Varianza campionaria: ' + str(sigma2))

  a = mu - 1.96 * (sqrt(sigma2) / sqrt(BATCH))
  b = mu + 1.96 * (sqrt(sigma2) / sqrt(BATCH))
  print('Intervallo di confidenza: (' + str(a) + ', ' + str(b) + ')')
\end{lstlisting}
\end{frame}
