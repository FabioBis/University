\subsection*{Studi preliminari}

%% Slide 1.
\begin{frame}
\frametitle{Studi preliminari}
\begin{thebibliography}{}
\bibitem{zyda} M. Zyda, \emph{From Visual Simulation to Virtual Reality to
Games}, IEE Computer Society, September 2005;
\bibitem{van} S.A. van Houten, P.H.M. Jacobs, \emph{An Architecture for
Distribuited Simulation Games},
Proceedings of the 2004 \emph{Winter Simulation Conference};
\bibitem{ghosh} C. Ghosh, R.P. Wiegand, B. Goldiez, T.Dere, \emph{An
Architecture Supporting Large Scale MMOGs},
Proceedings of the 3rd International ICST Conference on Simulation Tools
and Technique, 2010.
\end{thebibliography}
\end{frame}

%% Slide 2.
\begin{frame}
\frametitle{Gli articoli in sintesi}
\begin{itemize}%[<+->]
\item
Creare una scienza dei giochi ed una ``Research Agenda'';
\item
La tecnologia è considerata la ``spinta''  che guida lo sviluppo e
l'implementazione di un'architettura per un DVE;
\item
I requisiti per l'architettura le 3 U: \alert{uselfulness}, \alert{usability} e
\alert{usage};
\item
Proposta di un'architettura di giochi distribuiti;
\item
Proposta di un'architettura di giochi distribuiti su larga scala basata
sul concetto di ``overlapping zone''.
\end{itemize}
\end{frame}


\subsection*{Prestazioni dei DVE} %% Articolo P.Morillo et all

%% Slide 3.
\begin{frame}
\frametitle{Improving the Performance of DVE Systems}
\begin{thebibliography}{}
\bibitem{IDVE}
P.Morillo, J.M.Orduna, M.Fernandez, and J.Duato.
\newblock {Improving the Performance of Distributed Virtual Environment
  Systems}.
\newblock {\em IEEE TRANSACTIONS ON PARALLEL AND DISTRIBUTED SYSTEMS}, 16(7),
  2005.
\end{thebibliography}
\end{frame}

%% Slide 4.
\begin{frame}
\frametitle{Improving the Performance of DVE Systems}
\begin{itemize}[<+->]
\item
\emph{Data Model:} descrive alcuni metodi per distribuire dati persistenti o
semipersistenti in un DVE;
\item
\emph{Communication Model:} analizza i metodi con cui gli avatar comunicano
tra di loro;
\item
\emph{View Consistency:} mira ad assicurare che ogni avatar che condividono
un'area comune abbia la medesima percezione degli oggetti presenti;
\item
\emph{Network Traffic Reduction:} mantenere un basso numero di messaggi
permette ai sistemi DVE di scalare in modo efficiente con il numero
degli avatar connessi.
\end{itemize}
\end{frame}

%% Slide 5.
\begin{frame}
\frametitle{Risultati di P.Morillo et al.}
\begin{block}{Intenti}
Valutazione del \alert{Partitioning Problem} che è un punto chiave per il
design di DVE scalabili. L'obiettivo è individuare un modo efficiente per
assegnare a più server la gestione degli avatar.
\end{block}
\pause
\begin{block}{Risultati}
\begin{itemize}
\item
Assenza di correlazione ed un comportamento non-lineare in relazione
al numero degli avatar della \alert{funzione di qualità} proposta in
letteratura;
\item
con nuovo metodo di partizione, basato sul bilanciamento del carico dei server,
è possibile mantenere il carico al di sotto della soglia in cui
le prestazioni medie del DVE degradano velocemente.
\end{itemize}
\end{block}
\end{frame}

%% Slide 6.
\begin{frame}
\frametitle{Obiettivi del progetto}
\begin{itemize}%[<+->]
\item
P.Morillo et al. propongono un modello di DVE producendo alcuni risultati
importanti;
\item
grazie a questi è si propone un nuovo modello in cui si trascura il problema
del carico dei server;
\item
si focalizza l'attenzione sul traffico di rete e la comunicazione;
\item
c'è una relazione tra numero di utenti e le prestazioni?
\end{itemize}
\end{frame}
