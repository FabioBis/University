\section{Esercitazione VII.}

\begin{enumerate}

\item 
Date le coppie di valori $(x_i,y_i)$, $i=0,\ldots,n$, ottenuti tabulando a 
passo costante le funzioni $y(x) = \sin(2 \pi x)$ e $y = \frac{1}{1 + 
25(2x -1)^2}$ sull'intervallo $[0,1]$, costruire le seguenti funzioni lineari:
\[
s(x) = y_i + m_i(x-x_i), \quad m_i = \frac{y_{i+1}-y_i}{x_{i+1}-x_i}, \quad
x \in [x_i,x_{i+1}], \ i = 0, \ldots, n-1
\]
e il polinomio interpolatore di Lagrange $p(x)$ di grado $n$.

Indicato con $x_i^* = \frac{x_i + x_{i+1}}{2}$ il punto medio di ciascun 
sottointervallo, tabulare l'errore assoluto:
\[
e_i^s = y(x_i^*) - s(x_i^*), \quad e_i^p = y(x_i^*) - p(x_i^*), \qquad
i = 0, \ldots, n-1.
\] 

Successivamente fare il grafico delle funzioni $s(x)$, $p(x)$, $y(x)$ 
sull'intervallo $[0,1]$.

\begin{svol}

\end{svol}


\end{enumerate}
