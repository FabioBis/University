\chapter{Errori}
\section{Introduzione}
Si consideri il problema dello studio integrale. Sia $f \in C^0([a,b])$, 
allora:
\[\int_a^bf(x)dx = F(b) - F(a).\]
Ma se non si conosce la primitva $F$?
\begin{flushleft}
$\underbrace{\sum_{i = 0}^n w_if(x_i)}_{calcolo} + \underbrace{R_n(f)}_{errore}$.
\end{flushleft}

Si ha in genere un problema reale ed occorre quindi trasformarlo in un
problema matematico tramite la fase di modellizzazione (semplifcazione) del
problema. A questo punto dal problema matematico (ben posto?) si passa al 
problema numerico mediante la fase di discretizzazione, qui verrà introdotto 
un \emph{errore analitico}. Infine si giunge alla soluzione 
\emph{approssimata} con algoritmi dell'Analisi Numerica utilizzando il 
calcolatore, anche in questa fase vengono introdotti errori di calcolo dovuti
alla finitezza della precisione di rappresentazione.

\section{Errore analitico o di discretizzazione}
Sia $f \in C^0([a,b])$, vorrei valutare $f'$ in un punto $x^*$ senza
l'espressione della derivata $f'(x)$.

Dallo sviluppo di Taylor di $f$ centrato in $x^*$ relativo all'incremento $h$
($ 0 < h \leq 1$) si ha:
\[
f(x^* + h) = f(x^*) + hf'(x^*) + \frac{h^2}{2}f''(\psi x^*), \qquad x^* < 
\psi x^* < x^* +h. 
\]
Si ottiene quindi:
\[f'(x^*) = 
\underbrace{\frac{f(x^* + h) - f(x^*)}{h}}_{\textrm{calcolabile}}  - 
\underbrace{\frac{h}{2}f''(\psi x^*)}_{\textrm{non calcolabile}}.
\]

Tramite la discretizzazione si può approssimare quindi:
\[f'(x^*) \simeq \frac{f(x^* + h) - f(x^*)}{h} .\]
Costo dell'algoritmo.
\begin{itemize}
\item[]Valutazioni di $f$: 2.
\item[]Somme algebriche: 2.
\item[]Divisioni: 1.
\end{itemize}
\[E_I(f, x^*, h) := - \frac{h}{2}f''(\psi x^*) = \textrm{O}(h).\]


Sempre dallo sviluppo di Taylor di $f$ si ottiene:
\[
f(x^* + h) = f(x^*) + hf'(x^*) + \frac{h^2}{2}f''(x^*) + 
\frac{h^3}{6}f'''(\eta x^*), \qquad x^* < \eta x^*
< x^* +h. 
\]
\[
f(x^* - h) = f(x^*) - hf'(x^*) + \frac{h^2}{2}f''(x^*) - 
\frac{h^3}{6}f'''(\eta x^*), \qquad x^*- h < \eta x^*
< x^*. 
\]

\[
\begin{array}{lcl}
f(x^* + h) - f(x^* - h) & = & 2hf'(x^*) + \frac{h^3}{6}\left(f'''(\eta x^*) +
f'''(\eta x^*)\right) \\
& = & 2hf'(x^*) + \frac{h^3}{6}\left(2f'''(\eta x^*)\right).
\end{array}
\]

\[
f'(x^*) = \frac{f(x^* + h) - f(x^* - h) -\frac{h^3}{3}f'''(\eta x^*)}{2h}.
\]

\[
f'(x^*) = \underbrace{\frac{f(x^* + h) - f(x^* - h)}{2h}}_{\textrm{calcolabile}} -
\underbrace{\frac{h^3}{6}f'''(\eta x^*)}_{\textrm{non calcolabile}}
\qquad x^*- h < \eta x^* < x^*+h. 
\]

Tramite la discretizzazione si può approssimare quindi:
\[f'(x^*) \simeq \frac{f(x^* + h) - f(x^*)}{2h} .\]
Costo dell'algoritmo.
\begin{itemize}
\item[]Valutazioni di $f$: 2.
\item[]Somme algebriche: 3.
\item[]Divisioni: 1.
\end{itemize}
\[E_{II}(f, x^*, h) := - \frac{h^2}{6}f'''(\eta x^*) = \textrm{O}(h^2).\]

Il costo di questo algoritmo è all'incirca uguale al primo, la differenza
tra i due è l'errore analitico. Il primo ha un errore dell'ordine di $10^{-2}$
mentre il secondo di $10^{-4}$. Di contro il primo algoritmo ha meno ipotesi
su $f$.

\subsection{Problema Test o campione.}
Un Test è un problema di cui si conosce la soluzione in forma chiusa, che
viene utilizzato per testare le proprietà matematiche scritte sulla carta.

\begin{exe}
\begin{flushleft}
$f(x) = x^2, \qquad x^* = 1, \qquad f'(1) = 2.$
\end{flushleft}
\begin{flushleft}
$E_I = (f, x^*, h) = - \frac{h}{2}f''(\psi x^*) = - \frac{h}{2}2 = -h.$
\end{flushleft}
\begin{flushleft}
$E_{II}(f, x^*, h) = - \frac{h^2}{6}f'''(\eta x^*)
= \frac{h^2}{6}0 = 0.$
\end{flushleft}
\end{exe}

%%%---- Grafico 1

Meglio la rappresentazione in scala logaritmica.

%%%---- Grafico 2

\begin{ese}
Calcolo numerico del limite: 
\[\lim_{x \to +\infty}[x(\sqrt{x^2+1}-x)].\]


%%% ---- codice e grafico

\[
\lim_{x \to +\infty}[x(\sqrt{x^2+1}-x)] \cdot 
\frac{\sqrt{x^2+1}+x}{\sqrt{x^2+1}+x}
= \lim_{x \to +\infty}\frac{x (\cancel{x^2} + 1 \cancel{- x^2})}{\sqrt{x^2+x}+x}
= \frac{1}{2}.
\]
\end{ese}


\section{Errori di rappresentazione.}
\subsection{Rappresentazione in virgola mobile.}
Si consideri la rappresentazione dei numeri reali nel sistema decimale:
\[
\begin{array}{lcl}
x & = & \underbrace{327.52}_{\textrm{mantissa}}\\
  & = & 3\cdot 10^2 + 2\cdot 10^1 + 7\cdot 10^0 + 5\cdot 10^{-1} + 2\cdot 
10^{-2}.
\end{array}
\]
Oppure una rappresentazione differente, sempre nel sistema decimale:
\[
\begin{array}{lcl}
x & = & \underbrace{0.32752}_{\textrm{nuova mantissa}}\\
  & = & 3\cdot 10^{-1} + 2\cdot 10^{-2} + 7\cdot 10^{-3} + 5\cdot 10^{-4} + 2
\cdot 
10^{-5}.
\end{array}
\]
\[
\begin{array}{lcl}
x & = & 00.32752 \\
  & = & 3\cdot 10^{-2} + 2\cdot 10^{-3} + 7\cdot 10^{-4} + 5\cdot 10^{-5} + 2
\cdot 
10^{-6}.
\end{array}
\]

Possiamo fissare una rappresentazione univoca di $x$ con una convenzione:
\[x = \textrm{sgn}(x)\cdot m \cdot 10^p,\]
con $m$ mantissa ($0.1 \leq m < 1$) e $p$ peso.
\begin{exe}
$x = 327.52 \quad m = 0.32752, \quad p = 3.$
\end{exe}

\begin{teo}
Sia $x \in \rr\setminus\{0\}$ e $\beta \in \nn, \, \beta \geq 2.$ Allora
$\exists ! \ p \in \zz$ e $\left\{d_i\right\}_{i = 1}^{\infty}, \ d_i \in \nn$
tale che:
\[
\begin{array}{lcl}
x & = & \mathrm{sgn}(x) \cdot \sum_{i = 1}^{\infty}d_i \beta^{-i} \cdot \beta^p \\
  & = & \mp(.d_1d_2\cdots)\cdot \beta^p.
\end{array}
\]
Con $ 0 \leq d_i < \beta$, $d_1 \neq 0$e $d_i \neq \beta -1$.
\end{teo}

\begin{defi}
Rappresentazione in virgola mobile normalizzata.
\begin{itemize}
\item[-]$\beta$: base.
\item[-]$p$: caratteristica.
\item[-]$d_i$: cifra.
\item[-]$m := \sum_{i = 1}^{\infty}d_i\beta^{-i}$: mantissa.
\end{itemize}
\end{defi}

\subsection{Algoritmi per il cambio di base.}
Sia $x$ un numero intero.
\begin{enumerate}
\item $i = 1 \quad c_i = x$.
\item $d_i = \mathrm{mod}(c_i, \beta)$.
\item $c_{i+1} = \mathrm{int}(\frac{c_i}{\beta})$.
\item se $c_{i+1} = 0$ stop.
\item $i = i+1$ e torna al punto $2$.
\end{enumerate}
Sia $x$ tale che $0 < x < 1$.
\begin{enumerate}
\item $i = 1 \quad c_i = x$.
\item $d_i = \mathrm{int}(c_i \beta)$.
\item $c_{i+1} = c_i\beta - d_i$.
\item se $c_{i+1} = 0$ stop.
\item $i = i+1$ e torna al punto $2$.
\end{enumerate}

\begin{ese}
$x = (0.1)_{10}$, effettuare il cambio di base con $\beta = 2$ e $\beta = 16$.
\begin{flushleft}
$\beta = 2.$\\
$c_1 = 0.1$\\
$d_1 = \mathrm{int}(c_i \beta)$
\end{flushleft}
\[
\begin{array}{c|ccccccr}
i        & 1   & 2   & 3   & 4   & 5   & 6   & \ldots \\
\hline
c_i      & 0.1 & 0.2 & 0.4 & 0.8 & 0.6 & 0.2 & \ldots \\
c_i\beta & 0.2 & \underline{0.4} & 0.8 & 1.6 & 1.2 & \underline{0.4} & 
\ldots \\
d_i      & 0   & 0   & 0   & 1   & 1   & 0   & \ldots
\end{array}
\]
$(.0\overline{0011}0\cdots)_2$.

Come si può notare $d_1 = 0$, quindi per ricondurci alla nostra convenzione
occorre moltiplicare tale mantissa per $2^3$ ovvero aumenttarne il peso.

\begin{osse}
Si può notare che in base $10$ il numero in questo esempio ha una 
rappresentazione esatta, mentre passando alla base binaria occorrerà troncare
la mantissa (poiché periodica), in una certo punto. Questo introduce un errore
di rappresentazione.
\end{osse}

\begin{flushleft}
$\beta = 16.$\\
$c_1 = 0.1$\\
$d_1 = \mathrm{int}(c_i \beta)$
\end{flushleft}
\[
\begin{array}{c|cccr}
i        & 1   & 2   & 3   & \ldots \\
\hline
c_i      & 0.1 & 0.6 & 0.6 &  \ldots \\
c_i\beta & 1.6 & \underline{9.6} & \underline{9.6} & 
\ldots \\
d_i      & 1   & 9   & 9   & \ldots
\end{array}
\]
$(.1999\overline{99}\cdots)_{16}$.
\begin{flushleft}
Oppure dato che $16 = 2^4$ si ha che:\\
$(.\underbrace{0001}_{1\cdot 2^0}\overline{\underbrace{1001}_{1\cdot 2^0 + 1\cdot 2^3}
\underbrace{1001}_{1\cdot 2^0 + 1\cdot 2^3}}\ \cdots)_2.$
\end{flushleft}
\end{ese}

\begin{ese}
$x = (026)_{10}$, effettuare il cambio di base con $\beta = 2$ e $\beta = 16$.
\[
\begin{array}{cccr|r|l}
26 : 2 & = & 13 & \textrm{Resto: }\ 0 & 0 &\\
13 : 2 & = & 6 & \textrm{Resto: }\ 1 & 1 &\\
6 : 2 & = & 3 & \textrm{Resto: }\ 0 & 0 & \\
3 : 2 & = & 1 & \textrm{Resto: }\ 1 & 1 &\\
1 : 2 & = & 0 & \textrm{Resto: }\ 1 & 1 & \uparrow \textrm{dall'ultimo al 
primo}
\end{array}
\]
$(11010)_2$.

\[
\begin{array}{cccr|r|l}
26 : 16 & = & 1 & \textrm{Resto: }\ 10 & A &\\
1 : 16 & = & 0 & \textrm{Resto: }\ 1 & 1 & \uparrow \textrm{dall'ultimo al 
primo}
\end{array}
\]
$(1A)_{16}$.
\begin{flushleft}
Oppure dato che $16 = 2^4$ si ha che:\\
$(1\underbrace{1010}_{(A)_{16}})_2$
\end{flushleft}
\end{ese}

\subsection{Sistema Floating Point.}

\begin{defi}(Sistema Floating Point)

Siano $\beta$, $t$, $m$ ed $M$ interi positivi tale che:
$\beta \geq 2, \quad t \geq 1, \quad m,n > 0.$ L'insieme dei numeri macchina
in base $\beta$ con $t$ cifre significative è l'insieme:
\[F(\beta, t, m, M) = \{0\} \cup \left\{x \in \rr \colon x = \textrm{sgn}(x)
\cdot \sum_{i = 1}^{t}d_i\beta^{-i}\cdot \beta^p\right\},\]
tale che:
\begin{itemize}
\item[-]$0 \leq d_i < \beta, \qquad i = 1, \ldots, t.$
\item[-]$d_1 \neq 0.$
\item[-]$-m \leq p \leq M$.
\end{itemize}
\end{defi}
