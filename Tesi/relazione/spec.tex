\chapter{Il Vincolo di Cardinalità}\label{cardinality}

\section{Preliminari sul vincolo di cardinalità}
L'interfaccia \files{Problem} della specifica JSR-331 introduce dei metodi
convenzionali per la creazione di vincoli che abbiano a che fare con la
cardinalità di certi valori in un
array di variabili vincolate. Questi vincoli contano
quanto spesso certi valori sono assegnati alle variabili nell'array dato.
La variabile di cardinalità risulta quindi vincolata al numero 
di quelle variabili dell'array che siano a loro volta
vincolate da uno specifico valore. 

Vediamo quindi un metodo della classe \files{Problem} che genera un vincolo
di cardinalità:

\begin{center}
\lstinline[language = Java]{Constraint postCardinality(Var[] vars, int cardValue, String oper, int value)}
\end{center}

Questo metodo crea, inserisce nel solver e restituisce un nuovo vincolo di
cardinalità tale che 
``Card(\texttt{vars}, \texttt{cardValue}) \texttt{oper}  \texttt{value}''.

Card(\texttt{vars}, \texttt{cardValue}) denota una variabile vincolata ad essere
uguale al numero di tutti quegli elementi nell'array \texttt{vars}  che sono
a loro volta uguali a \texttt{cardValue}.

Per esempio se la stringa \texttt{oper} rappresenta il simbolo ``$<$'' ciò 
significa che la 
variabile Card(\texttt{vars}, \texttt{cardValue}) deve essere strettamente 
minore
di \texttt{value}.

\begin{defi}\label{insCard}
Sia $V$ la famiglia indiciata da $I_V$ delle variabili vincolate 
appartenenti all'array \texttt{vars}, $n \in \nn$ la dimensione dell'array 
\texttt{vars},
$a \in \zz$ il valore corrispondente alla variabile intera \texttt{cardValue}, 
si definisce la famiglia $S$ tale che:
\[ \forall i \in I_V, \quad
\forall v_i \in V, \quad
v_i \in S \Leftrightarrow v_i = a.
\] 
\end{defi}

Si noti che $I_V = \{0, 1, \ldots, n-1\}$ e che la 
cardinalità
di $I_V$ è esattamente $n$. In questo modo si può mappare ogni
elemento dell'array \texttt{vars} con un elemento della famiglia indiciata 
$V$ mediante i rispettivi indici, che coincidono.

\begin{defi}
Sia $S$ la famiglia definita in \ref{insCard} ed indiciata da $I_S$, $\odot$ 
l'operatore descritto dalla stringa
\texttt{oper}  e $b \in \zz$ il valore corrispondente alla variabile intera
\texttt{value}, si definisce il \emph{vincolo di cardinalità} $\mathcal{C}$
tale che:
\[
\mathcal{C} \leftarrow |I_S| \odot b.
\]
\end{defi}
Se $n$ è la lunghezza del vettore \texttt{vars}, allora
il vincolo $\mathcal{C}$ appena definito si può anche esplicitare nel seguente
modo:
\begin{equation}\label{eqCard1}
\mathcal{C} \leftarrow
\left(v_0 = a \Leftrightarrow v_0 \in S  \right) \wedge \cdots 
\wedge \left(v_{n-1} = a \Leftrightarrow v_{n-1} \in S\right) 
\wedge |I_S| \odot b.
\end{equation}

\section{Una prima implementazione del vincolo di cardinalità}

Possiamo utilizzare nell'implementazione
il concetto di insieme di indici, sfruttando la classe 
\files{VarSet} che modella proprio insiemi di interi, restringendone il
dominio ai soli valori non negativi.

Se $n$ è la lunghezza del vettore \texttt{vars}, allora
il vincolo $\mathcal{C}$ può quindi essere riscritto come segue:
\begin{equation}\label{eqCard2}
\mathcal{C} \leftarrow
\left( v_0 = a \Leftrightarrow 0 \in I_S \right) \wedge \cdots 
\wedge \left(v_{n-1} = a \Leftrightarrow n-1 \in I \right) \wedge |I_S| \odot b.
\end{equation}

Vediamo la costruzione del vincolo $\mathcal{C}$ nell'implementazione
JSR-331 basata su JSetL. Inizialmente costruiamo l'insieme degli indici
di dominio $[0, n-1]$ ed il vincolo vuoto.
\begin{lstlisting}[language=Java,
                   caption = {postCardinality()},
                   frame = single]
// S rappresenta l'insieme degli indici.
SetLVar S = new SetLVar("_Card" + counterSet++, new MultiInterval(), new MultiInterval(0, vars.length-1));
// Il vincolo C.
JSetL.Constraint cardinality = new JSetL.Constraint();
\end{lstlisting}

Passiamo quindi ad inizializzare \texttt{cardinality}  con il vincolo sulla 
cardinalità dell'insieme degli indici, la funzione 
\texttt{getOperator(oper)}  restituisce un codice relativo all'operatore 
opportuno
e mediante lo statement
\texttt{switch}  viene richiamata la relativa funzione ed aggiornato il vincolo
$\mathcal{C}$.
\begin{lstlisting}[language=Java,
                   caption = {postCardinality()},
                   frame = single]
switch(getOperator(oper)) {
  case 1: {
    // Case = "equals". 
    cardinality.and(S.card().eq(value));
    } break;
  case 2: {
    // Case != "not equals".
    cardinality.and(S.card().neq(value));
    }
  case 3: {
    // Case < "less".
    cardinality.and(S.card().lt(value));
    } 
  case 4: {
    // Case <= "less equals".
    cardinality.and(S.card().le(value));
    } 
  case 5: {
    // Case > "greater".
    cardinality.and(S.card().gt(value));
    } 
  case 6: {
    // Case >= "greater equals".
    cardinality.and(S.card().ge(value));
    } 
  default: throw new UnsupportedOperationException();
}
\end{lstlisting}

A questo punto per ogni variabile del vettore dato \texttt{vars}  viene
aggiornato il vincolo con la congiunzione delle due implicazioni 
$v_i = a \Rightarrow i \in I_S$ e $i \in I_S \Rightarrow v_i = a$.
\begin{lstlisting}[language=Java,
                   caption = {postCardinality()},
                   frame = single]
for (int i = 0; i < vars.length; i++) {
  SetLVar X = new SetLVar("_X"+i, new MultiInterval(i,i));
  IntLVar tmp = (IntLVar) vars[i].getImpl();
  // C <- C e v_i = a --> i in S. 
  cardinality.and(tmp.eq(cardValue).impliesTest(X.subset(S)));
  // C <- C e i in S --> v_i = a.
  cardinality.and(X.subset(S).impliesTest(tmp.eq(cardValue)));
  }
\end{lstlisting}

Al termine del ciclo \texttt{cardinality}  rappresenta il vincolo
$\mathcal{C}$ definito da (\ref{eqCard2}).

\subsection*{Esempio di utilizzo}
Vediamo un semplice esempio di come si può utilizzare il vincolo di
cardinalità. Costruiamo cinque variabili intere $v_0, v_1, v_2, v_3$ e $v_4$ di 
dominio $[0, 4]$ e costruiamo un vincolo di cardinalità tale che il numero
delle variabili $v_i$ che assumono il valore $2$ sia esattamente $3$.

Ecco la definizione del problema:
\begin{lstlisting}[language=Java,
                   caption = {esempio di utilizzo di postCardinality},
                   frame = single]
int n = 5;
// Dichiarazione e inizializzazione delle 5 variabili.
Var[] vars = new Var[n];
for (int i = 0; i < vars.length; i++)
  vars[i] = p.variable("v" + i, 0, n - 1);

// Aggiunta nel problema p del vincolo di cardinalita' 
// sulle variabili.
p.postCardinality(vars, 2, "=", 3);
\end{lstlisting}

Chiamando in seguito la funzione \texttt{findSolution()}  ecco l'output di 
questo semplice esempio:
\begin{verbatim}
JSetL - 2.2
Solution #0:
	 v0[0] v1[0] v2[2] v3[2] v4[2]
\end{verbatim}

\section{Una soluzione più efficiente}
La soluzione sopra descritta, progettata ed applicata in prima istanza durante 
la fase d'implementazione, sebbene abbastanza elegante e dichiarativa, si è
poi rivelata inefficiente nella fase di testing. 
Probabilmente i problemi di
efficienza sono dovuti all'utilizzo eccessivo delle variabili insiemistiche
per chiamate multiple ai metodi \texttt{postCardinality} o l'istanziazione
di più classi \texttt{Cardinality} e \texttt{GlobalCardinality}.
Si è quindi deciso di adottare un sistema alternativo che sfrutta un nuovo
vincolo definito da utente con il solver JSetL: \texttt{Occurrence}.

\subsection{Il vincolo \texttt{Occurrence}}\label{occurrence}
Il vincolo \texttt{Occurrence} è stato creato come vincolo JSetL definito da 
utente, aggiunto quindi ad un package della libreria per supportare lo standard
JSR-331. Come ogni vincolo utente di JSetL viene definito nel seguente modo:
\begin{lstlisting}[language = Java, frame = single]
public class Occurence extends NewConstraintsClass {
\end{lstlisting}

La funzione fornita dal vincolo, chiamata \texttt{occurrence}, definisce la
semantica dello stesso: data una lista di variabili logiche intere \texttt{l}, 
due variabili \texttt{v} e \texttt{n}, vincola la lista \texttt{l} ad avere
esattamente  \texttt{n} elementi che assumono il valore \texttt{v}.
%\begin{lstlisting}[language=Java,
%                   caption = {metodo \texttt{occurrence}},
%                   frame = single]
%// True if the list l contains exactly N elements with value V
%public Constraint occurrence(Vector<IntLVar> l, IntLVar v, IntLVar n) { 
%         return new Constraint("occurrence", l, v, n); 
%}
%\end{lstlisting}


All'interno dell'implementazione della classe \texttt{Cardinality} il
questo metodo   viene utilizzato con l'ausilio di una nuova
variabile intera (libera) nel parametro \texttt{n}. Di fatto questa variabile
rappresenta la cardinalità dell'insieme $I_S$ definito in precedenza.
Questa variabile viene poi vincolata al valore richiesto mediante
l'operatore definito.

\begin{lstlisting}[language=Java,
                   caption = {\texttt{Occurrence} usato in 
\texttt{cardinality}}]
        SolverClass solver = ((Solver) p.getSolver()).getSolverClass();
        Occurrence listOps = new Occurrence(solver);
        Vector<IntLVar> varsList = new Vector<IntLVar>();
        for (int i = 0; i < vars.length; i++)
                varsList.add((IntLVar) vars[i].getImpl());
        cardinality.and(listOps.occurrence(varsList, v, k));
        switch(p.getOperator(oper)) {
        case 1: {
                // Case = "equals". 
                cardinality.and(k.eq(value));
        .
        .
        .
\end{lstlisting}
Inizialmente viene caricato il solver per poter inizializzare il vincolo
utente JSetL. Quindi gli oggetti implementativi (\texttt{IntLVar}) degli
elementi dell'array \texttt{vars} vengono inseriti in un \texttt{Vector}. A
questo punto si utilizza \texttt{occurrence} per vincolare \texttt{k} elementi
di \texttt{vars} ad essere uguali a \texttt{v}.

Infine, mediante il costrutto \texttt{switch}, a seconda della relazione
richiesta si vincola \texttt{k} per ottenere il risultato voluto:
\[
|I_S| \odot b.
\]
