\hyphenation{
Jacob
Feldman
}

\chapter{Conclusioni e lavori futuri}\label{conclusioni}

Il lavoro di tesi descritto verte sull'implementazione delle
specifiche individuate dal JSR-331 mediante il pacchetto JSetL.
Il modello di sviluppo utilizzato per il progetto può essere catalogato come
\emph{in cascata} in cui si sono eseguite più iterazioni delle seguenti fasi:
\begin{enumerate}[i]
\item \label{i}Studio del documento di specifica \cite{specifiche}.
\item \label{ii}Valutazione delle funzionalità offerte da JSetL.
\item \label{iii}Codifica.
\item \label{iv}Test e validazione.
\end{enumerate}
Prima di passare alla descrizione delle fasi indicate occorre sottolineare
che tutto il lavoro non si sarebbe potuto svolgere senza l'appoggio del
Dr. Jacob Feldman, il quale, oltre ad un contatto diretto e grande 
disponibilità, ha fornito l'accesso al repository del  progetto e uno spazio
dedicato all'implementazione JSetL. Dal repository, mediante \emph{SVN}, si
è potuto scaricare tutto il materiale necessario per iniziare l'implementazione,
inoltre il repository
è stato uno strumento fondamentale per mantenere il progetto aggiornato.

Per quanto riguarda il punto \ref{i} si è trattato di analizzare il documento di
specifica fornito. Sono stati molto utili, al fine di comprendere meglio i 
requisiti dei vari metodi e delle classi, anche gli esempi presenti nel TCK
(vedi cap. \ref{capJSR}).

La valutazione di JSetL (punto \ref{ii}) è stata facilitata dai numerevoli 
documenti
(articoli, tesi, \ldots) presenti al Dipartimento di Matematica, nonché dal
contatto diretto con persone che hanno lavorato attivamente al 
progetto, su tutti il prof. Gianfranco Rossi, promotore del progetto e 
sempre disponibile.

La fase di codifica ha portato alla realizzazione vera e propria delle classi
descritte nei capitoli precedenti. Indubbiamente lo sforzo maggiore è
stato quello di mappare i requisiti di input e di output richiesti sulle 
funzionalità di JSetL in modo corretto ed efficace. In questa fase, oltre
alla codifica delle classi, si sono realizzati anche i test di unità, ovvero
delle prove d'esecuzione (white box) scritte appositamente per verifiche 
interne.

L'ultima fase riguarda il TCK che di fatto rappresenta l'unico test di
conformità alle specifiche. Come accennato nel capitolo \ref{capJSR} il TCK è
fornito direttamente dalle specifiche JSR-331 ed un'implementazione, per
essere conforme allo standard, deve permettere a questi test di terminare con 
successo. Nelle varie iterazioni del processo di sviluppo si è utilizzato
questo pacchetto per validare le unità introdotte e per poter quindi
passare alla realizzazione del modulo successivo o rivalutare quello non
soddisfacente. Oltre al TCK si è sviluppato un ulteriore pacchetto di test
per verificare l'implementazione prodotta allo scopo di coprire il più
possibile il codice. Tale attività è stata svolta nelle ultime
iterazioni del processo, nel momento in cui il TCK ha dato un buon numero di
successi.

Il lavoro svolto, essendo completamente modulare e documentato oltre che dal
presente testo anche mediante una ricca documetazione JavaDoc, risulta
facilmente estendibile e modificabile. 

\section{Sviluppi futuri}
La specifica JSR-331, al momento dell'inizio del progetto, era in fase di 
valutazione da parte del Java Community Process. Durante la stesura del
presente testo la specifica è stata accettata ed è passata da uno stato di
richiesta allo stato di release finale. Tuttavia, come annunciato nel
blog (\cite{blog}) del Dr. Jacob Feldman, questo rappresenta una milestone del
progetto di standardizzazione e JSR-331 sarà sviluppato ulteriormente. Per tale
motivo uno dei lavori futuri sarà quello di mantenere allineata 
l'implementazione
basata su JSetL con le specifiche. Si può notare che
l'implementazione fornita è solo una delle possibili, sarebbe
possibile analizzare più in dettaglio il pacchetto JSetL e le sue funzionalità
al fine di migliorare l'implementazione attuale. 
\`E anche doveroso
sottolineare che durante lo sviluppo del progetto sono emerse funzionalità
richieste che si è dovuto aggiungere e quindi anche il JSR-331 
rappresenta uno spunto per migliorare JSetL stesso.
Inoltre essendo JSetL un progetto didattico, è soggetto naturalmente a frequenti
cambiamenti e sarà quindi importante studiare la possibilità di utilizzare
nuove funzionalità.
Infine si evidenzia la possibilità di produrre un report, in fase di stesura,
nella forma di un Rapporto Tecnico del Dipartimento di Matematica, che possa
guidare nell'implementazione delle specifiche anche altri solver.



