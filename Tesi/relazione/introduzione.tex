\hyphenation{
sil-la-ba-zio-ne
JSetL
Choco
JacoP
}

\chapter*{Introduzione\markboth{Introduzione}{}}
La \emph{programmazione a vincoli} (\emph{Constraint Programming} o \emph{CP}) 
è un paradigma che offre strumenti per modellare e risolvere 
efficacemente problemi di soddisfacimento ed ottimizzazione con vincoli.
Questa consiste nell'integrazione di vincoli all'interno di un linguaggio che 
funge da host. I primi linguaggi usati a tale scopo furono linguaggi di tipo 
logico, il più famoso è indubbiamente Prolog. Tali linguaggi differiscono 
fortemente da quelli imperativi come C, C++ o Java, poiché permettono al
programmatore di specificare cosa fare e non come farlo.

In questo lavoro di tesi ci si occuperà di implementare delle specifiche 
che consentono di fornire al linguaggio Java alcuni vantaggi del paradigma
dei linguaggi logici mediante l'implementazione dello standard JSR-331.

Java è un software multipiattaforma, ovvero progettato per funzionare su 
macchine
che possono essere di diversa natura. Tale piattaforma ha come caratteristica 
peculiare il fatto di rendere possibile la scrittura e l'esecuzione di 
applicazioni che siano indipendenti dall'hardware sul quale poi sono eseguite.

Di fatto, la portabilità, è quell'aspetto del linguaggio che ne ha
decretato il grande successo, insieme all'approccio object-oriented 
che consente un forte
 riuso del codice. Tali peculiarità sono valorizzate 
dall'introduzione delle API Java (Application Programming Interface), una 
collezione di componenti software  già scritti e pronti all'uso.

Lo sviluppo delle API è delegato ad una comunità di sviluppo aperta, essendo
il codice di Java open-source, denominata \emph{JCP} (\emph{Java Community 
Process}) che
detiene la responsabilità dello sviluppo della tecnologia Java guidando
l'implementazione e l'approvazione delle specifiche tecniche del linguaggio. 
Tali specifiche vengono richieste e descritte mediante i \emph{JSR}.

Con l'acronimo \emph{JSR} (\emph{Java Specification Requests}) 
si indicano
le descrizioni proposte o finali per le specifiche della piattaforma Java.
In qualsiasi momento vi sono più processi di revisione ed approvazione in 
corso, ogni processo viene denominato con la sigla JSR seguita da un numero 
identificativo (vedi \cite{specifiche}).

Il lavoro di tesi svolto si riferisce alle proposte relative al
JSR-331 in merito alla standardizzazione della programmazione a vincoli.
Partendo dalle specifiche richieste da tale documento si è realizzata 
un'interfaccia basata sulla libreria JSetL e sviluppata dal 
Dipartimento di Matematica dell'Università degli Studi di Parma.

Il lavoro di tesi è strutturato nel seguente modo:
%\vspace{5pt}

Il Capitolo \ref{capCP} introduce brevemente il concetto di programmazione
con vincoli e di soluzione di un problema, si fornisce quindi un semplice 
esempio.
%\vspace{5pt}

Il Capitolo \ref{capJSR} è dedicato al JSR-331, nel quale si 
riassumono i concetti principali del documento di specifica per quanto riguarda
gli obiettivi e la struttura dello standard proposto.
%\vspace{5pt}

Il Capitolo \ref{jsetl} descrive brevemente le funzionalità offerte dal solver
JSetL per quanto riguarda i costrutti utilizzati per la realizzazione
dell'implementazione concreta delle specifiche.
%\vspace{5pt}

Il Capitolo \ref{capImpl} riguarda l'implementazione vera e propria della parte
relativa alla definizione del problema. Ogni sezione del capitolo
è incentrata su un concetto ed una classe specifica, nella quale verrà 
introdotta
l'interfaccia, la classe realizzata e l'eventuale classe astratta
o implementazione di base, ove utilizzata.
%\vspace{5pt}

Il Capitolo \ref{capImpl2}, analogamente al precedente, descrive nel dettaglio
l'implementazione della parte  relativa alla risoluzione del problema.
%\vspace{5pt}

Nel Capitolo \ref{conclusioni} si darà spazio alle conclusioni del lavoro svolto
e ai possibili lavori futuri.
%\vspace{5pt}

Infine due appendici approfondiranno due aspetti affrontati
durante lo sviluppo. L'appendice \ref{cardinality} riguarda un vincolo la cui
trattazione non è stata banale, uno specifico vincolo di cardinalità richiesto
dalla specifica. L'appendice \ref{test}
invece riguarda i test svolti ed i relativi risultati.
