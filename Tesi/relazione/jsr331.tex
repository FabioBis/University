\hyphenation{
sil-la-ba-zio-ne
JSetL
Choco
JaCoP
}

\chapter{Java Specification Request 331}\label{capJSR}
JSR-331 è una \emph{richiesta per specifiche} Java in fase di sviluppo sotto
le regole redatte dal Java Community Process. Queste specifiche definiscono
le API per la programmazione a vincoli.

La specifica JSR-331 risponde alla necessità di ridurre i costi associati 
all'incorporazione di risolutori di vincoli (come JSetL, Choco, etc.) ad
applicazioni commerciali e non, operanti nel mondo reale. Esiste già un 
certo numero di fornitori di queste API, come già accennato possiamo
ricordare JSetL, Choco, JaCoP ed altri. Tuttavia le differenze tra questi
sono abbastanza significative da causare gravi difficoltà di utilizzo per 
gli sviluppatori di software.

\section{Obiettivi del JSR-331}
La standardizzazione della programmazione a vincoli si prefigge come scopo
quello di rendere tale tecnologia più accessibile per gli sviluppatori.
Avendo un'interfaccia unificata sarà possibile, per i programmatori, modellare
il problema in modo tale da poter provare la soluzione con più CP solver.
Questo minimizza la dipendenza da fornitori specifici, ma allo stesso
tempo non limita la possibilità di quest'ultimi nel procedere con lo sviluppo
del solver.

Gli obiettivi delle specifiche sono:
\begin{itemize}
\item[-]facilitare l'inserimento della tecnologia basata sui vincoli nelle 
applicazioni Java;
\item[-]aumentare la comunicazione e la standardizzazione tra i vari fornitori
di CP solver;
\item[-]incoraggiare il mercato delle applicazioni basate sulla programmazione
a vincoli e dei suoi strumenti mediante la standardizzazione delle suddette API;
\item[-]facilitare l'integrazione di tecniche basate sui vincoli in altri
JSR per supportare la programmazione dichiarativa;
\item[-]rendere le applicazioni Java più portabili tra vari fornitori di
risolutori di vincoli;
\item[-]fornire un modello per l'implementazione ed il supporto di
librerie di vincoli e strategie di ricerca per diverse applicazioni basate sulla
programmazione a vincoli;
\item[-] supportare i fornitori di solver offrendo API che vadano incontro alle
loro necessità e che siano di facile implementazione.
\end{itemize}

\subsection{A chi è rivolta?}
La specifica è rivolta principalmente a tre soggetti:
\begin{itemize}
\item[-]aziende che utilizzano le CP API
per sviluppare applicazioni di supporto a decisioni in ambito industriale;
\item[-]fornitori di risolutori di vincoli che intendono sviluppare e mantenere
la propria implementazione delle CP API;
\item[-]ricercatori in ambito di programmazione a vincoli che vogliono 
fornire o arricchire librerie di vincoli standard, algoritmi di ricerca e
problemi concreti che vengano mantenuti dalla CP community.
\end{itemize}

\subsection{Scopo della specifica}
Lo scopo della specifica JSR-331 è di definire un'interfaccia semplice da
utilizzare, leggera e che costituisca uno standard per acquisire ed utilizzare
risolutori di vincoli.

La specifica è mirata a piattaforme basate su Java ed è compatibile con JDK 1.5
o successivi.

L'ambito della specifica segue un approccio minimalista, con particolare cura
alla facilità d'uso. Ricopre i più comuni concetti dei problemi con vincoli e
della loro rappresentazione che è ormai già diventata una standardizzazione
di fatto, adottata dai solver e negli articoli scientifici. Tale ambito è
allo stesso tempo sufficientemente ampio da permettere agli sviluppatori di
applicazioni l'utilizzo delle interfacce standard per modellare
e risolvere tipici problemi di soddisfacimento di vincoli, all'interno
dei domini più comuni in ambito aziendale, come la pianificazione, l'allocazione
delle risorse e la configurazione.

I seguenti concetti ed oggetti rappresentano l'ambito iniziale delle
specifiche:
\begin{itemize}
\item[-]variabili vincolate dei tipi più utilizzati: interi, booleani, reali
ed insiemi di interi;
\item[-]vincoli unari e binari ed espressioni definite con variabili vincolate;
\item[-]i più comuni vincoli globali;
\item[-]la possibilità di ottenere una soluzione, ogni soluzione oppure
una soluzione ottimale, sotto certi limiti definiti dall'utente.
\end{itemize}

Ci si aspetta che le API vengano ampliate e così è anche specificato come
aggiungere nuovi concetti, funzionalità e strategie relative ai CP nel momento
in cui queste vengono comunemente adottate.

JSR-331 è focalizzato solo sull'interfaccia, per quanto riguarda le 
implementazioni non assume nessun approccio particolare. I seguenti concetti ed
oggetti sono prerogativa esclusiva dei vari solver e strumenti specifici:
\begin{itemize}
\item[-]meccanismi d'implementazione per i vari domini;
\item[-]implementazione dei più comuni vincoli binari e globali;
\item[-]meccanismi di propagazione di vincoli;
\item[-]backtracking e meccanismi di reversibilità.
\end{itemize}

\section{Struttura}
JSR-331 prescrive un insieme di operazioni fondamentali per definire e risolvere
problemi di soddisfacimento di vincoli e problemi di ottimizzazione.
La struttura consiste in tre principali componenti:
\begin{itemize}
\item[-]specifiche (CP API);
\item[-]implementazione basata su differenti CP solver;
\item[-]\emph{TCK} (\emph{Tecnology Compatibility Kit}), un pacchetto di test,
strumenti e
documentazione che verrà utilizzato per verificare la conformità alle
specifiche delle varie implementazioni.
\end{itemize}

Ogni implementazione del JSR-331 può implementare l'interfaccia direttamente
oppure può estendere le classi della \emph{common implementation} fornite
dalle API. La common implementation può essenzialmente semplificare
l'implementazione del JSR-331. Nel momento in cui lo standard diverrà più 
maturo e più
implementazioni adotteranno vincoli e strategie di ricerca comuni, queste
verranno gradualmente aggiunte alle librerie della common implementation.

\subsection{Specifiche}
Le specifiche per JSR-331 consistono in:
\begin{itemize}
\item[-]un'interfaccia pura (\files{javax.constraints});
\item[-]la common implementation (\files{javax.constraints.impl}).
\end{itemize}

\begin{itemize}
\item[-]\files{javax.constraints} 

Interfacce Java pure, come \files{Problem} o
\files{Solver}, che specificano i maggiori concetti e metodi per definire e 
risolvere problemi di soddisfacimento ed ottimizzazione di vincoli.

\item[-]\files{javax.constraints.impl}

Classi Java come \files{AbstractProblem} e 
\files{AbstractSolver} che implementano (parzialmente o completamente)
definizioni di problemi, concetti di risoluzione e metodi che non
dipendono direttamente da uno specifico CP solver.
\end{itemize}

\subsection{Implementazioni per JSR-331}
Ogni implementazione della specifica JSR-331 è basata su un 
CP solver (come ad esempio JSetL) e deve fornire le definizioni per tutte le
interfacce presenti nel pacchetto \files{javax.constraints}.
Alcune classi possono essere implementate direttamente dall'interfaccia
standard, altre si possono ottenere estendendo l'implementazione comune
fornita dal pacchetto \files{javax.constraints.impl}. 

JSR-331 richiede
che ogni implementazione fornisca almeno la definizione delle classi
presenti nei seguenti pacchetti:
\begin{itemize}
\item[-]\files{javax.constraints.impl}

Classi Java come \files{Problem} o \files{Solver} 
che forniscono
una implementazione finale per la definizione di problemi, concetti
e metodi. Queste classi possono essere derivate dall'implementazione
comune.

\item[-]\files{javax.constraints.impl.constraint}

Una libreria di vincoli che contiene
le implementazioni di vincoli basilari e globali che sono di fatto basati
sulle varie implementazioni dei CP solver.

\item[-]\files{javax.constraints.impl.search}

Una libreria di strategie di ricerca che contiene 
le implementazioni basate sui solver concreti.
\end{itemize}

In aggiunta ogni implementazione può fornire propri vincoli (nativi) e
proprie strategie di ricerca, assumendo che questi seguano le interfacce 
standard \files{javax.constraints.Constraint} e 
\files{javax.constraints.SearchStrategy}.

Il fatto che ogni implementazione per JSR-331 debba adottare gli stessi
nomi per i package, per le principali classi e metodi permetterà agli 
sviluppatori di applicazioni di passare facilmente da un solver ad
un altro (differenti implementazioni) senza  dover cambiare nulla nel proprio
codice. Possono di fatto scrivere applicazioni specifiche, motori basati su
vincoli una sola volta utilizzando le CP API e quindi utilizzare i differenti 
solver cambiando solo i file \files{.jar} all'interno del classpath.

\begin{nota}
La possibilità di cambiare i solver senza modificare il codice delle 
applicazioni ha comunque una limitazione: fissando  per convenzione i nomi
dei package si ha come effetto collaterale quello di non poter unire
le funzionalità di due o più solver differenti. La scelta di una determinata 
implementazione sottostante è definita dal solo file \files{.jar} nel
classpath dell'applicazione.
\end{nota}

\subsection{Tecnology Compatibility Kit}
Il \emph{TCK} (\emph{Tecnology Compatibility Kit}) è un pacchetto di 
documentazione, test e  strumenti utilizzati per testare la correttezza delle 
implementazioni per le specifiche di JSR-331.

Il TCK consiste in due package:
\begin{itemize}
\item[-]\files{org.jcp.jsr331.tests}

Contiene i moduli 
JUnit che permettono all'utente di validare automaticamente la correttezza
dell'implementazione con la specifica JSR-331.

\item[-]\files{org.jcp.jsr331.samples}

Contiene esempi di CSP che 
forniscono test integrati per i più comuni vincoli e
strategie di ricerca incluse in JSR-331, mostrano l'utilizzo della
programmazione a vincoli per problemi reali.
\end{itemize}

Non tutti i concetti introdotti nel package \files{javax.constraints} sono
richiesti per l'implementazione del solver al fine di essere validato.
Il pacchetto \files{org.jcp.jsr331.tests} contiene solo quei test che
sono normativi per la specifica (devono essere soddisfatti da ogni
implementazione al fine di essere validate). Il file 
\files{AllTests.java} all'interno del package contiene tutti i test normativi.

Il pacchetto \files{javax.constraints.impl} fornisce le implementazioni di
base per alcune interfacce e metodi opzionali. Ci sono due tipi di
implementazioni opzionali:
\begin{enumerate}
\item implementazioni di default (non le più efficienti) che possono
essere sovrascritte da una particolare implementazione per JSR-331;
\item semplici frammenti di codice che lanciano un'eccezione a runtime e che 
informano l'utente che quel metodo non è stato implementato dalla
particolare implementazione.
\end{enumerate}

\begin{nota}
Il package \files{org.jcp.jsr331.samples} è presente a soli fini dimostrativi e
non tutto ciò che è ivi incluso deve essere supportato da ogni implementazione.
\end{nota}

\subsection{Modello di sviluppo}
Il modello di sviluppo per le applicazioni finali che utilizzeranno le
API JSR-331 richiederà che i seguenti file \files{.jar} siano inclusi nel 
classpath:
\begin{itemize}
\item[-]\files{jsr331.jar}: include tutte le classi e le interfacce standard;
\item[-]\files{jsr331.<solver>.jar}: include tutte le classi della specifica
implementazione;
\item[-]\files{<solver>.jar}: include tutte le classi del solver utilizzato su 
cui si basa l'implementazione.
\end{itemize}
Ad esempio lo sviluppo basato su JSetL richiederà:
\begin{itemize}
\item[-]\files{jsr331.jar}
\item[-]\files{jsr331.jsetl.jar}
\item[-]\files{jsetl.jar}
\end{itemize}

\section{Rappresentazione di un CSP}
Come si è evidenziato nel capitolo \ref{capCP} un CSP è definito da un insieme
di variabili con il relativo dominio ed un insieme di vincoli; a queste 
variabili viene quindi ristretto il dominio per trovare una o più soluzioni.

JSR-331 definisce tutti i concetti Java per la \emph{rappresentazione} e la
\emph{risoluzione} di un problema di soddisfacimento di vincoli. Suddivide
quindi in maniera naturale un CSP in due parti fondamentali:
\begin{itemize}
\item[-]la \textbf{definizione} del problema, rappresentata dall'interfaccia
\files{Problem};
\item[-]la \textbf{soluzione} del problema, rappresentata dall'interfaccia
\files{Solver}. 
\end{itemize}  

Ogni concetto fondamentale di un CSP appartiene ad una delle due categorie, a
livello utente è possibile presentare un CSP mediante la seguente suddivisione:
\begin{enumerate}\samepage
\item Problema:
  \begin{itemize}
  \item[(a)]Variabili vincolate.
  \item[(b)]Vincoli.
  \end{itemize}
\item Risoluzione:
  \begin{itemize}
  \item[(c)]Strategia.
  \item[(d)]Soluzione.
  \end{itemize}
\end{enumerate}

Ogni CP solver utilizza nomi e rappresentazioni differenti per questi concetti,
che comunque sono semanticamente invarianti per la maggior parte di essi.
JSR-331 fornisce una nomenclatura unificata e specifiche dettagliate.

La definizione di un problema non conosce nulla a riguardo della sua 
risoluzione. Un'istanza della classe \files{Problem} può quindi esistere a
prescindere dall'esistenza di un'istanza di \files{Solver}. Non vale il
viceversa, poiché un'istanza della classe \files{Solver} può essere creata
solo partendo da uno specifico problema. Durante la ricerca di una soluzione
il solver può cambiare lo stato del problema (modifica dei domini delle 
variabili, semplificazione dei vincoli, etc). \`E quindi responsabilità
degli specifici risolutori mantenere (o meno) i possibili stati del problema in
base alla strategia di ricerca utilizzata.
